\hypertarget{group__gtx__norm}{
\section{GLM\_\-GTX\_\-norm}
\label{group__gtx__norm}\index{GLM\_\-GTX\_\-norm@{GLM\_\-GTX\_\-norm}}
}


Collaboration diagram for GLM\_\-GTX\_\-norm:Various ways to compute vector norms.  
\subsection*{Functions}
\begin{CompactItemize}
\item 
{\footnotesize template$<$typename T$>$ }\\GLM\_\-FUNC\_\-DECL T \hyperlink{group__gtx__norm_g6f970aba05e1299ed89d2ec3a410b7a9}{glm::length2} (T const \&x)
\item 
{\footnotesize template$<$typename genType$>$ }\\GLM\_\-FUNC\_\-DECL genType::value\_\-type \hyperlink{group__gtx__norm_g2dfcdff0cc9119aa37c501d2c7a45020}{glm::length2} (genType const \&x)
\item 
{\footnotesize template$<$typename T$>$ }\\GLM\_\-FUNC\_\-DECL T \hyperlink{group__gtx__norm_gaf6befa643aa9616f3c19e5548b11b54}{glm::distance2} (T const \&p0, T const \&p1)
\item 
{\footnotesize template$<$typename genType$>$ }\\GLM\_\-FUNC\_\-DECL genType::value\_\-type \hyperlink{group__gtx__norm_g205e08f24b9e35f9f892b563f2b8dd94}{glm::distance2} (genType const \&p0, genType const \&p1)
\item 
{\footnotesize template$<$typename T, precision P$>$ }\\GLM\_\-FUNC\_\-DECL T \hyperlink{group__gtx__norm_gf20fc187dfe66a474ecdba26e32ce4c6}{glm::l1Norm} (detail::tvec3$<$ T, P $>$ const \&x, detail::tvec3$<$ T, P $>$ const \&y)
\item 
{\footnotesize template$<$typename T, precision P$>$ }\\GLM\_\-FUNC\_\-DECL T \hyperlink{group__gtx__norm_g5a6d616a6e0340b9329d5018f04c1aca}{glm::l1Norm} (detail::tvec3$<$ T, P $>$ const \&v)
\item 
{\footnotesize template$<$typename T, precision P$>$ }\\GLM\_\-FUNC\_\-DECL T \hyperlink{group__gtx__norm_g2c87568b2888f01baf0133320c09b8ca}{glm::l2Norm} (detail::tvec3$<$ T, P $>$ const \&x, detail::tvec3$<$ T, P $>$ const \&y)
\item 
{\footnotesize template$<$typename T, precision P$>$ }\\GLM\_\-FUNC\_\-DECL T \hyperlink{group__gtx__norm_g17bb46915f9694cbae612a7bae4c5116}{glm::l2Norm} (detail::tvec3$<$ T, P $>$ const \&x)
\item 
{\footnotesize template$<$typename T, precision P$>$ }\\GLM\_\-FUNC\_\-DECL T \hyperlink{group__gtx__norm_g2f42190c8743abab279d0a8f5a321692}{glm::lxNorm} (detail::tvec3$<$ T, P $>$ const \&x, detail::tvec3$<$ T, P $>$ const \&y, unsigned int Depth)
\item 
{\footnotesize template$<$typename T, precision P$>$ }\\GLM\_\-FUNC\_\-DECL T \hyperlink{group__gtx__norm_g955869c61ab902e4e3cf061303efdaef}{glm::lxNorm} (detail::tvec3$<$ T, P $>$ const \&x, unsigned int Depth)
\end{CompactItemize}


\subsection{Detailed Description}
Various ways to compute vector norms. 

$<$glm/gtx/norm.hpp$>$ need to be included to use these functionalities. 

\subsection{Function Documentation}
\hypertarget{group__gtx__norm_g205e08f24b9e35f9f892b563f2b8dd94}{
\index{gtx\_\-norm@{gtx\_\-norm}!distance2@{distance2}}
\index{distance2@{distance2}!gtx_norm@{gtx\_\-norm}}
\subsubsection[distance2]{\setlength{\rightskip}{0pt plus 5cm}template$<$typename genType$>$ GLM\_\-FUNC\_\-DECL genType::value\_\-type glm::distance2 (genType const \& {\em p0}, \/  genType const \& {\em p1})\hspace{0.3cm}{\tt  \mbox{[}inline\mbox{]}}}}
\label{group__gtx__norm_g205e08f24b9e35f9f892b563f2b8dd94}


Returns the squared distance between p0 and p1, i.e., length(p0 - p1). From GLM\_\-GTX\_\-norm extension. 

Referenced by btGjkPairDetector::getClosestPointsNonVirtual().

Here is the caller graph for this function:\hypertarget{group__gtx__norm_gaf6befa643aa9616f3c19e5548b11b54}{
\index{gtx\_\-norm@{gtx\_\-norm}!distance2@{distance2}}
\index{distance2@{distance2}!gtx_norm@{gtx\_\-norm}}
\subsubsection[distance2]{\setlength{\rightskip}{0pt plus 5cm}template$<$typename T$>$ GLM\_\-FUNC\_\-QUALIFIER T glm::distance2 (T const \& {\em p0}, \/  T const \& {\em p1})\hspace{0.3cm}{\tt  \mbox{[}inline\mbox{]}}}}
\label{group__gtx__norm_gaf6befa643aa9616f3c19e5548b11b54}


Returns the squared distance between p0 and p1, i.e., length(p0 - p1). From GLM\_\-GTX\_\-norm extension. 

Definition at line 50 of file norm.inl.

References glm::length2().

\begin{Code}\begin{verbatim}54         {
55                 return length2(p1 - p0);
56         }
\end{verbatim}
\end{Code}




Here is the call graph for this function:\hypertarget{group__gtx__norm_g5a6d616a6e0340b9329d5018f04c1aca}{
\index{gtx\_\-norm@{gtx\_\-norm}!l1Norm@{l1Norm}}
\index{l1Norm@{l1Norm}!gtx_norm@{gtx\_\-norm}}
\subsubsection[l1Norm]{\setlength{\rightskip}{0pt plus 5cm}template$<$typename T, precision P$>$ GLM\_\-FUNC\_\-QUALIFIER T glm::l1Norm (detail::tvec3$<$ T, P $>$ const \& {\em v})\hspace{0.3cm}{\tt  \mbox{[}inline\mbox{]}}}}
\label{group__gtx__norm_g5a6d616a6e0340b9329d5018f04c1aca}


Returns the L1 norm of v. From GLM\_\-GTX\_\-norm extension. 

Definition at line 100 of file norm.inl.

References glm::abs().

\begin{Code}\begin{verbatim}103         {
104                 return abs(v.x) + abs(v.y) + abs(v.z);
105         }
\end{verbatim}
\end{Code}




Here is the call graph for this function:\hypertarget{group__gtx__norm_gf20fc187dfe66a474ecdba26e32ce4c6}{
\index{gtx\_\-norm@{gtx\_\-norm}!l1Norm@{l1Norm}}
\index{l1Norm@{l1Norm}!gtx_norm@{gtx\_\-norm}}
\subsubsection[l1Norm]{\setlength{\rightskip}{0pt plus 5cm}template$<$typename T, precision P$>$ GLM\_\-FUNC\_\-QUALIFIER T glm::l1Norm (detail::tvec3$<$ T, P $>$ const \& {\em x}, \/  detail::tvec3$<$ T, P $>$ const \& {\em y})\hspace{0.3cm}{\tt  \mbox{[}inline\mbox{]}}}}
\label{group__gtx__norm_gf20fc187dfe66a474ecdba26e32ce4c6}


Returns the L1 norm between x and y. From GLM\_\-GTX\_\-norm extension. 

Definition at line 90 of file norm.inl.

References glm::abs().

\begin{Code}\begin{verbatim}94         {
95                 return abs(b.x - a.x) + abs(b.y - a.y) + abs(b.z - a.z);
96         }
\end{verbatim}
\end{Code}




Here is the call graph for this function:\hypertarget{group__gtx__norm_g17bb46915f9694cbae612a7bae4c5116}{
\index{gtx\_\-norm@{gtx\_\-norm}!l2Norm@{l2Norm}}
\index{l2Norm@{l2Norm}!gtx_norm@{gtx\_\-norm}}
\subsubsection[l2Norm]{\setlength{\rightskip}{0pt plus 5cm}template$<$typename T, precision P$>$ GLM\_\-FUNC\_\-QUALIFIER T glm::l2Norm (detail::tvec3$<$ T, P $>$ const \& {\em x})\hspace{0.3cm}{\tt  \mbox{[}inline\mbox{]}}}}
\label{group__gtx__norm_g17bb46915f9694cbae612a7bae4c5116}


Returns the L2 norm of v. From GLM\_\-GTX\_\-norm extension. 

Definition at line 119 of file norm.inl.

References glm::length().

\begin{Code}\begin{verbatim}122         {
123                 return length(v);
124         }
\end{verbatim}
\end{Code}




Here is the call graph for this function:\hypertarget{group__gtx__norm_g2c87568b2888f01baf0133320c09b8ca}{
\index{gtx\_\-norm@{gtx\_\-norm}!l2Norm@{l2Norm}}
\index{l2Norm@{l2Norm}!gtx_norm@{gtx\_\-norm}}
\subsubsection[l2Norm]{\setlength{\rightskip}{0pt plus 5cm}template$<$typename T, precision P$>$ GLM\_\-FUNC\_\-QUALIFIER T glm::l2Norm (detail::tvec3$<$ T, P $>$ const \& {\em x}, \/  detail::tvec3$<$ T, P $>$ const \& {\em y})\hspace{0.3cm}{\tt  \mbox{[}inline\mbox{]}}}}
\label{group__gtx__norm_g2c87568b2888f01baf0133320c09b8ca}


Returns the L2 norm between x and y. From GLM\_\-GTX\_\-norm extension. 

Definition at line 109 of file norm.inl.

References glm::length().

\begin{Code}\begin{verbatim}113         {
114                 return length(b - a);
115         }
\end{verbatim}
\end{Code}




Here is the call graph for this function:\hypertarget{group__gtx__norm_g2dfcdff0cc9119aa37c501d2c7a45020}{
\index{gtx\_\-norm@{gtx\_\-norm}!length2@{length2}}
\index{length2@{length2}!gtx_norm@{gtx\_\-norm}}
\subsubsection[length2]{\setlength{\rightskip}{0pt plus 5cm}template$<$typename genType$>$ GLM\_\-FUNC\_\-DECL genType::value\_\-type glm::length2 (genType const \& {\em x})\hspace{0.3cm}{\tt  \mbox{[}inline\mbox{]}}}}
\label{group__gtx__norm_g2dfcdff0cc9119aa37c501d2c7a45020}


Returns the squared length of x. From GLM\_\-GTX\_\-norm extension. \hypertarget{group__gtx__norm_g6f970aba05e1299ed89d2ec3a410b7a9}{
\index{gtx\_\-norm@{gtx\_\-norm}!length2@{length2}}
\index{length2@{length2}!gtx_norm@{gtx\_\-norm}}
\subsubsection[length2]{\setlength{\rightskip}{0pt plus 5cm}template$<$typename T$>$ GLM\_\-FUNC\_\-QUALIFIER T glm::length2 (T const \& {\em x})\hspace{0.3cm}{\tt  \mbox{[}inline\mbox{]}}}}
\label{group__gtx__norm_g6f970aba05e1299ed89d2ec3a410b7a9}


Returns the squared length of x. From GLM\_\-GTX\_\-norm extension. 

Definition at line 14 of file norm.inl.

Referenced by glm::distance2(), glm::mat3x4\_\-cast(), and glm::rotation().

\begin{Code}\begin{verbatim}17         {
18                 return x * x;
19         }
\end{verbatim}
\end{Code}




Here is the caller graph for this function:\hypertarget{group__gtx__norm_g955869c61ab902e4e3cf061303efdaef}{
\index{gtx\_\-norm@{gtx\_\-norm}!lxNorm@{lxNorm}}
\index{lxNorm@{lxNorm}!gtx_norm@{gtx\_\-norm}}
\subsubsection[lxNorm]{\setlength{\rightskip}{0pt plus 5cm}template$<$typename T, precision P$>$ GLM\_\-FUNC\_\-QUALIFIER T glm::lxNorm (detail::tvec3$<$ T, P $>$ const \& {\em x}, \/  unsigned int {\em Depth})\hspace{0.3cm}{\tt  \mbox{[}inline\mbox{]}}}}
\label{group__gtx__norm_g955869c61ab902e4e3cf061303efdaef}


Returns the L norm of v. From GLM\_\-GTX\_\-norm extension. 

Definition at line 139 of file norm.inl.

References glm::pow().

\begin{Code}\begin{verbatim}143         {
144                 return pow(pow(v.x, T(Depth)) + pow(v.y, T(Depth)) + pow(v.z, T(Depth)), T(1) / T(Depth));
145         }
\end{verbatim}
\end{Code}




Here is the call graph for this function:\hypertarget{group__gtx__norm_g2f42190c8743abab279d0a8f5a321692}{
\index{gtx\_\-norm@{gtx\_\-norm}!lxNorm@{lxNorm}}
\index{lxNorm@{lxNorm}!gtx_norm@{gtx\_\-norm}}
\subsubsection[lxNorm]{\setlength{\rightskip}{0pt plus 5cm}template$<$typename T, precision P$>$ GLM\_\-FUNC\_\-QUALIFIER T glm::lxNorm (detail::tvec3$<$ T, P $>$ const \& {\em x}, \/  detail::tvec3$<$ T, P $>$ const \& {\em y}, \/  unsigned int {\em Depth})\hspace{0.3cm}{\tt  \mbox{[}inline\mbox{]}}}}
\label{group__gtx__norm_g2f42190c8743abab279d0a8f5a321692}


Returns the L norm between x and y. From GLM\_\-GTX\_\-norm extension. 

Definition at line 128 of file norm.inl.

References glm::pow().

\begin{Code}\begin{verbatim}133         {
134                 return pow(pow(y.x - x.x, T(Depth)) + pow(y.y - x.y, T(Depth)) + pow(y.z - x.z, T(Depth)), T(1) / T(Depth));
135         }
\end{verbatim}
\end{Code}




Here is the call graph for this function: