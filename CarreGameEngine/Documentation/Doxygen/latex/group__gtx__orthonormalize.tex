\hypertarget{group__gtx__orthonormalize}{
\section{GLM\_\-GTX\_\-orthonormalize}
\label{group__gtx__orthonormalize}\index{GLM\_\-GTX\_\-orthonormalize@{GLM\_\-GTX\_\-orthonormalize}}
}


Collaboration diagram for GLM\_\-GTX\_\-orthonormalize:Orthonormalize matrices.  
\subsection*{Functions}
\begin{CompactItemize}
\item 
{\footnotesize template$<$typename T, precision P$>$ }\\GLM\_\-FUNC\_\-DECL detail::tmat3x3$<$ T, P $>$ \hyperlink{group__gtx__orthonormalize_ge0c06d8312a13b38747271ea68f00121}{glm::orthonormalize} (const detail::tmat3x3$<$ T, P $>$ \&m)
\item 
{\footnotesize template$<$typename T, precision P$>$ }\\GLM\_\-FUNC\_\-DECL detail::tvec3$<$ T, P $>$ \hyperlink{group__gtx__orthonormalize_g6e5ecb7642087e4ff470c8f068c903be}{glm::orthonormalize} (const detail::tvec3$<$ T, P $>$ \&x, const detail::tvec3$<$ T, P $>$ \&y)
\end{CompactItemize}


\subsection{Detailed Description}
Orthonormalize matrices. 

$<$glm/gtx/orthonormalize.hpp$>$ need to be included to use these functionalities. 

\subsection{Function Documentation}
\hypertarget{group__gtx__orthonormalize_g6e5ecb7642087e4ff470c8f068c903be}{
\index{gtx\_\-orthonormalize@{gtx\_\-orthonormalize}!orthonormalize@{orthonormalize}}
\index{orthonormalize@{orthonormalize}!gtx_orthonormalize@{gtx\_\-orthonormalize}}
\subsubsection[orthonormalize]{\setlength{\rightskip}{0pt plus 5cm}template$<$typename T, precision P$>$ GLM\_\-FUNC\_\-QUALIFIER detail::tvec3$<$ T, P $>$ glm::orthonormalize (const detail::tvec3$<$ T, P $>$ \& {\em x}, \/  const detail::tvec3$<$ T, P $>$ \& {\em y})\hspace{0.3cm}{\tt  \mbox{[}inline\mbox{]}}}}
\label{group__gtx__orthonormalize_g6e5ecb7642087e4ff470c8f068c903be}


Orthonormalizes x according y. From GLM\_\-GTX\_\-orthonormalize extension. 

Definition at line 36 of file orthonormalize.inl.

References glm::dot(), and glm::normalize().

\begin{Code}\begin{verbatim}40         {
41                 return normalize(x - y * dot(y, x));
42         }
\end{verbatim}
\end{Code}




Here is the call graph for this function:\hypertarget{group__gtx__orthonormalize_ge0c06d8312a13b38747271ea68f00121}{
\index{gtx\_\-orthonormalize@{gtx\_\-orthonormalize}!orthonormalize@{orthonormalize}}
\index{orthonormalize@{orthonormalize}!gtx_orthonormalize@{gtx\_\-orthonormalize}}
\subsubsection[orthonormalize]{\setlength{\rightskip}{0pt plus 5cm}template$<$typename T, precision P$>$ GLM\_\-FUNC\_\-QUALIFIER detail::tmat3x3$<$ T, P $>$ glm::orthonormalize (const detail::tmat3x3$<$ T, P $>$ \& {\em m})\hspace{0.3cm}{\tt  \mbox{[}inline\mbox{]}}}}
\label{group__gtx__orthonormalize_ge0c06d8312a13b38747271ea68f00121}


Returns the orthonormalized matrix of m. From GLM\_\-GTX\_\-orthonormalize extension. 

Definition at line 14 of file orthonormalize.inl.

References glm::dot(), and glm::normalize().

\begin{Code}\begin{verbatim}17         {
18                 detail::tmat3x3<T, P> r = m;
19 
20                 r[0] = normalize(r[0]);
21 
22                 float d0 = dot(r[0], r[1]);
23                 r[1] -= r[0] * d0;
24                 r[1] = normalize(r[1]);
25 
26                 float d1 = dot(r[1], r[2]);
27                 d0 = dot(r[0], r[2]);
28                 r[2] -= r[0] * d0 + r[1] * d1;
29                 r[2] = normalize(r[2]);
30 
31                 return r;
32         }
\end{verbatim}
\end{Code}




Here is the call graph for this function: