\hypertarget{group__gtx__closest__point}{
\section{GLM\_\-GTX\_\-closest\_\-point}
\label{group__gtx__closest__point}\index{GLM\_\-GTX\_\-closest\_\-point@{GLM\_\-GTX\_\-closest\_\-point}}
}


Collaboration diagram for GLM\_\-GTX\_\-closest\_\-point:Find the point on a straight line which is the closet of a point.  
\subsection*{Functions}
\begin{CompactItemize}
\item 
{\footnotesize template$<$typename T, precision P$>$ }\\GLM\_\-FUNC\_\-DECL detail::tvec3$<$ T, P $>$ \hyperlink{group__gtx__closest__point_g2fe2729eb32015953823c96c2e15daf9}{glm::closestPointOnLine} (detail::tvec3$<$ T, P $>$ const \&point, detail::tvec3$<$ T, P $>$ const \&a, detail::tvec3$<$ T, P $>$ const \&b)
\end{CompactItemize}


\subsection{Detailed Description}
Find the point on a straight line which is the closet of a point. 

$<$glm/gtx/closest\_\-point.hpp$>$ need to be included to use these functionalities. 

\subsection{Function Documentation}
\hypertarget{group__gtx__closest__point_g2fe2729eb32015953823c96c2e15daf9}{
\index{gtx\_\-closest\_\-point@{gtx\_\-closest\_\-point}!closestPointOnLine@{closestPointOnLine}}
\index{closestPointOnLine@{closestPointOnLine}!gtx_closest_point@{gtx\_\-closest\_\-point}}
\subsubsection[closestPointOnLine]{\setlength{\rightskip}{0pt plus 5cm}template$<$typename T, precision P$>$ GLM\_\-FUNC\_\-QUALIFIER detail::tvec3$<$ T, P $>$ glm::closestPointOnLine (detail::tvec3$<$ T, P $>$ const \& {\em point}, \/  detail::tvec3$<$ T, P $>$ const \& {\em a}, \/  detail::tvec3$<$ T, P $>$ const \& {\em b})\hspace{0.3cm}{\tt  \mbox{[}inline\mbox{]}}}}
\label{group__gtx__closest__point_g2fe2729eb32015953823c96c2e15daf9}


Find the point on a straight line which is the closet of a point. \begin{Desc}
\item[See also:]\hyperlink{group__gtx__closest__point}{GLM\_\-GTX\_\-closest\_\-point} \end{Desc}


Definition at line 17 of file closest\_\-point.inl.

References glm::distance(), and glm::dot().

\begin{Code}\begin{verbatim}22         {
23                 T LineLength = distance(a, b);
24                 detail::tvec3<T, P> Vector = point - a;
25                 detail::tvec3<T, P> LineDirection = (b - a) / LineLength;
26 
27                 // Project Vector to LineDirection to get the distance of point from a
28                 T Distance = dot(Vector, LineDirection);
29 
30                 if(Distance <= T(0)) return a;
31                 if(Distance >= LineLength) return b;
32                 return a + LineDirection * Distance;
33         }
\end{verbatim}
\end{Code}




Here is the call graph for this function: