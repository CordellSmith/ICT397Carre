\hypertarget{group__gtx__perpendicular}{
\section{GLM\_\-GTX\_\-perpendicular}
\label{group__gtx__perpendicular}\index{GLM\_\-GTX\_\-perpendicular@{GLM\_\-GTX\_\-perpendicular}}
}


Collaboration diagram for GLM\_\-GTX\_\-perpendicular:Perpendicular of a vector from other one.  
\subsection*{Functions}
\begin{CompactItemize}
\item 
{\footnotesize template$<$typename vecType$>$ }\\GLM\_\-FUNC\_\-DECL vecType \hyperlink{group__gtx__perpendicular_gcd6201d43400cf027df57552bf92301d}{glm::perp} (vecType const \&x, vecType const \&Normal)
\end{CompactItemize}


\subsection{Detailed Description}
Perpendicular of a vector from other one. 

$<$glm/gtx/perpendicular.hpp$>$ need to be included to use these functionalities. 

\subsection{Function Documentation}
\hypertarget{group__gtx__perpendicular_gcd6201d43400cf027df57552bf92301d}{
\index{gtx\_\-perpendicular@{gtx\_\-perpendicular}!perp@{perp}}
\index{perp@{perp}!gtx_perpendicular@{gtx\_\-perpendicular}}
\subsubsection[perp]{\setlength{\rightskip}{0pt plus 5cm}template$<$typename vecType$>$ GLM\_\-FUNC\_\-QUALIFIER vecType glm::perp (vecType const \& {\em x}, \/  vecType const \& {\em Normal})\hspace{0.3cm}{\tt  \mbox{[}inline\mbox{]}}}}
\label{group__gtx__perpendicular_gcd6201d43400cf027df57552bf92301d}


Projects x a perpendicular axis of Normal. From GLM\_\-GTX\_\-perpendicular extension. 

Definition at line 14 of file perpendicular.inl.

References glm::proj().

\begin{Code}\begin{verbatim}18         {
19                 return x - proj(x, Normal);
20         }
\end{verbatim}
\end{Code}




Here is the call graph for this function: