\hypertarget{group__gtc__random}{
\section{GLM\_\-GTC\_\-random}
\label{group__gtc__random}\index{GLM\_\-GTC\_\-random@{GLM\_\-GTC\_\-random}}
}


Collaboration diagram for GLM\_\-GTC\_\-random:Generate random number from various distribution methods.  
\subsection*{Functions}
\begin{CompactItemize}
\item 
{\footnotesize template$<$typename genType$>$ }\\GLM\_\-FUNC\_\-DECL genType \hyperlink{group__gtc__random_gb955b990407d8d4b62cfe2a2a89d9492}{glm::linearRand} (genType const \&Min, genType const \&Max)
\item 
{\footnotesize template$<$typename genType$>$ }\\GLM\_\-FUNC\_\-DECL genType \hyperlink{group__gtc__random_gc045638daf634b0221ef4baaaf643cb2}{glm::gaussRand} (genType const \&Mean, genType const \&Deviation)
\item 
{\footnotesize template$<$typename T$>$ }\\GLM\_\-FUNC\_\-DECL detail::tvec2$<$ T, defaultp $>$ \hyperlink{group__gtc__random_gb633996dfedab1a7f45bb5cfa1b14443}{glm::circularRand} (T const \&Radius)
\item 
{\footnotesize template$<$typename T$>$ }\\GLM\_\-FUNC\_\-DECL detail::tvec3$<$ T, defaultp $>$ \hyperlink{group__gtc__random_g02f5a6b98ae4e494eef340bcdd38ed35}{glm::sphericalRand} (T const \&Radius)
\item 
{\footnotesize template$<$typename T$>$ }\\GLM\_\-FUNC\_\-DECL detail::tvec2$<$ T, defaultp $>$ \hyperlink{group__gtc__random_gea707507f941467133dfdd4f3de9eeed}{glm::diskRand} (T const \&Radius)
\item 
{\footnotesize template$<$typename T$>$ }\\GLM\_\-FUNC\_\-DECL detail::tvec3$<$ T, defaultp $>$ \hyperlink{group__gtc__random_g8543dd05af529c33cb10deb832aab03a}{glm::ballRand} (T const \&Radius)
\end{CompactItemize}


\subsection{Detailed Description}
Generate random number from various distribution methods. 

$<$glm/gtc/random.hpp$>$ need to be included to use these functionalities. 

\subsection{Function Documentation}
\hypertarget{group__gtc__random_g8543dd05af529c33cb10deb832aab03a}{
\index{gtc\_\-random@{gtc\_\-random}!ballRand@{ballRand}}
\index{ballRand@{ballRand}!gtc_random@{gtc\_\-random}}
\subsubsection[ballRand]{\setlength{\rightskip}{0pt plus 5cm}template$<$typename T$>$ GLM\_\-FUNC\_\-QUALIFIER detail::tvec3$<$ T, defaultp $>$ glm::ballRand (T const \& {\em Radius})\hspace{0.3cm}{\tt  \mbox{[}inline\mbox{]}}}}
\label{group__gtc__random_g8543dd05af529c33cb10deb832aab03a}


Generate a random 3D vector which coordinates are regulary distributed within the volume of a ball of a given radius

\begin{Desc}
\item[Parameters:]
\begin{description}
\item[{\em Radius}]\end{description}
\end{Desc}
\begin{Desc}
\item[See also:]\hyperlink{group__gtc__random}{GLM\_\-GTC\_\-random} \end{Desc}


Definition at line 126 of file random.inl.

References glm::length(), and glm::linearRand().

\begin{Code}\begin{verbatim}129         {               
130                 detail::tvec3<T, defaultp> Result(T(0));
131                 T LenRadius(T(0));
132                 
133                 do
134                 {
135                         Result = linearRand(
136                                 detail::tvec3<T, defaultp>(-Radius),
137                                 detail::tvec3<T, defaultp>(Radius));
138                         LenRadius = length(Result);
139                 }
140                 while(LenRadius > Radius);
141                 
142                 return Result;
143         }
\end{verbatim}
\end{Code}




Here is the call graph for this function:\hypertarget{group__gtc__random_gb633996dfedab1a7f45bb5cfa1b14443}{
\index{gtc\_\-random@{gtc\_\-random}!circularRand@{circularRand}}
\index{circularRand@{circularRand}!gtc_random@{gtc\_\-random}}
\subsubsection[circularRand]{\setlength{\rightskip}{0pt plus 5cm}template$<$typename T$>$ GLM\_\-FUNC\_\-QUALIFIER detail::tvec2$<$ T, defaultp $>$ glm::circularRand (T const \& {\em Radius})\hspace{0.3cm}{\tt  \mbox{[}inline\mbox{]}}}}
\label{group__gtc__random_gb633996dfedab1a7f45bb5cfa1b14443}


Generate a random 2D vector which coordinates are regulary distributed on a circle of a given radius

\begin{Desc}
\item[Parameters:]
\begin{description}
\item[{\em Radius}]\end{description}
\end{Desc}
\begin{Desc}
\item[See also:]\hyperlink{group__gtc__random}{GLM\_\-GTC\_\-random} \end{Desc}


Definition at line 147 of file random.inl.

References glm::cos(), glm::linearRand(), and glm::sin().

\begin{Code}\begin{verbatim}150         {
151                 T a = linearRand(T(0), T(6.283185307179586476925286766559f));
152                 return detail::tvec2<T, defaultp>(cos(a), sin(a)) * Radius;             
153         }
\end{verbatim}
\end{Code}




Here is the call graph for this function:\hypertarget{group__gtc__random_gea707507f941467133dfdd4f3de9eeed}{
\index{gtc\_\-random@{gtc\_\-random}!diskRand@{diskRand}}
\index{diskRand@{diskRand}!gtc_random@{gtc\_\-random}}
\subsubsection[diskRand]{\setlength{\rightskip}{0pt plus 5cm}template$<$typename T$>$ GLM\_\-FUNC\_\-QUALIFIER detail::tvec2$<$ T, defaultp $>$ glm::diskRand (T const \& {\em Radius})\hspace{0.3cm}{\tt  \mbox{[}inline\mbox{]}}}}
\label{group__gtc__random_gea707507f941467133dfdd4f3de9eeed}


Generate a random 2D vector which coordinates are regulary distributed within the area of a disk of a given radius

\begin{Desc}
\item[Parameters:]
\begin{description}
\item[{\em Radius}]\end{description}
\end{Desc}
\begin{Desc}
\item[See also:]\hyperlink{group__gtc__random}{GLM\_\-GTC\_\-random} \end{Desc}


Definition at line 105 of file random.inl.

References glm::length(), and glm::linearRand().

\begin{Code}\begin{verbatim}108         {               
109                 detail::tvec2<T, defaultp> Result(T(0));
110                 T LenRadius(T(0));
111                 
112                 do
113                 {
114                         Result = linearRand(
115                                 detail::tvec2<T, defaultp>(-Radius),
116                                 detail::tvec2<T, defaultp>(Radius));
117                         LenRadius = length(Result);
118                 }
119                 while(LenRadius > Radius);
120                 
121                 return Result;
122         }
\end{verbatim}
\end{Code}




Here is the call graph for this function:\hypertarget{group__gtc__random_gc045638daf634b0221ef4baaaf643cb2}{
\index{gtc\_\-random@{gtc\_\-random}!gaussRand@{gaussRand}}
\index{gaussRand@{gaussRand}!gtc_random@{gtc\_\-random}}
\subsubsection[gaussRand]{\setlength{\rightskip}{0pt plus 5cm}template$<$typename genType$>$ GLM\_\-FUNC\_\-QUALIFIER genType glm::gaussRand (genType const \& {\em Mean}, \/  genType const \& {\em Deviation})\hspace{0.3cm}{\tt  \mbox{[}inline\mbox{]}}}}
\label{group__gtc__random_gc045638daf634b0221ef4baaaf643cb2}


Generate random numbers in the interval \mbox{[}Min, Max\mbox{]}, according a gaussian distribution

\begin{Desc}
\item[Parameters:]
\begin{description}
\item[{\em Mean}]\item[{\em Deviation}]\end{description}
\end{Desc}
\begin{Desc}
\item[See also:]\hyperlink{group__gtc__random}{GLM\_\-GTC\_\-random} \end{Desc}


Definition at line 83 of file random.inl.

References glm::linearRand(), glm::log(), and glm::sqrt().

\begin{Code}\begin{verbatim}87         {
88                 genType w, x1, x2;
89         
90                 do
91                 {
92                         x1 = linearRand(genType(-1), genType(1));
93                         x2 = linearRand(genType(-1), genType(1));
94                 
95                         w = x1 * x1 + x2 * x2;
96                 } while(w > genType(1));
97         
98                 return x2 * Deviation * Deviation * sqrt((genType(-2) * log(w)) / w) + Mean;
99         }
\end{verbatim}
\end{Code}




Here is the call graph for this function:\hypertarget{group__gtc__random_gb955b990407d8d4b62cfe2a2a89d9492}{
\index{gtc\_\-random@{gtc\_\-random}!linearRand@{linearRand}}
\index{linearRand@{linearRand}!gtc_random@{gtc\_\-random}}
\subsubsection[linearRand]{\setlength{\rightskip}{0pt plus 5cm}template$<$typename genType$>$ GLM\_\-FUNC\_\-QUALIFIER genType glm::linearRand (genType const \& {\em Min}, \/  genType const \& {\em Max})\hspace{0.3cm}{\tt  \mbox{[}inline\mbox{]}}}}
\label{group__gtc__random_gb955b990407d8d4b62cfe2a2a89d9492}


Generate random numbers in the interval \mbox{[}Min, Max\mbox{]}, according a linear distribution

\begin{Desc}
\item[Parameters:]
\begin{description}
\item[{\em Min}]\item[{\em Max}]\end{description}
\end{Desc}
\begin{Desc}
\item[Template Parameters:]
\begin{description}
\item[{\em genType}]Value type. Currently supported: half (not recommanded), float or double scalars and vectors. \end{description}
\end{Desc}
\begin{Desc}
\item[See also:]\hyperlink{group__gtc__random}{GLM\_\-GTC\_\-random} \end{Desc}


Definition at line 71 of file random.inl.

Referenced by glm::ballRand(), glm::circularRand(), glm::diskRand(), glm::gaussRand(), and glm::sphericalRand().

\begin{Code}\begin{verbatim}75         {
76                 return detail::compute_linearRand()(Min, Max);
77         }
\end{verbatim}
\end{Code}




Here is the caller graph for this function:\hypertarget{group__gtc__random_g02f5a6b98ae4e494eef340bcdd38ed35}{
\index{gtc\_\-random@{gtc\_\-random}!sphericalRand@{sphericalRand}}
\index{sphericalRand@{sphericalRand}!gtc_random@{gtc\_\-random}}
\subsubsection[sphericalRand]{\setlength{\rightskip}{0pt plus 5cm}template$<$typename T$>$ GLM\_\-FUNC\_\-QUALIFIER detail::tvec3$<$ T, defaultp $>$ glm::sphericalRand (T const \& {\em Radius})\hspace{0.3cm}{\tt  \mbox{[}inline\mbox{]}}}}
\label{group__gtc__random_g02f5a6b98ae4e494eef340bcdd38ed35}


Generate a random 3D vector which coordinates are regulary distributed on a sphere of a given radius

\begin{Desc}
\item[Parameters:]
\begin{description}
\item[{\em Radius}]\end{description}
\end{Desc}
\begin{Desc}
\item[See also:]\hyperlink{group__gtc__random}{GLM\_\-GTC\_\-random} \end{Desc}


Definition at line 157 of file random.inl.

References glm::cos(), glm::linearRand(), glm::sin(), and glm::sqrt().

\begin{Code}\begin{verbatim}160         {
161                 T z = linearRand(T(-1), T(1));
162                 T a = linearRand(T(0), T(6.283185307179586476925286766559f));
163         
164                 T r = sqrt(T(1) - z * z);
165         
166                 T x = r * cos(a);
167                 T y = r * sin(a);
168         
169                 return detail::tvec3<T, defaultp>(x, y, z) * Radius;    
170         }
\end{verbatim}
\end{Code}




Here is the call graph for this function: