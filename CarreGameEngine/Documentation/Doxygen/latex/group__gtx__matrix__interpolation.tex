\hypertarget{group__gtx__matrix__interpolation}{
\section{GLM\_\-GTX\_\-matrix\_\-interpolation}
\label{group__gtx__matrix__interpolation}\index{GLM\_\-GTX\_\-matrix\_\-interpolation@{GLM\_\-GTX\_\-matrix\_\-interpolation}}
}


Collaboration diagram for GLM\_\-GTX\_\-matrix\_\-interpolation:Allows to directly interpolate two exiciting matrices.  
\subsection*{Functions}
\begin{CompactItemize}
\item 
{\footnotesize template$<$typename T, precision P$>$ }\\GLM\_\-FUNC\_\-DECL void \hyperlink{group__gtx__matrix__interpolation_g16474d44af6a32a07c50df2409526d95}{glm::axisAngle} (detail::tmat4x4$<$ T, P $>$ const \&mat, detail::tvec3$<$ T, P $>$ \&axis, T \&angle)
\item 
{\footnotesize template$<$typename T, precision P$>$ }\\GLM\_\-FUNC\_\-DECL detail::tmat4x4$<$ T, P $>$ \hyperlink{group__gtx__matrix__interpolation_gf7c0106d03d55a7b670a6261b107f22b}{glm::axisAngleMatrix} (detail::tvec3$<$ T, P $>$ const \&axis, T const angle)
\item 
{\footnotesize template$<$typename T, precision P$>$ }\\GLM\_\-FUNC\_\-DECL detail::tmat4x4$<$ T, P $>$ \hyperlink{group__gtx__matrix__interpolation_gf559937fe5cea133f5e7f3c228255960}{glm::extractMatrixRotation} (detail::tmat4x4$<$ T, P $>$ const \&mat)
\item 
{\footnotesize template$<$typename T, precision P$>$ }\\GLM\_\-FUNC\_\-DECL detail::tmat4x4$<$ T, P $>$ \hyperlink{group__gtx__matrix__interpolation_gc618bbef632f87be5c570aa5afe63f30}{glm::interpolate} (detail::tmat4x4$<$ T, P $>$ const \&m1, detail::tmat4x4$<$ T, P $>$ const \&m2, T const delta)
\end{CompactItemize}


\subsection{Detailed Description}
Allows to directly interpolate two exiciting matrices. 

$<$glm/gtx/matrix\_\-interpolation.hpp$>$ need to be included to use these functionalities. 

\subsection{Function Documentation}
\hypertarget{group__gtx__matrix__interpolation_g16474d44af6a32a07c50df2409526d95}{
\index{gtx\_\-matrix\_\-interpolation@{gtx\_\-matrix\_\-interpolation}!axisAngle@{axisAngle}}
\index{axisAngle@{axisAngle}!gtx_matrix_interpolation@{gtx\_\-matrix\_\-interpolation}}
\subsubsection[axisAngle]{\setlength{\rightskip}{0pt plus 5cm}template$<$typename T, precision P$>$ GLM\_\-FUNC\_\-QUALIFIER void glm::axisAngle (detail::tmat4x4$<$ T, P $>$ const \& {\em mat}, \/  detail::tvec3$<$ T, P $>$ \& {\em axis}, \/  T \& {\em angle})\hspace{0.3cm}{\tt  \mbox{[}inline\mbox{]}}}}
\label{group__gtx__matrix__interpolation_g16474d44af6a32a07c50df2409526d95}


Get the axis and angle of the rotation from a matrix. From GLM\_\-GTX\_\-matrix\_\-interpolation extension. 

Definition at line 14 of file matrix\_\-interpolation.inl.

References glm::abs(), glm::acos(), glm::epsilon(), and glm::sqrt().

Referenced by glm::interpolate().

\begin{Code}\begin{verbatim}19         {
20                 T epsilon = (T)0.01;
21                 T epsilon2 = (T)0.1;
22 
23                 if((abs(mat[1][0] - mat[0][1]) < epsilon) && (abs(mat[2][0] - mat[0][2]) < epsilon) && (abs(mat[2][1] - mat[1][2]) < epsilon))
24                 {
25                         if ((abs(mat[1][0] + mat[0][1]) < epsilon2) && (abs(mat[2][0] + mat[0][2]) < epsilon2) && (abs(mat[2][1] + mat[1][2]) < epsilon2) && (abs(mat[0][0] + mat[1][1] + mat[2][2] - (T)3.0) < epsilon2))
26                         {
27                                 angle = (T)0.0;
28                                 axis.x = (T)1.0;
29                                 axis.y = (T)0.0;
30                                 axis.z = (T)0.0;
31                                 return;
32                         }
33                         angle = static_cast<T>(3.1415926535897932384626433832795);
34                         T xx = (mat[0][0] + (T)1.0) / (T)2.0;
35                         T yy = (mat[1][1] + (T)1.0) / (T)2.0;
36                         T zz = (mat[2][2] + (T)1.0) / (T)2.0;
37                         T xy = (mat[1][0] + mat[0][1]) / (T)4.0;
38                         T xz = (mat[2][0] + mat[0][2]) / (T)4.0;
39                         T yz = (mat[2][1] + mat[1][2]) / (T)4.0;
40                         if((xx > yy) && (xx > zz))
41                         {
42                                 if (xx < epsilon) {
43                                         axis.x = (T)0.0;
44                                         axis.y = (T)0.7071;
45                                         axis.z = (T)0.7071;
46                                 } else {
47                                         axis.x = sqrt(xx);
48                                         axis.y = xy / axis.x;
49                                         axis.z = xz / axis.x;
50                                 }
51                         }
52                         else if (yy > zz)
53                         {
54                                 if (yy < epsilon) {
55                                         axis.x = (T)0.7071;
56                                         axis.y = (T)0.0;
57                                         axis.z = (T)0.7071;
58                                 } else {
59                                         axis.y = sqrt(yy);
60                                         axis.x = xy / axis.y;
61                                         axis.z = yz / axis.y;
62                                 }
63                         }
64                         else
65                         {
66                                 if (zz < epsilon) {
67                                         axis.x = (T)0.7071;
68                                         axis.y = (T)0.7071;
69                                         axis.z = (T)0.0;
70                                 } else {
71                                         axis.z = sqrt(zz);
72                                         axis.x = xz / axis.z;
73                                         axis.y = yz / axis.z;
74                                 }
75                         }
76                         return;
77                 }
78                 T s = sqrt((mat[2][1] - mat[1][2]) * (mat[2][1] - mat[1][2]) + (mat[2][0] - mat[0][2]) * (mat[2][0] - mat[0][2]) + (mat[1][0] - mat[0][1]) * (mat[1][0] - mat[0][1]));
79                 if (glm::abs(s) < T(0.001))
80                         s = (T)1.0;
81                 angle = acos((mat[0][0] + mat[1][1] + mat[2][2] - (T)1.0) / (T)2.0);
82                 axis.x = (mat[1][2] - mat[2][1]) / s;
83                 axis.y = (mat[2][0] - mat[0][2]) / s;
84                 axis.z = (mat[0][1] - mat[1][0]) / s;
85         }
\end{verbatim}
\end{Code}




Here is the call graph for this function:

Here is the caller graph for this function:\hypertarget{group__gtx__matrix__interpolation_gf7c0106d03d55a7b670a6261b107f22b}{
\index{gtx\_\-matrix\_\-interpolation@{gtx\_\-matrix\_\-interpolation}!axisAngleMatrix@{axisAngleMatrix}}
\index{axisAngleMatrix@{axisAngleMatrix}!gtx_matrix_interpolation@{gtx\_\-matrix\_\-interpolation}}
\subsubsection[axisAngleMatrix]{\setlength{\rightskip}{0pt plus 5cm}template$<$typename T, precision P$>$ GLM\_\-FUNC\_\-QUALIFIER detail::tmat4x4$<$ T, P $>$ glm::axisAngleMatrix (detail::tvec3$<$ T, P $>$ const \& {\em axis}, \/  T const  {\em angle})\hspace{0.3cm}{\tt  \mbox{[}inline\mbox{]}}}}
\label{group__gtx__matrix__interpolation_gf7c0106d03d55a7b670a6261b107f22b}


Build a matrix from axis and angle. From GLM\_\-GTX\_\-matrix\_\-interpolation extension. 

Definition at line 89 of file matrix\_\-interpolation.inl.

References glm::cos(), glm::normalize(), and glm::sin().

Referenced by glm::interpolate().

\begin{Code}\begin{verbatim}93         {
94                 T c = cos(angle);
95                 T s = sin(angle);
96                 T t = static_cast<T>(1) - c;
97                 detail::tvec3<T, P> n = normalize(axis);
98 
99                 return detail::tmat4x4<T, P>(
100                         t * n.x * n.x + c,          t * n.x * n.y + n.z * s,    t * n.x * n.z - n.y * s,    T(0),
101                         t * n.x * n.y - n.z * s,    t * n.y * n.y + c,          t * n.y * n.z + n.x * s,    T(0),
102                         t * n.x * n.z + n.y * s,    t * n.y * n.z - n.x * s,    t * n.z * n.z + c,          T(0),
103                         T(0),                        T(0),                        T(0),                     T(1)
104                 );
105         }
\end{verbatim}
\end{Code}




Here is the call graph for this function:

Here is the caller graph for this function:\hypertarget{group__gtx__matrix__interpolation_gf559937fe5cea133f5e7f3c228255960}{
\index{gtx\_\-matrix\_\-interpolation@{gtx\_\-matrix\_\-interpolation}!extractMatrixRotation@{extractMatrixRotation}}
\index{extractMatrixRotation@{extractMatrixRotation}!gtx_matrix_interpolation@{gtx\_\-matrix\_\-interpolation}}
\subsubsection[extractMatrixRotation]{\setlength{\rightskip}{0pt plus 5cm}template$<$typename T, precision P$>$ GLM\_\-FUNC\_\-QUALIFIER detail::tmat4x4$<$ T, P $>$ glm::extractMatrixRotation (detail::tmat4x4$<$ T, P $>$ const \& {\em mat})\hspace{0.3cm}{\tt  \mbox{[}inline\mbox{]}}}}
\label{group__gtx__matrix__interpolation_gf559937fe5cea133f5e7f3c228255960}


Extracts the rotation part of a matrix. From GLM\_\-GTX\_\-matrix\_\-interpolation extension. 

Definition at line 109 of file matrix\_\-interpolation.inl.

Referenced by glm::interpolate().

\begin{Code}\begin{verbatim}112         {
113                 return detail::tmat4x4<T, P>(
114                         mat[0][0], mat[0][1], mat[0][2], 0.0,
115                         mat[1][0], mat[1][1], mat[1][2], 0.0,
116                         mat[2][0], mat[2][1], mat[2][2], 0.0,
117                         0.0,       0.0,       0.0,       1.0
118                 );
119         }
\end{verbatim}
\end{Code}




Here is the caller graph for this function:\hypertarget{group__gtx__matrix__interpolation_gc618bbef632f87be5c570aa5afe63f30}{
\index{gtx\_\-matrix\_\-interpolation@{gtx\_\-matrix\_\-interpolation}!interpolate@{interpolate}}
\index{interpolate@{interpolate}!gtx_matrix_interpolation@{gtx\_\-matrix\_\-interpolation}}
\subsubsection[interpolate]{\setlength{\rightskip}{0pt plus 5cm}template$<$typename T, precision P$>$ GLM\_\-FUNC\_\-QUALIFIER detail::tmat4x4$<$ T, P $>$ glm::interpolate (detail::tmat4x4$<$ T, P $>$ const \& {\em m1}, \/  detail::tmat4x4$<$ T, P $>$ const \& {\em m2}, \/  T const  {\em delta})\hspace{0.3cm}{\tt  \mbox{[}inline\mbox{]}}}}
\label{group__gtx__matrix__interpolation_gc618bbef632f87be5c570aa5afe63f30}


Build a interpolation of 4 $\ast$ 4 matrixes. From GLM\_\-GTX\_\-matrix\_\-interpolation extension. Warning! works only with rotation and/or translation matrixes, scale will generate unexpected results. 

Definition at line 123 of file matrix\_\-interpolation.inl.

References glm::axisAngle(), glm::axisAngleMatrix(), and glm::extractMatrixRotation().

\begin{Code}\begin{verbatim}128         {
129                 detail::tmat4x4<T, P> m1rot = extractMatrixRotation(m1);
130                 detail::tmat4x4<T, P> dltRotation = m2 * transpose(m1rot);
131                 detail::tvec3<T, P> dltAxis;
132                 T dltAngle;
133                 axisAngle(dltRotation, dltAxis, dltAngle);
134                 detail::tmat4x4<T, P> out = axisAngleMatrix(dltAxis, dltAngle * delta) * m1rot;
135                 out[3][0] = m1[3][0] + delta * (m2[3][0] - m1[3][0]);
136                 out[3][1] = m1[3][1] + delta * (m2[3][1] - m1[3][1]);
137                 out[3][2] = m1[3][2] + delta * (m2[3][2] - m1[3][2]);
138                 return out;
139         }
\end{verbatim}
\end{Code}




Here is the call graph for this function: