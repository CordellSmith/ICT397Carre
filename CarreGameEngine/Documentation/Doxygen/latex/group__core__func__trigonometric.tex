\hypertarget{group__core__func__trigonometric}{
\section{Angle and Trigonometry Functions}
\label{group__core__func__trigonometric}\index{Angle and Trigonometry Functions@{Angle and Trigonometry Functions}}
}


Collaboration diagram for Angle and Trigonometry Functions:\subsection*{Functions}
\begin{CompactItemize}
\item 
{\footnotesize template$<$typename genType$>$ }\\GLM\_\-FUNC\_\-DECL genType \hyperlink{group__core__func__trigonometric_g87953103f3ac701b8440a7d904fa2e4d}{glm::radians} (genType const \&degrees)
\item 
{\footnotesize template$<$typename genType$>$ }\\GLM\_\-FUNC\_\-DECL genType \hyperlink{group__core__func__trigonometric_gcb63bdf23d5e084a5b6a2ed0ae395e64}{glm::degrees} (genType const \&radians)
\item 
{\footnotesize template$<$typename genType$>$ }\\GLM\_\-FUNC\_\-DECL genType \hyperlink{group__core__func__trigonometric_gd4d4eda735d915be9af695fe2b4cded2}{glm::sin} (genType const \&angle)
\item 
{\footnotesize template$<$typename genType$>$ }\\GLM\_\-FUNC\_\-DECL genType \hyperlink{group__core__func__trigonometric_gfef15df90786cd24fe786cc0ff2cbc98}{glm::cos} (genType const \&angle)
\item 
{\footnotesize template$<$typename genType$>$ }\\GLM\_\-FUNC\_\-DECL genType \hyperlink{group__core__func__trigonometric_g93a81f04757351ba92d924e237cbeb61}{glm::tan} (genType const \&angle)
\item 
{\footnotesize template$<$typename genType$>$ }\\GLM\_\-FUNC\_\-DECL genType \hyperlink{group__core__func__trigonometric_gb87756fced3e8d3f58b24b65c8166b77}{glm::asin} (genType const \&x)
\item 
{\footnotesize template$<$typename genType$>$ }\\GLM\_\-FUNC\_\-DECL genType \hyperlink{group__core__func__trigonometric_gd945cb7263cb202d93ea76aa5d419078}{glm::acos} (genType const \&x)
\item 
{\footnotesize template$<$typename genType$>$ }\\GLM\_\-FUNC\_\-DECL genType \hyperlink{group__core__func__trigonometric_gb89f4e2a1ea1426dc87ab3a06901b68a}{glm::atan} (genType const \&y, genType const \&x)
\item 
{\footnotesize template$<$typename genType$>$ }\\GLM\_\-FUNC\_\-DECL genType \hyperlink{group__core__func__trigonometric_g459eaa7149e799125acda24938114746}{glm::atan} (genType const \&y\_\-over\_\-x)
\item 
{\footnotesize template$<$typename genType$>$ }\\GLM\_\-FUNC\_\-DECL genType \hyperlink{group__core__func__trigonometric_g925002c6a847894241278997d189429a}{glm::sinh} (genType const \&angle)
\item 
{\footnotesize template$<$typename genType$>$ }\\GLM\_\-FUNC\_\-DECL genType \hyperlink{group__core__func__trigonometric_g522e0c2f8dbac0df60a2bf436fb88e69}{glm::cosh} (genType const \&angle)
\item 
{\footnotesize template$<$typename genType$>$ }\\GLM\_\-FUNC\_\-DECL genType \hyperlink{group__core__func__trigonometric_gaa29bdd7f3b57755a7b6a9834d59887f}{glm::tanh} (genType const \&angle)
\item 
{\footnotesize template$<$typename genType$>$ }\\GLM\_\-FUNC\_\-DECL genType \hyperlink{group__core__func__trigonometric_gaedecffe9a7c10e5930c4ec938a0ca2e}{glm::asinh} (genType const \&x)
\item 
{\footnotesize template$<$typename genType$>$ }\\GLM\_\-FUNC\_\-DECL genType \hyperlink{group__core__func__trigonometric_g7d91deddd26925a390f08448a1b9ab1a}{glm::acosh} (genType const \&x)
\item 
{\footnotesize template$<$typename genType$>$ }\\GLM\_\-FUNC\_\-DECL genType \hyperlink{group__core__func__trigonometric_g5207916954ad98477bf488a9a188f045}{glm::atanh} (genType const \&x)
\end{CompactItemize}


\subsection{Detailed Description}
Function parameters specified as angle are assumed to be in units of radians. In no case will any of these functions result in a divide by zero error. If the divisor of a ratio is 0, then results will be undefined.

These all operate component-wise. The description is per component. 

\subsection{Function Documentation}
\hypertarget{group__core__func__trigonometric_gd945cb7263cb202d93ea76aa5d419078}{
\index{core\_\-func\_\-trigonometric@{core\_\-func\_\-trigonometric}!acos@{acos}}
\index{acos@{acos}!core_func_trigonometric@{core\_\-func\_\-trigonometric}}
\subsubsection[acos]{\setlength{\rightskip}{0pt plus 5cm}template$<$typename genType$>$ GLM\_\-FUNC\_\-QUALIFIER genType glm::acos (genType const \& {\em x})\hspace{0.3cm}{\tt  \mbox{[}inline\mbox{]}}}}
\label{group__core__func__trigonometric_gd945cb7263cb202d93ea76aa5d419078}


Arc cosine. Returns an angle whose sine is x. The range of values returned by this function is \mbox{[}0, PI\mbox{]}. Results are undefined if $|$x$|$ $>$ 1.

\begin{Desc}
\item[Template Parameters:]
\begin{description}
\item[{\em genType}]Floating-point scalar or vector types.\end{description}
\end{Desc}
\begin{Desc}
\item[See also:]\href{http://www.opengl.org/sdk/docs/manglsl/xhtml/acos.xml}{\tt GLSL acos man page} 

\href{http://www.opengl.org/registry/doc/GLSLangSpec.4.20.8.pdf}{\tt GLSL 4.20.8 specification, section 8.1 Angle and Trigonometry Functions} \end{Desc}


Definition at line 119 of file func\_\-trigonometric.inl.

Referenced by glm::angle(), glm::asec(), glm::axisAngle(), glm::mix(), glm::orientation(), glm::orientedAngle(), glm::pow(), and glm::slerp().

\begin{Code}\begin{verbatim}122         {
123                 GLM_STATIC_ASSERT(std::numeric_limits<genType>::is_iec559, "'acos' only accept floating-point input");
124 
125                 return genType(::std::acos(x));
126         }
\end{verbatim}
\end{Code}




Here is the caller graph for this function:\hypertarget{group__core__func__trigonometric_g7d91deddd26925a390f08448a1b9ab1a}{
\index{core\_\-func\_\-trigonometric@{core\_\-func\_\-trigonometric}!acosh@{acosh}}
\index{acosh@{acosh}!core_func_trigonometric@{core\_\-func\_\-trigonometric}}
\subsubsection[acosh]{\setlength{\rightskip}{0pt plus 5cm}template$<$typename genType$>$ GLM\_\-FUNC\_\-QUALIFIER genType glm::acosh (genType const \& {\em x})\hspace{0.3cm}{\tt  \mbox{[}inline\mbox{]}}}}
\label{group__core__func__trigonometric_g7d91deddd26925a390f08448a1b9ab1a}


Arc hyperbolic cosine; returns the non-negative inverse of cosh. Results are undefined if x $<$ 1.

\begin{Desc}
\item[Template Parameters:]
\begin{description}
\item[{\em genType}]Floating-point scalar or vector types.\end{description}
\end{Desc}
\begin{Desc}
\item[See also:]\href{http://www.opengl.org/sdk/docs/manglsl/xhtml/acosh.xml}{\tt GLSL acosh man page} 

\href{http://www.opengl.org/registry/doc/GLSLangSpec.4.20.8.pdf}{\tt GLSL 4.20.8 specification, section 8.1 Angle and Trigonometry Functions} \end{Desc}


Definition at line 217 of file func\_\-trigonometric.inl.

References glm::log(), and glm::sqrt().

Referenced by glm::asech().

\begin{Code}\begin{verbatim}220         {
221                 GLM_STATIC_ASSERT(std::numeric_limits<genType>::is_iec559, "'acosh' only accept floating-point input");
222 
223                 if(x < genType(1))
224                         return genType(0);
225                 return log(x + sqrt(x * x - genType(1)));
226         }
\end{verbatim}
\end{Code}




Here is the call graph for this function:

Here is the caller graph for this function:\hypertarget{group__core__func__trigonometric_gb87756fced3e8d3f58b24b65c8166b77}{
\index{core\_\-func\_\-trigonometric@{core\_\-func\_\-trigonometric}!asin@{asin}}
\index{asin@{asin}!core_func_trigonometric@{core\_\-func\_\-trigonometric}}
\subsubsection[asin]{\setlength{\rightskip}{0pt plus 5cm}template$<$typename genType$>$ GLM\_\-FUNC\_\-QUALIFIER genType glm::asin (genType const \& {\em x})\hspace{0.3cm}{\tt  \mbox{[}inline\mbox{]}}}}
\label{group__core__func__trigonometric_gb87756fced3e8d3f58b24b65c8166b77}


Arc sine. Returns an angle whose sine is x. The range of values returned by this function is \mbox{[}-PI/2, PI/2\mbox{]}. Results are undefined if $|$x$|$ $>$ 1.

\begin{Desc}
\item[Template Parameters:]
\begin{description}
\item[{\em genType}]Floating-point scalar or vector types.\end{description}
\end{Desc}
\begin{Desc}
\item[See also:]\href{http://www.opengl.org/sdk/docs/manglsl/xhtml/asin.xml}{\tt GLSL asin man page} 

\href{http://www.opengl.org/registry/doc/GLSLangSpec.4.20.8.pdf}{\tt GLSL 4.20.8 specification, section 8.1 Angle and Trigonometry Functions} \end{Desc}


Definition at line 105 of file func\_\-trigonometric.inl.

Referenced by glm::acsc(), and glm::yaw().

\begin{Code}\begin{verbatim}108         {
109                 GLM_STATIC_ASSERT(std::numeric_limits<genType>::is_iec559, "'asin' only accept floating-point input");
110 
111                 return genType(::std::asin(x));
112         }
\end{verbatim}
\end{Code}




Here is the caller graph for this function:\hypertarget{group__core__func__trigonometric_gaedecffe9a7c10e5930c4ec938a0ca2e}{
\index{core\_\-func\_\-trigonometric@{core\_\-func\_\-trigonometric}!asinh@{asinh}}
\index{asinh@{asinh}!core_func_trigonometric@{core\_\-func\_\-trigonometric}}
\subsubsection[asinh]{\setlength{\rightskip}{0pt plus 5cm}template$<$typename genType$>$ GLM\_\-FUNC\_\-QUALIFIER genType glm::asinh (genType const \& {\em x})\hspace{0.3cm}{\tt  \mbox{[}inline\mbox{]}}}}
\label{group__core__func__trigonometric_gaedecffe9a7c10e5930c4ec938a0ca2e}


Arc hyperbolic sine; returns the inverse of sinh.

\begin{Desc}
\item[Template Parameters:]
\begin{description}
\item[{\em genType}]Floating-point scalar or vector types.\end{description}
\end{Desc}
\begin{Desc}
\item[See also:]\href{http://www.opengl.org/sdk/docs/manglsl/xhtml/asinh.xml}{\tt GLSL asinh man page} 

\href{http://www.opengl.org/registry/doc/GLSLangSpec.4.20.8.pdf}{\tt GLSL 4.20.8 specification, section 8.1 Angle and Trigonometry Functions} \end{Desc}


Definition at line 203 of file func\_\-trigonometric.inl.

References glm::abs(), glm::log(), and glm::sqrt().

Referenced by glm::acsch().

\begin{Code}\begin{verbatim}206         {
207                 GLM_STATIC_ASSERT(std::numeric_limits<genType>::is_iec559, "'asinh' only accept floating-point input");
208                 
209                 return (x < genType(0) ? genType(-1) : (x > genType(0) ? genType(1) : genType(0))) * log(abs(x) + sqrt(genType(1) + x * x));
210         }
\end{verbatim}
\end{Code}




Here is the call graph for this function:

Here is the caller graph for this function:\hypertarget{group__core__func__trigonometric_g459eaa7149e799125acda24938114746}{
\index{core\_\-func\_\-trigonometric@{core\_\-func\_\-trigonometric}!atan@{atan}}
\index{atan@{atan}!core_func_trigonometric@{core\_\-func\_\-trigonometric}}
\subsubsection[atan]{\setlength{\rightskip}{0pt plus 5cm}template$<$typename genType$>$ GLM\_\-FUNC\_\-QUALIFIER genType glm::atan (genType const \& {\em y\_\-over\_\-x})\hspace{0.3cm}{\tt  \mbox{[}inline\mbox{]}}}}
\label{group__core__func__trigonometric_g459eaa7149e799125acda24938114746}


Arc tangent. Returns an angle whose tangent is y\_\-over\_\-x. The range of values returned by this function is \mbox{[}-PI/2, PI/2\mbox{]}.

\begin{Desc}
\item[Template Parameters:]
\begin{description}
\item[{\em genType}]Floating-point scalar or vector types.\end{description}
\end{Desc}
\begin{Desc}
\item[See also:]\href{http://www.opengl.org/sdk/docs/manglsl/xhtml/atan.xml}{\tt GLSL atan man page} 

\href{http://www.opengl.org/registry/doc/GLSLangSpec.4.20.8.pdf}{\tt GLSL 4.20.8 specification, section 8.1 Angle and Trigonometry Functions} \end{Desc}


Definition at line 147 of file func\_\-trigonometric.inl.

Referenced by glm::pitch(), and glm::roll().

\begin{Code}\begin{verbatim}150         {
151                 GLM_STATIC_ASSERT(std::numeric_limits<genType>::is_iec559, "'atan' only accept floating-point input");
152 
153                 return genType(::std::atan(x));
154         }
\end{verbatim}
\end{Code}




Here is the caller graph for this function:\hypertarget{group__core__func__trigonometric_gb89f4e2a1ea1426dc87ab3a06901b68a}{
\index{core\_\-func\_\-trigonometric@{core\_\-func\_\-trigonometric}!atan@{atan}}
\index{atan@{atan}!core_func_trigonometric@{core\_\-func\_\-trigonometric}}
\subsubsection[atan]{\setlength{\rightskip}{0pt plus 5cm}template$<$typename genType$>$ GLM\_\-FUNC\_\-QUALIFIER genType glm::atan (genType const \& {\em y}, \/  genType const \& {\em x})\hspace{0.3cm}{\tt  \mbox{[}inline\mbox{]}}}}
\label{group__core__func__trigonometric_gb89f4e2a1ea1426dc87ab3a06901b68a}


Arc tangent. Returns an angle whose tangent is y/x. The signs of x and y are used to determine what quadrant the angle is in. The range of values returned by this function is \mbox{[}-PI, PI\mbox{]}. Results are undefined if x and y are both 0.

\begin{Desc}
\item[Template Parameters:]
\begin{description}
\item[{\em genType}]Floating-point scalar or vector types.\end{description}
\end{Desc}
\begin{Desc}
\item[See also:]\href{http://www.opengl.org/sdk/docs/manglsl/xhtml/atan.xml}{\tt GLSL atan man page} 

\href{http://www.opengl.org/registry/doc/GLSLangSpec.4.20.8.pdf}{\tt GLSL 4.20.8 specification, section 8.1 Angle and Trigonometry Functions} \end{Desc}


Definition at line 133 of file func\_\-trigonometric.inl.

References glm::atan2().

Referenced by glm::acot(), glm::atan2(), glm::log(), glm::polar(), and glm::shortMix().

\begin{Code}\begin{verbatim}137         {
138                 GLM_STATIC_ASSERT(std::numeric_limits<genType>::is_iec559, "'atan' only accept floating-point input");
139 
140                 return genType(::std::atan2(y, x));
141         }
\end{verbatim}
\end{Code}




Here is the call graph for this function:

Here is the caller graph for this function:\hypertarget{group__core__func__trigonometric_g5207916954ad98477bf488a9a188f045}{
\index{core\_\-func\_\-trigonometric@{core\_\-func\_\-trigonometric}!atanh@{atanh}}
\index{atanh@{atanh}!core_func_trigonometric@{core\_\-func\_\-trigonometric}}
\subsubsection[atanh]{\setlength{\rightskip}{0pt plus 5cm}template$<$typename genType$>$ GLM\_\-FUNC\_\-QUALIFIER genType glm::atanh (genType const \& {\em x})\hspace{0.3cm}{\tt  \mbox{[}inline\mbox{]}}}}
\label{group__core__func__trigonometric_g5207916954ad98477bf488a9a188f045}


Arc hyperbolic tangent; returns the inverse of tanh. Results are undefined if abs(x) $>$= 1.

\begin{Desc}
\item[Template Parameters:]
\begin{description}
\item[{\em genType}]Floating-point scalar or vector types.\end{description}
\end{Desc}
\begin{Desc}
\item[See also:]\href{http://www.opengl.org/sdk/docs/manglsl/xhtml/atanh.xml}{\tt GLSL atanh man page} 

\href{http://www.opengl.org/registry/doc/GLSLangSpec.4.20.8.pdf}{\tt GLSL 4.20.8 specification, section 8.1 Angle and Trigonometry Functions} \end{Desc}


Definition at line 233 of file func\_\-trigonometric.inl.

References glm::abs(), and glm::log().

Referenced by glm::acoth().

\begin{Code}\begin{verbatim}236         {
237                 GLM_STATIC_ASSERT(std::numeric_limits<genType>::is_iec559, "'atanh' only accept floating-point input");
238                 
239                 if(abs(x) >= genType(1))
240                         return 0;
241                 return genType(0.5) * log((genType(1) + x) / (genType(1) - x));
242         }
\end{verbatim}
\end{Code}




Here is the call graph for this function:

Here is the caller graph for this function:\hypertarget{group__core__func__trigonometric_gfef15df90786cd24fe786cc0ff2cbc98}{
\index{core\_\-func\_\-trigonometric@{core\_\-func\_\-trigonometric}!cos@{cos}}
\index{cos@{cos}!core_func_trigonometric@{core\_\-func\_\-trigonometric}}
\subsubsection[cos]{\setlength{\rightskip}{0pt plus 5cm}template$<$typename genType$>$ GLM\_\-FUNC\_\-QUALIFIER genType glm::cos (genType const \& {\em angle})\hspace{0.3cm}{\tt  \mbox{[}inline\mbox{]}}}}
\label{group__core__func__trigonometric_gfef15df90786cd24fe786cc0ff2cbc98}


The standard trigonometric cosine function. The values returned by this function will range from \mbox{[}-1, 1\mbox{]}.

\begin{Desc}
\item[Template Parameters:]
\begin{description}
\item[{\em genType}]Floating-point scalar or vector types.\end{description}
\end{Desc}
\begin{Desc}
\item[See also:]\href{http://www.opengl.org/sdk/docs/manglsl/xhtml/cos.xml}{\tt GLSL cos man page} 

\href{http://www.opengl.org/registry/doc/GLSLangSpec.4.20.8.pdf}{\tt GLSL 4.20.8 specification, section 8.1 Angle and Trigonometry Functions} \end{Desc}


Definition at line 79 of file func\_\-trigonometric.inl.

Referenced by glm::angleAxis(), glm::axisAngleMatrix(), glm::circularRand(), glm::euclidean(), glm::eulerAngleX(), glm::eulerAngleXY(), glm::eulerAngleY(), glm::eulerAngleYX(), glm::eulerAngleYXZ(), glm::eulerAngleZ(), glm::exp(), Player::MoveBackward(), Player::MoveForward(), glm::orientate2(), glm::orientate3(), glm::perspectiveFov(), glm::pow(), glm::rotate(), glm::rotateNormalizedAxis(), glm::rotateX(), glm::rotateY(), glm::rotateZ(), glm::sec(), glm::sphericalRand(), and glm::yawPitchRoll().

\begin{Code}\begin{verbatim}80         {
81                 GLM_STATIC_ASSERT(std::numeric_limits<genType>::is_iec559, "'cos' only accept floating-point input");
82 
83                 return genType(::std::cos(angle));
84         }
\end{verbatim}
\end{Code}




Here is the caller graph for this function:\hypertarget{group__core__func__trigonometric_g522e0c2f8dbac0df60a2bf436fb88e69}{
\index{core\_\-func\_\-trigonometric@{core\_\-func\_\-trigonometric}!cosh@{cosh}}
\index{cosh@{cosh}!core_func_trigonometric@{core\_\-func\_\-trigonometric}}
\subsubsection[cosh]{\setlength{\rightskip}{0pt plus 5cm}template$<$typename genType$>$ GLM\_\-FUNC\_\-QUALIFIER genType glm::cosh (genType const \& {\em angle})\hspace{0.3cm}{\tt  \mbox{[}inline\mbox{]}}}}
\label{group__core__func__trigonometric_g522e0c2f8dbac0df60a2bf436fb88e69}


Returns the hyperbolic cosine function, (exp(x) + exp(-x)) / 2

\begin{Desc}
\item[Template Parameters:]
\begin{description}
\item[{\em genType}]Floating-point scalar or vector types.\end{description}
\end{Desc}
\begin{Desc}
\item[See also:]\href{http://www.opengl.org/sdk/docs/manglsl/xhtml/cosh.xml}{\tt GLSL cosh man page} 

\href{http://www.opengl.org/registry/doc/GLSLangSpec.4.20.8.pdf}{\tt GLSL 4.20.8 specification, section 8.1 Angle and Trigonometry Functions} \end{Desc}


Definition at line 175 of file func\_\-trigonometric.inl.

Referenced by glm::coth(), and glm::sech().

\begin{Code}\begin{verbatim}178         {
179                 GLM_STATIC_ASSERT(std::numeric_limits<genType>::is_iec559, "'cosh' only accept floating-point input");
180 
181                 return genType(std::cosh(angle));
182         }
\end{verbatim}
\end{Code}




Here is the caller graph for this function:\hypertarget{group__core__func__trigonometric_gcb63bdf23d5e084a5b6a2ed0ae395e64}{
\index{core\_\-func\_\-trigonometric@{core\_\-func\_\-trigonometric}!degrees@{degrees}}
\index{degrees@{degrees}!core_func_trigonometric@{core\_\-func\_\-trigonometric}}
\subsubsection[degrees]{\setlength{\rightskip}{0pt plus 5cm}template$<$typename genType$>$ GLM\_\-FUNC\_\-QUALIFIER genType glm::degrees (genType const \& {\em radians})\hspace{0.3cm}{\tt  \mbox{[}inline\mbox{]}}}}
\label{group__core__func__trigonometric_gcb63bdf23d5e084a5b6a2ed0ae395e64}


Converts radians to degrees and returns the result.

\begin{Desc}
\item[Template Parameters:]
\begin{description}
\item[{\em genType}]Floating-point scalar or vector types.\end{description}
\end{Desc}
\begin{Desc}
\item[See also:]\href{http://www.opengl.org/sdk/docs/manglsl/xhtml/degrees.xml}{\tt GLSL degrees man page} 

\href{http://www.opengl.org/registry/doc/GLSLangSpec.4.20.8.pdf}{\tt GLSL 4.20.8 specification, section 8.1 Angle and Trigonometry Functions} \end{Desc}


Definition at line 52 of file func\_\-trigonometric.inl.

Referenced by glm::angle(), Camera::GetRotationMatrix(), glm::orientation(), glm::orientedAngle(), glm::pitch(), glm::polar(), glm::roll(), and glm::yaw().

\begin{Code}\begin{verbatim}55         {
56                 GLM_STATIC_ASSERT(std::numeric_limits<genType>::is_iec559, "'degrees' only accept floating-point input");
57 
58                 return radians * genType(57.295779513082320876798154814105);
59         }
\end{verbatim}
\end{Code}




Here is the caller graph for this function:\hypertarget{group__core__func__trigonometric_g87953103f3ac701b8440a7d904fa2e4d}{
\index{core\_\-func\_\-trigonometric@{core\_\-func\_\-trigonometric}!radians@{radians}}
\index{radians@{radians}!core_func_trigonometric@{core\_\-func\_\-trigonometric}}
\subsubsection[radians]{\setlength{\rightskip}{0pt plus 5cm}template$<$typename genType$>$ GLM\_\-FUNC\_\-QUALIFIER genType glm::radians (genType const \& {\em degrees})\hspace{0.3cm}{\tt  \mbox{[}inline\mbox{]}}}}
\label{group__core__func__trigonometric_g87953103f3ac701b8440a7d904fa2e4d}


Converts degrees to radians and returns the result.

\begin{Desc}
\item[Template Parameters:]
\begin{description}
\item[{\em genType}]Floating-point scalar or vector types.\end{description}
\end{Desc}
\begin{Desc}
\item[See also:]\href{http://www.opengl.org/sdk/docs/manglsl/xhtml/radians.xml}{\tt GLSL radians man page} 

\href{http://www.opengl.org/registry/doc/GLSLangSpec.4.20.8.pdf}{\tt GLSL 4.20.8 specification, section 8.1 Angle and Trigonometry Functions} \end{Desc}


Definition at line 38 of file func\_\-trigonometric.inl.

Referenced by glm::angleAxis(), glm::euclidean(), glm::infinitePerspective(), glm::perspective(), glm::perspectiveFov(), glm::rotate(), glm::rotateNormalizedAxis(), glm::rotateX(), glm::rotateY(), glm::rotateZ(), and glm::tweakedInfinitePerspective().

\begin{Code}\begin{verbatim}41         {
42                 GLM_STATIC_ASSERT(std::numeric_limits<genType>::is_iec559, "'radians' only accept floating-point input");
43 
44                 return degrees * genType(0.01745329251994329576923690768489);
45         }
\end{verbatim}
\end{Code}




Here is the caller graph for this function:\hypertarget{group__core__func__trigonometric_gd4d4eda735d915be9af695fe2b4cded2}{
\index{core\_\-func\_\-trigonometric@{core\_\-func\_\-trigonometric}!sin@{sin}}
\index{sin@{sin}!core_func_trigonometric@{core\_\-func\_\-trigonometric}}
\subsubsection[sin]{\setlength{\rightskip}{0pt plus 5cm}template$<$typename genType$>$ GLM\_\-FUNC\_\-QUALIFIER genType glm::sin (genType const \& {\em angle})\hspace{0.3cm}{\tt  \mbox{[}inline\mbox{]}}}}
\label{group__core__func__trigonometric_gd4d4eda735d915be9af695fe2b4cded2}


The standard trigonometric sine function. The values returned by this function will range from \mbox{[}-1, 1\mbox{]}.

\begin{Desc}
\item[Template Parameters:]
\begin{description}
\item[{\em genType}]Floating-point scalar or vector types.\end{description}
\end{Desc}
\begin{Desc}
\item[See also:]\href{http://www.opengl.org/sdk/docs/manglsl/xhtml/sin.xml}{\tt GLSL sin man page} 

\href{http://www.opengl.org/registry/doc/GLSLangSpec.4.20.8.pdf}{\tt GLSL 4.20.8 specification, section 8.1 Angle and Trigonometry Functions} \end{Desc}


Definition at line 66 of file func\_\-trigonometric.inl.

Referenced by glm::angleAxis(), glm::axisAngleMatrix(), glm::circularRand(), glm::csc(), glm::euclidean(), glm::eulerAngleX(), glm::eulerAngleXY(), glm::eulerAngleY(), glm::eulerAngleYX(), glm::eulerAngleYXZ(), glm::eulerAngleZ(), glm::exp(), glm::mix(), Player::MoveBackward(), Player::MoveForward(), glm::orientate2(), glm::orientate3(), glm::perspectiveFov(), glm::pow(), glm::rotate(), glm::rotateNormalizedAxis(), glm::rotateX(), glm::rotateY(), glm::rotateZ(), glm::shortMix(), glm::slerp(), glm::sphericalRand(), and glm::yawPitchRoll().

\begin{Code}\begin{verbatim}69         {
70                 GLM_STATIC_ASSERT(std::numeric_limits<genType>::is_iec559, "'sin' only accept floating-point input");
71 
72                 return genType(::std::sin(angle));
73         }
\end{verbatim}
\end{Code}




Here is the caller graph for this function:\hypertarget{group__core__func__trigonometric_g925002c6a847894241278997d189429a}{
\index{core\_\-func\_\-trigonometric@{core\_\-func\_\-trigonometric}!sinh@{sinh}}
\index{sinh@{sinh}!core_func_trigonometric@{core\_\-func\_\-trigonometric}}
\subsubsection[sinh]{\setlength{\rightskip}{0pt plus 5cm}template$<$typename genType$>$ GLM\_\-FUNC\_\-QUALIFIER genType glm::sinh (genType const \& {\em angle})\hspace{0.3cm}{\tt  \mbox{[}inline\mbox{]}}}}
\label{group__core__func__trigonometric_g925002c6a847894241278997d189429a}


Returns the hyperbolic sine function, (exp(x) - exp(-x)) / 2

\begin{Desc}
\item[Template Parameters:]
\begin{description}
\item[{\em genType}]Floating-point scalar or vector types.\end{description}
\end{Desc}
\begin{Desc}
\item[See also:]\href{http://www.opengl.org/sdk/docs/manglsl/xhtml/sinh.xml}{\tt GLSL sinh man page} 

\href{http://www.opengl.org/registry/doc/GLSLangSpec.4.20.8.pdf}{\tt GLSL 4.20.8 specification, section 8.1 Angle and Trigonometry Functions} \end{Desc}


Definition at line 161 of file func\_\-trigonometric.inl.

Referenced by glm::coth(), and glm::csch().

\begin{Code}\begin{verbatim}164         {
165                 GLM_STATIC_ASSERT(std::numeric_limits<genType>::is_iec559, "'sinh' only accept floating-point input");
166 
167                 return genType(std::sinh(angle));
168         }
\end{verbatim}
\end{Code}




Here is the caller graph for this function:\hypertarget{group__core__func__trigonometric_g93a81f04757351ba92d924e237cbeb61}{
\index{core\_\-func\_\-trigonometric@{core\_\-func\_\-trigonometric}!tan@{tan}}
\index{tan@{tan}!core_func_trigonometric@{core\_\-func\_\-trigonometric}}
\subsubsection[tan]{\setlength{\rightskip}{0pt plus 5cm}template$<$typename genType$>$ GLM\_\-FUNC\_\-QUALIFIER genType glm::tan (genType const \& {\em angle})\hspace{0.3cm}{\tt  \mbox{[}inline\mbox{]}}}}
\label{group__core__func__trigonometric_g93a81f04757351ba92d924e237cbeb61}


The standard trigonometric tangent function.

\begin{Desc}
\item[Template Parameters:]
\begin{description}
\item[{\em genType}]Floating-point scalar or vector types.\end{description}
\end{Desc}
\begin{Desc}
\item[See also:]\href{http://www.opengl.org/sdk/docs/manglsl/xhtml/tan.xml}{\tt GLSL tan man page} 

\href{http://www.opengl.org/registry/doc/GLSLangSpec.4.20.8.pdf}{\tt GLSL 4.20.8 specification, section 8.1 Angle and Trigonometry Functions} \end{Desc}


Definition at line 91 of file func\_\-trigonometric.inl.

Referenced by glm::cot(), glm::infinitePerspective(), glm::perspective(), and glm::tweakedInfinitePerspective().

\begin{Code}\begin{verbatim}94         {
95                 GLM_STATIC_ASSERT(std::numeric_limits<genType>::is_iec559, "'tan' only accept floating-point input");
96 
97                 return genType(::std::tan(angle));
98         }
\end{verbatim}
\end{Code}




Here is the caller graph for this function:\hypertarget{group__core__func__trigonometric_gaa29bdd7f3b57755a7b6a9834d59887f}{
\index{core\_\-func\_\-trigonometric@{core\_\-func\_\-trigonometric}!tanh@{tanh}}
\index{tanh@{tanh}!core_func_trigonometric@{core\_\-func\_\-trigonometric}}
\subsubsection[tanh]{\setlength{\rightskip}{0pt plus 5cm}template$<$typename genType$>$ GLM\_\-FUNC\_\-QUALIFIER genType glm::tanh (genType const \& {\em angle})\hspace{0.3cm}{\tt  \mbox{[}inline\mbox{]}}}}
\label{group__core__func__trigonometric_gaa29bdd7f3b57755a7b6a9834d59887f}


Returns the hyperbolic tangent function, sinh(angle) / cosh(angle)

\begin{Desc}
\item[Template Parameters:]
\begin{description}
\item[{\em genType}]Floating-point scalar or vector types.\end{description}
\end{Desc}
\begin{Desc}
\item[See also:]\href{http://www.opengl.org/sdk/docs/manglsl/xhtml/tanh.xml}{\tt GLSL tanh man page} 

\href{http://www.opengl.org/registry/doc/GLSLangSpec.4.20.8.pdf}{\tt GLSL 4.20.8 specification, section 8.1 Angle and Trigonometry Functions} \end{Desc}


Definition at line 189 of file func\_\-trigonometric.inl.

\begin{Code}\begin{verbatim}192         {
193                 GLM_STATIC_ASSERT(std::numeric_limits<genType>::is_iec559, "'tanh' only accept floating-point input");
194 
195                 return genType(std::tanh(angle));
196         }
\end{verbatim}
\end{Code}


