\hypertarget{gim__radixsort_8h}{
\section{C:/Users/New/Documents/Games\_\-Technology/Year4\_\-Semester1/ICT397/$\sim$My Work/Assignment2/ICT397Carre/CarreGameEngine/Dependencies/BulletPhysicsEngine/include/BulletCollision/Gimpact/gim\_\-radixsort.h File Reference}
\label{gim__radixsort_8h}\index{C:/Users/New/Documents/Games\_\-Technology/Year4\_\-Semester1/ICT397/$\sim$My Work/Assignment2/ICT397Carre/CarreGameEngine/Dependencies/BulletPhysicsEngine/include/BulletCollision/Gimpact/gim\_\-radixsort.h@{C:/Users/New/Documents/Games\_\-Technology/Year4\_\-Semester1/ICT397/$\sim$My Work/Assignment2/ICT397Carre/CarreGameEngine/Dependencies/BulletPhysicsEngine/include/BulletCollision/Gimpact/gim\_\-radixsort.h}}
}
{\tt \#include \char`\"{}gim\_\-memory.h\char`\"{}}\par


Include dependency graph for gim\_\-radixsort.h:

This graph shows which files directly or indirectly include this file:\subsection*{Classes}
\begin{CompactItemize}
\item 
class \hyperlink{classless__comparator}{less\_\-comparator}
\item 
class \hyperlink{classinteger__comparator}{integer\_\-comparator}
\begin{CompactList}\small\item\em Prototype for comparators. \item\end{CompactList}\item 
class \hyperlink{classuint__key__func}{uint\_\-key\_\-func}
\begin{CompactList}\small\item\em Prototype for getting the integer representation of an object. \item\end{CompactList}\item 
class \hyperlink{classcopy__elements__func}{copy\_\-elements\_\-func}
\begin{CompactList}\small\item\em Prototype for copying elements. \item\end{CompactList}\item 
class \hyperlink{classmemcopy__elements__func}{memcopy\_\-elements\_\-func}
\begin{CompactList}\small\item\em Prototype for copying elements. \item\end{CompactList}\item 
class \hyperlink{class_g_i_m___r_s_o_r_t___t_o_k_e_n___c_o_m_p_a_r_a_t_o_r}{GIM\_\-RSORT\_\-TOKEN\_\-COMPARATOR}
\begin{CompactList}\small\item\em Prototype for comparators. \item\end{CompactList}\end{CompactItemize}
\begin{CompactItemize}
\item 
\hypertarget{gim__radixsort_8h_7465b49642a34b13aaaf84bc5b38ab67}{
void \hyperlink{gim__radixsort_8h_7465b49642a34b13aaaf84bc5b38ab67}{gim\_\-radix\_\-sort\_\-rtokens} (GIM\_\-RSORT\_\-TOKEN $\ast$array, GIM\_\-RSORT\_\-TOKEN $\ast$sorted, GUINT element\_\-count)}
\label{gim__radixsort_8h_7465b49642a34b13aaaf84bc5b38ab67}

\begin{CompactList}\small\item\em Radix sort for unsigned integer keys. \item\end{CompactList}\item 
{\footnotesize template$<$typename T, class GETKEY\_\-CLASS$>$ }\\void \hyperlink{gim__radixsort_8h_dc804c6eb5898638d59eeed4b400f1d5}{gim\_\-radix\_\-sort\_\-array\_\-tokens} (T $\ast$array, GIM\_\-RSORT\_\-TOKEN $\ast$sorted\_\-tokens, GUINT element\_\-count, GETKEY\_\-CLASS uintkey\_\-macro)
\begin{CompactList}\small\item\em Get the sorted tokens from an array. For generic use. Tokens are IRR\_\-RSORT\_\-TOKEN. \item\end{CompactList}\item 
{\footnotesize template$<$typename T, class GETKEY\_\-CLASS, class COPY\_\-CLASS$>$ }\\void \hyperlink{gim__radixsort_8h_41009ea63c51a0a22ff221b583dc391d}{gim\_\-radix\_\-sort} (T $\ast$array, GUINT element\_\-count, GETKEY\_\-CLASS get\_\-uintkey\_\-macro, COPY\_\-CLASS copy\_\-elements\_\-macro)
\begin{CompactList}\small\item\em Sorts array in place. For generic use. \item\end{CompactList}\item 
{\footnotesize template$<$class T, typename KEYCLASS, typename COMP\_\-CLASS$>$ }\\bool \hyperlink{gim__radixsort_8h_7be883eaa7e11dab454089d654537e35}{gim\_\-binary\_\-search\_\-ex} (const T $\ast$\_\-array, GUINT \_\-start\_\-i, GUINT \_\-end\_\-i, GUINT \&\_\-result\_\-index, const KEYCLASS \&\_\-search\_\-key, COMP\_\-CLASS \_\-comp\_\-macro)
\begin{CompactList}\small\item\em Failsafe Iterative binary search,. \item\end{CompactList}\item 
{\footnotesize template$<$class T$>$ }\\bool \hyperlink{gim__radixsort_8h_f8659fc44c5ded1334ef6957654bc76b}{gim\_\-binary\_\-search} (const T $\ast$\_\-array, GUINT \_\-start\_\-i, GUINT \_\-end\_\-i, const T \&\_\-search\_\-key, GUINT \&\_\-result\_\-index)
\begin{CompactList}\small\item\em Failsafe Iterative binary search,Template version. \item\end{CompactList}\item 
\hypertarget{gim__radixsort_8h_e14d8ed5dd848db8f3623db9d6e2b3c5}{
{\footnotesize template$<$typename T, typename COMP\_\-CLASS$>$ }\\void \hyperlink{gim__radixsort_8h_e14d8ed5dd848db8f3623db9d6e2b3c5}{gim\_\-down\_\-heap} (T $\ast$pArr, GUINT k, GUINT n, COMP\_\-CLASS CompareFunc)}
\label{gim__radixsort_8h_e14d8ed5dd848db8f3623db9d6e2b3c5}

\begin{CompactList}\small\item\em heap sort from \href{http://www.csse.monash.edu.au/~lloyd/tildeAlgDS/Sort/Heap/}{\tt http://www.csse.monash.edu.au/$\sim$lloyd/tildeAlgDS/Sort/Heap/} \item\end{CompactList}\end{CompactItemize}


\subsection{Detailed Description}
\begin{Desc}
\item[Author:]Francisco Leon Najera. Based on the work of Michael Herf : \char`\"{}fast floating-point radix sort\char`\"{} Avaliable on \href{http://www.stereopsis.com/radix.html}{\tt http://www.stereopsis.com/radix.html} \end{Desc}


Definition in file \hyperlink{gim__radixsort_8h-source}{gim\_\-radixsort.h}.

\subsection{Function Documentation}
\hypertarget{gim__radixsort_8h_f8659fc44c5ded1334ef6957654bc76b}{
\index{gim\_\-radixsort.h@{gim\_\-radixsort.h}!gim\_\-binary\_\-search@{gim\_\-binary\_\-search}}
\index{gim\_\-binary\_\-search@{gim\_\-binary\_\-search}!gim_radixsort.h@{gim\_\-radixsort.h}}
\subsubsection[gim\_\-binary\_\-search]{\setlength{\rightskip}{0pt plus 5cm}template$<$class T$>$ bool gim\_\-binary\_\-search (const T $\ast$ {\em \_\-array}, \/  GUINT {\em \_\-start\_\-i}, \/  GUINT {\em \_\-end\_\-i}, \/  const T \& {\em \_\-search\_\-key}, \/  GUINT \& {\em \_\-result\_\-index})\hspace{0.3cm}{\tt  \mbox{[}inline\mbox{]}}}}
\label{gim__radixsort_8h_f8659fc44c5ded1334ef6957654bc76b}


Failsafe Iterative binary search,Template version. 

If the element is not found, it returns the nearest upper element position, may be the further position after the last element. \begin{Desc}
\item[Parameters:]
\begin{description}
\item[{\em \_\-array}]\item[{\em \_\-start\_\-i}]the beginning of the array \item[{\em \_\-end\_\-i}]the ending index of the array \item[{\em \_\-search\_\-key}]Value to find \item[{\em \_\-result\_\-index}]the index of the found element, or if not found then it will get the index of the closest bigger value \end{description}
\end{Desc}
\begin{Desc}
\item[Returns:]true if found, else false \end{Desc}


Definition at line 318 of file gim\_\-radixsort.h.

\begin{Code}\begin{verbatim}322 {
323         GUINT _i = _start_i;
324         GUINT _j = _end_i+1;
325         GUINT _k;
326         while(_i < _j)
327         {
328                 _k = (_j+_i-1)/2;
329                 if(_array[_k]==_search_key)
330                 {
331                         _result_index = _k;
332                         return true;
333                 }
334                 else if (_array[_k]<_search_key)
335                 {
336                         _i = _k+1;
337                 }
338                 else
339                 {
340                         _j = _k;
341                 }
342         }
343         _result_index = _i;
344         return false;
345 }
\end{verbatim}
\end{Code}


\hypertarget{gim__radixsort_8h_7be883eaa7e11dab454089d654537e35}{
\index{gim\_\-radixsort.h@{gim\_\-radixsort.h}!gim\_\-binary\_\-search\_\-ex@{gim\_\-binary\_\-search\_\-ex}}
\index{gim\_\-binary\_\-search\_\-ex@{gim\_\-binary\_\-search\_\-ex}!gim_radixsort.h@{gim\_\-radixsort.h}}
\subsubsection[gim\_\-binary\_\-search\_\-ex]{\setlength{\rightskip}{0pt plus 5cm}template$<$class T, typename KEYCLASS, typename COMP\_\-CLASS$>$ bool gim\_\-binary\_\-search\_\-ex (const T $\ast$ {\em \_\-array}, \/  GUINT {\em \_\-start\_\-i}, \/  GUINT {\em \_\-end\_\-i}, \/  GUINT \& {\em \_\-result\_\-index}, \/  const KEYCLASS \& {\em \_\-search\_\-key}, \/  COMP\_\-CLASS {\em \_\-comp\_\-macro})\hspace{0.3cm}{\tt  \mbox{[}inline\mbox{]}}}}
\label{gim__radixsort_8h_7be883eaa7e11dab454089d654537e35}


Failsafe Iterative binary search,. 

If the element is not found, it returns the nearest upper element position, may be the further position after the last element. \begin{Desc}
\item[Parameters:]
\begin{description}
\item[{\em \_\-array}]\item[{\em \_\-start\_\-i}]the beginning of the array \item[{\em \_\-end\_\-i}]the ending index of the array \item[{\em \_\-search\_\-key}]Value to find \item[{\em \_\-comp\_\-macro}]macro for comparing elements \item[{\em \_\-found}]If true the value has found. Boolean \item[{\em \_\-result\_\-index}]the index of the found element, or if not found then it will get the index of the closest bigger value \end{description}
\end{Desc}


Definition at line 273 of file gim\_\-radixsort.h.

Referenced by gim\_\-hash\_\-table$<$ T $>$::\_\-insert\_\-sorted(), and gim\_\-hash\_\-table$<$ T $>$::find().

\begin{Code}\begin{verbatim}278 {
279         GUINT _k;
280         int _comp_result;
281         GUINT _i = _start_i;
282         GUINT _j = _end_i+1;
283         while (_i < _j)
284         {
285                 _k = (_j+_i-1)/2;
286                 _comp_result = _comp_macro(_array[_k], _search_key);
287                 if (_comp_result == 0)
288                 {
289                         _result_index = _k;
290                         return true;
291                 }
292                 else if (_comp_result < 0)
293                 {
294                         _i = _k+1;
295                 }
296                 else
297                 {
298                         _j = _k;
299                 }
300         }
301         _result_index = _i;
302         return false;
303 }
\end{verbatim}
\end{Code}




Here is the caller graph for this function:\hypertarget{gim__radixsort_8h_41009ea63c51a0a22ff221b583dc391d}{
\index{gim\_\-radixsort.h@{gim\_\-radixsort.h}!gim\_\-radix\_\-sort@{gim\_\-radix\_\-sort}}
\index{gim\_\-radix\_\-sort@{gim\_\-radix\_\-sort}!gim_radixsort.h@{gim\_\-radixsort.h}}
\subsubsection[gim\_\-radix\_\-sort]{\setlength{\rightskip}{0pt plus 5cm}template$<$typename T, class GETKEY\_\-CLASS, class COPY\_\-CLASS$>$ void gim\_\-radix\_\-sort (T $\ast$ {\em array}, \/  GUINT {\em element\_\-count}, \/  GETKEY\_\-CLASS {\em get\_\-uintkey\_\-macro}, \/  COPY\_\-CLASS {\em copy\_\-elements\_\-macro})\hspace{0.3cm}{\tt  \mbox{[}inline\mbox{]}}}}
\label{gim__radixsort_8h_41009ea63c51a0a22ff221b583dc391d}


Sorts array in place. For generic use. 

\begin{Desc}
\item[Parameters:]
\begin{description}
\item[{\em type}]Type of the array \item[{\em array}]\item[{\em element\_\-count}]\item[{\em get\_\-uintkey\_\-macro}]Macro for extract the Integer value of the element. Similar to SIMPLE\_\-GET\_\-UINTKEY \item[{\em copy\_\-elements\_\-macro}]Macro for copy elements, similar to SIMPLE\_\-COPY\_\-ELEMENTS \end{description}
\end{Desc}


Definition at line 245 of file gim\_\-radixsort.h.

References gim\_\-radix\_\-sort\_\-array\_\-tokens().

\begin{Code}\begin{verbatim}248 {
249         GIM_RSORT_TOKEN * _sorted = (GIM_RSORT_TOKEN  *) gim_alloc(sizeof(GIM_RSORT_TOKEN)*element_count);
250     gim_radix_sort_array_tokens(array,_sorted,element_count,get_uintkey_macro);
251     T * _original_array = (T *) gim_alloc(sizeof(T)*element_count);
252     gim_simd_memcpy(_original_array,array,sizeof(T)*element_count);
253     for (GUINT _i=0;_i<element_count;++_i)
254     {
255         copy_elements_macro(array[_i],_original_array[_sorted[_i].m_value]);
256     }
257     gim_free(_original_array);
258     gim_free(_sorted);
259 }
\end{verbatim}
\end{Code}




Here is the call graph for this function:\hypertarget{gim__radixsort_8h_dc804c6eb5898638d59eeed4b400f1d5}{
\index{gim\_\-radixsort.h@{gim\_\-radixsort.h}!gim\_\-radix\_\-sort\_\-array\_\-tokens@{gim\_\-radix\_\-sort\_\-array\_\-tokens}}
\index{gim\_\-radix\_\-sort\_\-array\_\-tokens@{gim\_\-radix\_\-sort\_\-array\_\-tokens}!gim_radixsort.h@{gim\_\-radixsort.h}}
\subsubsection[gim\_\-radix\_\-sort\_\-array\_\-tokens]{\setlength{\rightskip}{0pt plus 5cm}template$<$typename T, class GETKEY\_\-CLASS$>$ void gim\_\-radix\_\-sort\_\-array\_\-tokens (T $\ast$ {\em array}, \/  GIM\_\-RSORT\_\-TOKEN $\ast$ {\em sorted\_\-tokens}, \/  GUINT {\em element\_\-count}, \/  GETKEY\_\-CLASS {\em uintkey\_\-macro})\hspace{0.3cm}{\tt  \mbox{[}inline\mbox{]}}}}
\label{gim__radixsort_8h_dc804c6eb5898638d59eeed4b400f1d5}


Get the sorted tokens from an array. For generic use. Tokens are IRR\_\-RSORT\_\-TOKEN. 

\begin{Desc}
\item[Parameters:]
\begin{description}
\item[{\em array}]Array of elements to sort \item[{\em sorted\_\-tokens}]Tokens of sorted elements \item[{\em element\_\-count}]element count \item[{\em uintkey\_\-macro}]Functor which retrieves the integer representation of an array element \end{description}
\end{Desc}


Definition at line 220 of file gim\_\-radixsort.h.

References gim\_\-radix\_\-sort\_\-rtokens().

Referenced by gim\_\-radix\_\-sort().

\begin{Code}\begin{verbatim}224 {
225         GIM_RSORT_TOKEN * _unsorted = (GIM_RSORT_TOKEN *) gim_alloc(sizeof(GIM_RSORT_TOKEN)*element_count);
226     for (GUINT _i=0;_i<element_count;++_i)
227     {
228         _unsorted[_i].m_key = uintkey_macro(array[_i]);
229         _unsorted[_i].m_value = _i;
230     }
231     gim_radix_sort_rtokens(_unsorted,sorted_tokens,element_count);
232     gim_free(_unsorted);
233     gim_free(_unsorted);
234 }
\end{verbatim}
\end{Code}




Here is the call graph for this function:

Here is the caller graph for this function: