\hypertarget{group__gtx__multiple}{
\section{GLM\_\-GTX\_\-multiple}
\label{group__gtx__multiple}\index{GLM\_\-GTX\_\-multiple@{GLM\_\-GTX\_\-multiple}}
}


Collaboration diagram for GLM\_\-GTX\_\-multiple:Find the closest number of a number multiple of other number.  
\subsection*{Functions}
\begin{CompactItemize}
\item 
{\footnotesize template$<$typename genType$>$ }\\GLM\_\-FUNC\_\-DECL genType \hyperlink{group__gtx__multiple_gbd0db4e77a64ac833046cbbeb6fc5bf3}{glm::higherMultiple} (genType const \&Source, genType const \&Multiple)
\item 
{\footnotesize template$<$typename genType$>$ }\\GLM\_\-FUNC\_\-DECL genType \hyperlink{group__gtx__multiple_gecccd82257351764e3c2bec8973458e3}{glm::lowerMultiple} (genType const \&Source, genType const \&Multiple)
\end{CompactItemize}


\subsection{Detailed Description}
Find the closest number of a number multiple of other number. 

$<$glm/gtx/multiple.hpp$>$ need to be included to use these functionalities. 

\subsection{Function Documentation}
\hypertarget{group__gtx__multiple_gbd0db4e77a64ac833046cbbeb6fc5bf3}{
\index{gtx\_\-multiple@{gtx\_\-multiple}!higherMultiple@{higherMultiple}}
\index{higherMultiple@{higherMultiple}!gtx_multiple@{gtx\_\-multiple}}
\subsubsection[higherMultiple]{\setlength{\rightskip}{0pt plus 5cm}template$<$typename genType$>$ GLM\_\-FUNC\_\-QUALIFIER genType glm::higherMultiple (genType const \& {\em Source}, \/  genType const \& {\em Multiple})\hspace{0.3cm}{\tt  \mbox{[}inline\mbox{]}}}}
\label{group__gtx__multiple_gbd0db4e77a64ac833046cbbeb6fc5bf3}


Higher multiple number of Source.

\begin{Desc}
\item[Template Parameters:]
\begin{description}
\item[{\em genType}]Floating-point or integer scalar or vector types. \end{description}
\end{Desc}
\begin{Desc}
\item[Parameters:]
\begin{description}
\item[{\em Source}]\item[{\em Multiple}]Must be a null or positive value\end{description}
\end{Desc}
\begin{Desc}
\item[See also:]\hyperlink{group__gtx__multiple}{GLM\_\-GTX\_\-multiple} \end{Desc}


Definition at line 57 of file multiple.inl.

\begin{Code}\begin{verbatim}61         {
62                 detail::higherMultiple<std::numeric_limits<genType>::is_signed> Compute;
63                 return Compute(Source, Multiple);
64         }
\end{verbatim}
\end{Code}


\hypertarget{group__gtx__multiple_gecccd82257351764e3c2bec8973458e3}{
\index{gtx\_\-multiple@{gtx\_\-multiple}!lowerMultiple@{lowerMultiple}}
\index{lowerMultiple@{lowerMultiple}!gtx_multiple@{gtx\_\-multiple}}
\subsubsection[lowerMultiple]{\setlength{\rightskip}{0pt plus 5cm}template$<$typename genType$>$ GLM\_\-FUNC\_\-QUALIFIER genType glm::lowerMultiple (genType const \& {\em Source}, \/  genType const \& {\em Multiple})\hspace{0.3cm}{\tt  \mbox{[}inline\mbox{]}}}}
\label{group__gtx__multiple_gecccd82257351764e3c2bec8973458e3}


Lower multiple number of Source.

\begin{Desc}
\item[Template Parameters:]
\begin{description}
\item[{\em genType}]Floating-point or integer scalar or vector types. \end{description}
\end{Desc}
\begin{Desc}
\item[Parameters:]
\begin{description}
\item[{\em Source}]\item[{\em Multiple}]Must be a null or positive value\end{description}
\end{Desc}
\begin{Desc}
\item[See also:]\hyperlink{group__gtx__multiple}{GLM\_\-GTX\_\-multiple} \end{Desc}


Definition at line 105 of file multiple.inl.

\begin{Code}\begin{verbatim}109         {
110                 if (Source >= genType(0))
111                         return Source - Source % Multiple;
112                 else
113                 {
114                         genType Tmp = Source + genType(1);
115                         return Tmp - Tmp % Multiple - Multiple;
116                 }
117         }
\end{verbatim}
\end{Code}


