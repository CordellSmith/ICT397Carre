\hypertarget{group__core__func__packing}{
\section{Floating-Point Pack and Unpack Functions}
\label{group__core__func__packing}\index{Floating-Point Pack and Unpack Functions@{Floating-Point Pack and Unpack Functions}}
}


Collaboration diagram for Floating-Point Pack and Unpack Functions:\subsection*{Functions}
\begin{CompactItemize}
\item 
GLM\_\-FUNC\_\-DECL uint \hyperlink{group__core__func__packing_g0659ddaf09727551c7bf51655d2a65cf}{glm::packUnorm2x16} (vec2 const \&v)
\item 
GLM\_\-FUNC\_\-DECL uint \hyperlink{group__core__func__packing_g0c8005de240d6c4ca3d16c7bee25c622}{glm::packSnorm2x16} (vec2 const \&v)
\item 
GLM\_\-FUNC\_\-DECL uint \hyperlink{group__core__func__packing_g834ee9a9e73dcb0a7c1fc88143f3edb8}{glm::packUnorm4x8} (vec4 const \&v)
\item 
GLM\_\-FUNC\_\-DECL uint \hyperlink{group__core__func__packing_gfcf25acc0d361c6c696a433aa5dfd16b}{glm::packSnorm4x8} (vec4 const \&v)
\item 
GLM\_\-FUNC\_\-DECL vec2 \hyperlink{group__core__func__packing_gff327a2fca8abfe31b74b914b68ac5ec}{glm::unpackUnorm2x16} (uint const \&p)
\item 
GLM\_\-FUNC\_\-DECL vec2 \hyperlink{group__core__func__packing_ga3f9bd6a71d7bdfab090b9626f2466aa}{glm::unpackSnorm2x16} (uint const \&p)
\item 
GLM\_\-FUNC\_\-DECL vec4 \hyperlink{group__core__func__packing_g5d3c4d354b48a317935349dd62a8b8a5}{glm::unpackUnorm4x8} (uint const \&p)
\item 
GLM\_\-FUNC\_\-DECL vec4 \hyperlink{group__core__func__packing_g126a0deffef1f2d10dd67237981a870b}{glm::unpackSnorm4x8} (uint const \&p)
\item 
GLM\_\-FUNC\_\-DECL double \hyperlink{group__core__func__packing_gf728fdfb98ce34da6f968d9f6bf154d7}{glm::packDouble2x32} (uvec2 const \&v)
\item 
GLM\_\-FUNC\_\-DECL uvec2 \hyperlink{group__core__func__packing_g7e8cf88c278c18969c99af83bceed024}{glm::unpackDouble2x32} (double const \&v)
\item 
GLM\_\-FUNC\_\-DECL uint \hyperlink{group__core__func__packing_g082f6dd65f73a547ed3067ef00be036f}{glm::packHalf2x16} (vec2 const \&v)
\item 
GLM\_\-FUNC\_\-DECL vec2 \hyperlink{group__core__func__packing_g4051804cc2c930ba4ca73382b79edf1d}{glm::unpackHalf2x16} (uint const \&v)
\end{CompactItemize}


\subsection{Detailed Description}
These functions do not operate component-wise, rather as described in each case. 

\subsection{Function Documentation}
\hypertarget{group__core__func__packing_gf728fdfb98ce34da6f968d9f6bf154d7}{
\index{core\_\-func\_\-packing@{core\_\-func\_\-packing}!packDouble2x32@{packDouble2x32}}
\index{packDouble2x32@{packDouble2x32}!core_func_packing@{core\_\-func\_\-packing}}
\subsubsection[packDouble2x32]{\setlength{\rightskip}{0pt plus 5cm}GLM\_\-FUNC\_\-QUALIFIER double glm::packDouble2x32 (uvec2 const \& {\em v})}}
\label{group__core__func__packing_gf728fdfb98ce34da6f968d9f6bf154d7}


Returns a double-precision value obtained by packing the components of v into a 64-bit value. If an IEEE 754 Inf or NaN is created, it will not signal, and the resulting floating point value is unspecified. Otherwise, the bit- level representation of v is preserved. The first vector component specifies the 32 least significant bits; the second component specifies the 32 most significant bits.

\begin{Desc}
\item[See also:]\href{http://www.opengl.org/sdk/docs/manglsl/xhtml/packDouble2x32.xml}{\tt GLSL packDouble2x32 man page} 

\href{http://www.opengl.org/registry/doc/GLSLangSpec.4.20.8.pdf}{\tt GLSL 4.20.8 specification, section 8.4 Floating-Point Pack and Unpack Functions} \end{Desc}


Definition at line 91 of file func\_\-packing.inl.

\begin{Code}\begin{verbatim}92         {
93                 return reinterpret_cast<double const &>(v);
94         }
\end{verbatim}
\end{Code}


\hypertarget{group__core__func__packing_g082f6dd65f73a547ed3067ef00be036f}{
\index{core\_\-func\_\-packing@{core\_\-func\_\-packing}!packHalf2x16@{packHalf2x16}}
\index{packHalf2x16@{packHalf2x16}!core_func_packing@{core\_\-func\_\-packing}}
\subsubsection[packHalf2x16]{\setlength{\rightskip}{0pt plus 5cm}GLM\_\-FUNC\_\-QUALIFIER uint glm::packHalf2x16 (vec2 const \& {\em v})}}
\label{group__core__func__packing_g082f6dd65f73a547ed3067ef00be036f}


Returns an unsigned integer obtained by converting the components of a two-component floating-point vector to the 16-bit floating-point representation found in the OpenGL Specification, and then packing these two 16- bit integers into a 32-bit unsigned integer. The first vector component specifies the 16 least-significant bits of the result; the second component specifies the 16 most-significant bits.

\begin{Desc}
\item[See also:]\href{http://www.opengl.org/sdk/docs/manglsl/xhtml/packHalf2x16.xml}{\tt GLSL packHalf2x16 man page} 

\href{http://www.opengl.org/registry/doc/GLSLangSpec.4.20.8.pdf}{\tt GLSL 4.20.8 specification, section 8.4 Floating-Point Pack and Unpack Functions} \end{Desc}


Definition at line 101 of file func\_\-packing.inl.

\begin{Code}\begin{verbatim}102         {
103                 i16vec2 Unpack(
104                         detail::toFloat16(v.x),
105                         detail::toFloat16(v.y));
106 
107                 uint * Result = reinterpret_cast<uint*>(&Unpack);
108                 return *Result;
109         }
\end{verbatim}
\end{Code}


\hypertarget{group__core__func__packing_g0c8005de240d6c4ca3d16c7bee25c622}{
\index{core\_\-func\_\-packing@{core\_\-func\_\-packing}!packSnorm2x16@{packSnorm2x16}}
\index{packSnorm2x16@{packSnorm2x16}!core_func_packing@{core\_\-func\_\-packing}}
\subsubsection[packSnorm2x16]{\setlength{\rightskip}{0pt plus 5cm}GLM\_\-FUNC\_\-QUALIFIER uint glm::packSnorm2x16 (vec2 const \& {\em v})}}
\label{group__core__func__packing_g0c8005de240d6c4ca3d16c7bee25c622}


First, converts each component of the normalized floating-point value v into 8- or 16-bit integer values. Then, the results are packed into the returned 32-bit unsigned integer.

The conversion for component c of v to fixed point is done as follows: packSnorm2x16: round(clamp(v, -1, +1) $\ast$ 32767.0)

The first component of the vector will be written to the least significant bits of the output; the last component will be written to the most significant bits.

\begin{Desc}
\item[See also:]\href{http://www.opengl.org/sdk/docs/manglsl/xhtml/packSnorm2x16.xml}{\tt GLSL packSnorm2x16 man page} 

\href{http://www.opengl.org/registry/doc/GLSLangSpec.4.20.8.pdf}{\tt GLSL 4.20.8 specification, section 8.4 Floating-Point Pack and Unpack Functions} \end{Desc}


Definition at line 49 of file func\_\-packing.inl.

References glm::clamp(), and glm::round().

\begin{Code}\begin{verbatim}50         {
51                 i16vec2 Topack(round(clamp(v ,-1.0f, 1.0f) * 32767.0f));
52                 // return reinterpret_cast<uint32&>(Topack);
53     uint* ptr(reinterpret_cast<uint*>(&Topack));
54     return *ptr;
55         }
\end{verbatim}
\end{Code}




Here is the call graph for this function:\hypertarget{group__core__func__packing_gfcf25acc0d361c6c696a433aa5dfd16b}{
\index{core\_\-func\_\-packing@{core\_\-func\_\-packing}!packSnorm4x8@{packSnorm4x8}}
\index{packSnorm4x8@{packSnorm4x8}!core_func_packing@{core\_\-func\_\-packing}}
\subsubsection[packSnorm4x8]{\setlength{\rightskip}{0pt plus 5cm}GLM\_\-FUNC\_\-QUALIFIER uint glm::packSnorm4x8 (vec4 const \& {\em v})}}
\label{group__core__func__packing_gfcf25acc0d361c6c696a433aa5dfd16b}


First, converts each component of the normalized floating-point value v into 8- or 16-bit integer values. Then, the results are packed into the returned 32-bit unsigned integer.

The conversion for component c of v to fixed point is done as follows: packSnorm4x8: round(clamp(c, -1, +1) $\ast$ 127.0)

The first component of the vector will be written to the least significant bits of the output; the last component will be written to the most significant bits.

\begin{Desc}
\item[See also:]\href{http://www.opengl.org/sdk/docs/manglsl/xhtml/packSnorm4x8.xml}{\tt GLSL packSnorm4x8 man page} 

\href{http://www.opengl.org/registry/doc/GLSLangSpec.4.20.8.pdf}{\tt GLSL 4.20.8 specification, section 8.4 Floating-Point Pack and Unpack Functions} \end{Desc}


Definition at line 77 of file func\_\-packing.inl.

References glm::clamp(), and glm::round().

\begin{Code}\begin{verbatim}78         {
79                 i8vec4 Topack(round(clamp(v ,-1.0f, 1.0f) * 127.0f));
80                 return reinterpret_cast<uint&>(Topack);
81         }
\end{verbatim}
\end{Code}




Here is the call graph for this function:\hypertarget{group__core__func__packing_g0659ddaf09727551c7bf51655d2a65cf}{
\index{core\_\-func\_\-packing@{core\_\-func\_\-packing}!packUnorm2x16@{packUnorm2x16}}
\index{packUnorm2x16@{packUnorm2x16}!core_func_packing@{core\_\-func\_\-packing}}
\subsubsection[packUnorm2x16]{\setlength{\rightskip}{0pt plus 5cm}GLM\_\-FUNC\_\-QUALIFIER uint glm::packUnorm2x16 (vec2 const \& {\em v})}}
\label{group__core__func__packing_g0659ddaf09727551c7bf51655d2a65cf}


First, converts each component of the normalized floating-point value v into 8- or 16-bit integer values. Then, the results are packed into the returned 32-bit unsigned integer.

The conversion for component c of v to fixed point is done as follows: packUnorm2x16: round(clamp(c, 0, +1) $\ast$ 65535.0)

The first component of the vector will be written to the least significant bits of the output; the last component will be written to the most significant bits.

\begin{Desc}
\item[See also:]\href{http://www.opengl.org/sdk/docs/manglsl/xhtml/packUnorm2x16.xml}{\tt GLSL packUnorm2x16 man page} 

\href{http://www.opengl.org/registry/doc/GLSLangSpec.4.20.8.pdf}{\tt GLSL 4.20.8 specification, section 8.4 Floating-Point Pack and Unpack Functions} \end{Desc}


Definition at line 35 of file func\_\-packing.inl.

References glm::clamp(), and glm::round().

\begin{Code}\begin{verbatim}36         {
37                 u16vec2 Topack(round(clamp(v, 0.0f, 1.0f) * 65535.0f));
38                 // return reinterpret_cast<uint&>(Topack);
39     uint* ptr(reinterpret_cast<uint*>(&Topack));
40     return *ptr;
41         }
\end{verbatim}
\end{Code}




Here is the call graph for this function:\hypertarget{group__core__func__packing_g834ee9a9e73dcb0a7c1fc88143f3edb8}{
\index{core\_\-func\_\-packing@{core\_\-func\_\-packing}!packUnorm4x8@{packUnorm4x8}}
\index{packUnorm4x8@{packUnorm4x8}!core_func_packing@{core\_\-func\_\-packing}}
\subsubsection[packUnorm4x8]{\setlength{\rightskip}{0pt plus 5cm}GLM\_\-FUNC\_\-QUALIFIER uint glm::packUnorm4x8 (vec4 const \& {\em v})}}
\label{group__core__func__packing_g834ee9a9e73dcb0a7c1fc88143f3edb8}


First, converts each component of the normalized floating-point value v into 8- or 16-bit integer values. Then, the results are packed into the returned 32-bit unsigned integer.

The conversion for component c of v to fixed point is done as follows: packUnorm4x8: round(clamp(c, 0, +1) $\ast$ 255.0)

The first component of the vector will be written to the least significant bits of the output; the last component will be written to the most significant bits.

\begin{Desc}
\item[See also:]\href{http://www.opengl.org/sdk/docs/manglsl/xhtml/packUnorm4x8.xml}{\tt GLSL packUnorm4x8 man page} 

\href{http://www.opengl.org/registry/doc/GLSLangSpec.4.20.8.pdf}{\tt GLSL 4.20.8 specification, section 8.4 Floating-Point Pack and Unpack Functions} \end{Desc}


Definition at line 65 of file func\_\-packing.inl.

References glm::clamp(), and glm::round().

\begin{Code}\begin{verbatim}66         {
67                 u8vec4 Topack(round(clamp(v, 0.0f, 1.0f) * 255.0f));
68                 return reinterpret_cast<uint&>(Topack);
69         }
\end{verbatim}
\end{Code}




Here is the call graph for this function:\hypertarget{group__core__func__packing_g7e8cf88c278c18969c99af83bceed024}{
\index{core\_\-func\_\-packing@{core\_\-func\_\-packing}!unpackDouble2x32@{unpackDouble2x32}}
\index{unpackDouble2x32@{unpackDouble2x32}!core_func_packing@{core\_\-func\_\-packing}}
\subsubsection[unpackDouble2x32]{\setlength{\rightskip}{0pt plus 5cm}GLM\_\-FUNC\_\-QUALIFIER uvec2 glm::unpackDouble2x32 (double const \& {\em v})}}
\label{group__core__func__packing_g7e8cf88c278c18969c99af83bceed024}


Returns a two-component unsigned integer vector representation of v. The bit-level representation of v is preserved. The first component of the vector contains the 32 least significant bits of the double; the second component consists the 32 most significant bits.

\begin{Desc}
\item[See also:]\href{http://www.opengl.org/sdk/docs/manglsl/xhtml/unpackDouble2x32.xml}{\tt GLSL unpackDouble2x32 man page} 

\href{http://www.opengl.org/registry/doc/GLSLangSpec.4.20.8.pdf}{\tt GLSL 4.20.8 specification, section 8.4 Floating-Point Pack and Unpack Functions} \end{Desc}


Definition at line 96 of file func\_\-packing.inl.

\begin{Code}\begin{verbatim}97         {
98                 return reinterpret_cast<uvec2 const &>(v);
99         }
\end{verbatim}
\end{Code}


\hypertarget{group__core__func__packing_g4051804cc2c930ba4ca73382b79edf1d}{
\index{core\_\-func\_\-packing@{core\_\-func\_\-packing}!unpackHalf2x16@{unpackHalf2x16}}
\index{unpackHalf2x16@{unpackHalf2x16}!core_func_packing@{core\_\-func\_\-packing}}
\subsubsection[unpackHalf2x16]{\setlength{\rightskip}{0pt plus 5cm}GLM\_\-FUNC\_\-QUALIFIER vec2 glm::unpackHalf2x16 (uint const \& {\em v})}}
\label{group__core__func__packing_g4051804cc2c930ba4ca73382b79edf1d}


Returns a two-component floating-point vector with components obtained by unpacking a 32-bit unsigned integer into a pair of 16-bit values, interpreting those values as 16-bit floating-point numbers according to the OpenGL Specification, and converting them to 32-bit floating-point values. The first component of the vector is obtained from the 16 least-significant bits of v; the second component is obtained from the 16 most-significant bits of v.

\begin{Desc}
\item[See also:]\href{http://www.opengl.org/sdk/docs/manglsl/xhtml/unpackHalf2x16.xml}{\tt GLSL unpackHalf2x16 man page} 

\href{http://www.opengl.org/registry/doc/GLSLangSpec.4.20.8.pdf}{\tt GLSL 4.20.8 specification, section 8.4 Floating-Point Pack and Unpack Functions} \end{Desc}


Definition at line 111 of file func\_\-packing.inl.

\begin{Code}\begin{verbatim}112         {
113                 i16vec2 Unpack(reinterpret_cast<i16vec2 const &>(v));
114         
115                 return vec2(
116                         detail::toFloat32(Unpack.x), 
117                         detail::toFloat32(Unpack.y));
118         }
\end{verbatim}
\end{Code}


\hypertarget{group__core__func__packing_ga3f9bd6a71d7bdfab090b9626f2466aa}{
\index{core\_\-func\_\-packing@{core\_\-func\_\-packing}!unpackSnorm2x16@{unpackSnorm2x16}}
\index{unpackSnorm2x16@{unpackSnorm2x16}!core_func_packing@{core\_\-func\_\-packing}}
\subsubsection[unpackSnorm2x16]{\setlength{\rightskip}{0pt plus 5cm}GLM\_\-FUNC\_\-QUALIFIER vec2 glm::unpackSnorm2x16 (uint const \& {\em p})}}
\label{group__core__func__packing_ga3f9bd6a71d7bdfab090b9626f2466aa}


First, unpacks a single 32-bit unsigned integer p into a pair of 16-bit unsigned integers, four 8-bit unsigned integers, or four 8-bit signed integers. Then, each component is converted to a normalized floating-point value to generate the returned two- or four-component vector.

The conversion for unpacked fixed-point value f to floating point is done as follows: unpackSnorm2x16: clamp(f / 32767.0, -1, +1)

The first component of the returned vector will be extracted from the least significant bits of the input; the last component will be extracted from the most significant bits.

\begin{Desc}
\item[See also:]\href{http://www.opengl.org/sdk/docs/manglsl/xhtml/unpackSnorm2x16.xml}{\tt GLSL unpackSnorm2x16 man page} 

\href{http://www.opengl.org/registry/doc/GLSLangSpec.4.20.8.pdf}{\tt GLSL 4.20.8 specification, section 8.4 Floating-Point Pack and Unpack Functions} \end{Desc}


Definition at line 57 of file func\_\-packing.inl.

References glm::clamp(), and glm::e().

\begin{Code}\begin{verbatim}58         {
59                 vec2 Unpack(reinterpret_cast<i16vec2 const &>(p));
60                 return clamp(
61                         Unpack * 3.0518509475997192297128208258309e-5f, //1.0f / 32767.0f,
62                         -1.0f, 1.0f);
63         }
\end{verbatim}
\end{Code}




Here is the call graph for this function:\hypertarget{group__core__func__packing_g126a0deffef1f2d10dd67237981a870b}{
\index{core\_\-func\_\-packing@{core\_\-func\_\-packing}!unpackSnorm4x8@{unpackSnorm4x8}}
\index{unpackSnorm4x8@{unpackSnorm4x8}!core_func_packing@{core\_\-func\_\-packing}}
\subsubsection[unpackSnorm4x8]{\setlength{\rightskip}{0pt plus 5cm}GLM\_\-FUNC\_\-QUALIFIER {\bf glm::vec4} glm::unpackSnorm4x8 (uint const \& {\em p})}}
\label{group__core__func__packing_g126a0deffef1f2d10dd67237981a870b}


First, unpacks a single 32-bit unsigned integer p into a pair of 16-bit unsigned integers, four 8-bit unsigned integers, or four 8-bit signed integers. Then, each component is converted to a normalized floating-point value to generate the returned two- or four-component vector.

The conversion for unpacked fixed-point value f to floating point is done as follows: unpackSnorm4x8: clamp(f / 127.0, -1, +1)

The first component of the returned vector will be extracted from the least significant bits of the input; the last component will be extracted from the most significant bits.

\begin{Desc}
\item[See also:]\href{http://www.opengl.org/sdk/docs/manglsl/xhtml/unpackSnorm4x8.xml}{\tt GLSL unpackSnorm4x8 man page} 

\href{http://www.opengl.org/registry/doc/GLSLangSpec.4.20.8.pdf}{\tt GLSL 4.20.8 specification, section 8.4 Floating-Point Pack and Unpack Functions} \end{Desc}


Definition at line 83 of file func\_\-packing.inl.

References glm::clamp().

\begin{Code}\begin{verbatim}84         {
85                 vec4 Unpack(reinterpret_cast<i8vec4 const &>(p));
86                 return clamp(
87                         Unpack * 0.0078740157480315f, // 1.0f / 127.0f
88                         -1.0f, 1.0f);
89         }
\end{verbatim}
\end{Code}




Here is the call graph for this function:\hypertarget{group__core__func__packing_gff327a2fca8abfe31b74b914b68ac5ec}{
\index{core\_\-func\_\-packing@{core\_\-func\_\-packing}!unpackUnorm2x16@{unpackUnorm2x16}}
\index{unpackUnorm2x16@{unpackUnorm2x16}!core_func_packing@{core\_\-func\_\-packing}}
\subsubsection[unpackUnorm2x16]{\setlength{\rightskip}{0pt plus 5cm}GLM\_\-FUNC\_\-QUALIFIER vec2 glm::unpackUnorm2x16 (uint const \& {\em p})}}
\label{group__core__func__packing_gff327a2fca8abfe31b74b914b68ac5ec}


First, unpacks a single 32-bit unsigned integer p into a pair of 16-bit unsigned integers, four 8-bit unsigned integers, or four 8-bit signed integers. Then, each component is converted to a normalized floating-point value to generate the returned two- or four-component vector.

The conversion for unpacked fixed-point value f to floating point is done as follows: unpackUnorm2x16: f / 65535.0

The first component of the returned vector will be extracted from the least significant bits of the input; the last component will be extracted from the most significant bits.

\begin{Desc}
\item[See also:]\href{http://www.opengl.org/sdk/docs/manglsl/xhtml/unpackUnorm2x16.xml}{\tt GLSL unpackUnorm2x16 man page} 

\href{http://www.opengl.org/registry/doc/GLSLangSpec.4.20.8.pdf}{\tt GLSL 4.20.8 specification, section 8.4 Floating-Point Pack and Unpack Functions} \end{Desc}


Definition at line 43 of file func\_\-packing.inl.

References glm::e().

\begin{Code}\begin{verbatim}44         {
45                 vec2 Unpack(reinterpret_cast<u16vec2 const &>(p));
46                 return Unpack * float(1.5259021896696421759365224689097e-5); // 1.0 / 65535.0
47         }
\end{verbatim}
\end{Code}




Here is the call graph for this function:\hypertarget{group__core__func__packing_g5d3c4d354b48a317935349dd62a8b8a5}{
\index{core\_\-func\_\-packing@{core\_\-func\_\-packing}!unpackUnorm4x8@{unpackUnorm4x8}}
\index{unpackUnorm4x8@{unpackUnorm4x8}!core_func_packing@{core\_\-func\_\-packing}}
\subsubsection[unpackUnorm4x8]{\setlength{\rightskip}{0pt plus 5cm}GLM\_\-FUNC\_\-QUALIFIER vec4 glm::unpackUnorm4x8 (uint const \& {\em p})}}
\label{group__core__func__packing_g5d3c4d354b48a317935349dd62a8b8a5}


First, unpacks a single 32-bit unsigned integer p into a pair of 16-bit unsigned integers, four 8-bit unsigned integers, or four 8-bit signed integers. Then, each component is converted to a normalized floating-point value to generate the returned two- or four-component vector.

The conversion for unpacked fixed-point value f to floating point is done as follows: unpackUnorm4x8: f / 255.0

The first component of the returned vector will be extracted from the least significant bits of the input; the last component will be extracted from the most significant bits.

\begin{Desc}
\item[See also:]\href{http://www.opengl.org/sdk/docs/manglsl/xhtml/unpackUnorm4x8.xml}{\tt GLSL unpackUnorm4x8 man page} 

\href{http://www.opengl.org/registry/doc/GLSLangSpec.4.20.8.pdf}{\tt GLSL 4.20.8 specification, section 8.4 Floating-Point Pack and Unpack Functions} \end{Desc}


Definition at line 71 of file func\_\-packing.inl.

\begin{Code}\begin{verbatim}72         {
73                 vec4 Unpack(reinterpret_cast<u8vec4 const&>(p));
74                 return Unpack * float(0.0039215686274509803921568627451); // 1 / 255
75         }
\end{verbatim}
\end{Code}


