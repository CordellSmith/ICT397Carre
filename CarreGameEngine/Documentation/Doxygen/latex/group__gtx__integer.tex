\hypertarget{group__gtx__integer}{
\section{GLM\_\-GTX\_\-integer}
\label{group__gtx__integer}\index{GLM\_\-GTX\_\-integer@{GLM\_\-GTX\_\-integer}}
}


Collaboration diagram for GLM\_\-GTX\_\-integer:Add support for integer for core functions.  
\subsection*{Typedefs}
\begin{CompactItemize}
\item 
typedef signed int \hyperlink{group__gtx__integer_g73643e09d8c6d362418aec541fdb987d}{glm::sint}
\end{CompactItemize}
\subsection*{Functions}
\begin{CompactItemize}
\item 
GLM\_\-FUNC\_\-DECL int \hyperlink{group__gtx__integer_g9642514a44a67afa70966d756f040ca9}{glm::pow} (int x, int y)
\item 
GLM\_\-FUNC\_\-DECL int \hyperlink{group__gtx__integer_g78e2e68330e91d350fcfc2f4831cad12}{glm::sqrt} (int x)
\item 
{\footnotesize template$<$typename genIUType$>$ }\\GLM\_\-FUNC\_\-DECL genIUType \hyperlink{group__gtx__integer_g43dcff81ada0f7d4a29b25ca2a0cef2f}{glm::log2} (genIUType x)
\item 
GLM\_\-FUNC\_\-DECL unsigned int \hyperlink{group__gtx__integer_g7011b4e1c1e1ed492149b028feacc00e}{glm::floor\_\-log2} (unsigned int x)
\item 
GLM\_\-FUNC\_\-DECL int \hyperlink{group__gtx__integer_gb9d22df91aac4d9eb925a4910f556f1b}{glm::mod} (int x, int y)
\item 
{\footnotesize template$<$typename genType$>$ }\\GLM\_\-FUNC\_\-DECL genType \hyperlink{group__gtx__integer_g57ba2a6a2729f23ba4848bbad551dcd1}{glm::factorial} (genType const \&x)
\item 
GLM\_\-FUNC\_\-DECL uint \hyperlink{group__gtx__integer_ga8229e850c3cc4ad83492fe390ada044}{glm::pow} (uint x, uint y)
\item 
GLM\_\-FUNC\_\-DECL uint \hyperlink{group__gtx__integer_g457e9efca8339bf918d319e9c55f7c8f}{glm::sqrt} (uint x)
\item 
GLM\_\-FUNC\_\-DECL uint \hyperlink{group__gtx__integer_gb8f9ec0ca93ca90669434224818f0750}{glm::mod} (uint x, uint y)
\item 
GLM\_\-FUNC\_\-DECL uint \hyperlink{group__gtx__integer_gcbe62fd2384464c16ea30ecc4defc11c}{glm::nlz} (uint x)
\end{CompactItemize}


\subsection{Detailed Description}
Add support for integer for core functions. 

$<$glm/gtx/integer.hpp$>$ need to be included to use these functionalities. 

\subsection{Typedef Documentation}
\hypertarget{group__gtx__integer_g73643e09d8c6d362418aec541fdb987d}{
\index{gtx\_\-integer@{gtx\_\-integer}!sint@{sint}}
\index{sint@{sint}!gtx_integer@{gtx\_\-integer}}
\subsubsection[sint]{\setlength{\rightskip}{0pt plus 5cm}typedef signed int {\bf glm::sint}}}
\label{group__gtx__integer_g73643e09d8c6d362418aec541fdb987d}


32bit signed integer. From GLM\_\-GTX\_\-integer extension. 

Definition at line 81 of file integer.hpp.

\subsection{Function Documentation}
\hypertarget{group__gtx__integer_g57ba2a6a2729f23ba4848bbad551dcd1}{
\index{gtx\_\-integer@{gtx\_\-integer}!factorial@{factorial}}
\index{factorial@{factorial}!gtx_integer@{gtx\_\-integer}}
\subsubsection[factorial]{\setlength{\rightskip}{0pt plus 5cm}template$<$typename genType$>$ GLM\_\-FUNC\_\-QUALIFIER genType glm::factorial (genType const \& {\em x})\hspace{0.3cm}{\tt  \mbox{[}inline\mbox{]}}}}
\label{group__gtx__integer_g57ba2a6a2729f23ba4848bbad551dcd1}


Return the factorial value of a number (!12 max, integer only) From GLM\_\-GTX\_\-integer extension. 

Definition at line 93 of file integer.inl.

\begin{Code}\begin{verbatim}94         {
95                 genType Temp = x;
96                 genType Result;
97                 for(Result = 1; Temp > 1; --Temp)
98                         Result *= Temp;
99                 return Result;
100         }
\end{verbatim}
\end{Code}


\hypertarget{group__gtx__integer_g7011b4e1c1e1ed492149b028feacc00e}{
\index{gtx\_\-integer@{gtx\_\-integer}!floor\_\-log2@{floor\_\-log2}}
\index{floor\_\-log2@{floor\_\-log2}!gtx_integer@{gtx\_\-integer}}
\subsubsection[floor\_\-log2]{\setlength{\rightskip}{0pt plus 5cm}GLM\_\-FUNC\_\-DECL unsigned int glm::floor\_\-log2 (unsigned int {\em x})}}
\label{group__gtx__integer_g7011b4e1c1e1ed492149b028feacc00e}


Returns the floor log2 of x. From GLM\_\-GTX\_\-integer extension. \hypertarget{group__gtx__integer_g43dcff81ada0f7d4a29b25ca2a0cef2f}{
\index{gtx\_\-integer@{gtx\_\-integer}!log2@{log2}}
\index{log2@{log2}!gtx_integer@{gtx\_\-integer}}
\subsubsection[log2]{\setlength{\rightskip}{0pt plus 5cm}template$<$typename genIUType$>$ GLM\_\-FUNC\_\-DECL genIUType glm::log2 (genIUType {\em x})\hspace{0.3cm}{\tt  \mbox{[}inline\mbox{]}}}}
\label{group__gtx__integer_g43dcff81ada0f7d4a29b25ca2a0cef2f}


Returns the log2 of x. Can be reliably using to compute mipmap count from the texture size. From GLM\_\-GTX\_\-integer extension. \hypertarget{group__gtx__integer_gb8f9ec0ca93ca90669434224818f0750}{
\index{gtx\_\-integer@{gtx\_\-integer}!mod@{mod}}
\index{mod@{mod}!gtx_integer@{gtx\_\-integer}}
\subsubsection[mod]{\setlength{\rightskip}{0pt plus 5cm}GLM\_\-FUNC\_\-QUALIFIER uint glm::mod (uint {\em x}, \/  uint {\em y})}}
\label{group__gtx__integer_gb8f9ec0ca93ca90669434224818f0750}


Modulus. Returns x - y $\ast$ floor(x / y) for each component in x using the floating point value y. From GLM\_\-GTX\_\-integer extension. 

Definition at line 156 of file integer.inl.

\begin{Code}\begin{verbatim}157         {
158                 return x - y * (x / y);
159         }
\end{verbatim}
\end{Code}


\hypertarget{group__gtx__integer_gb9d22df91aac4d9eb925a4910f556f1b}{
\index{gtx\_\-integer@{gtx\_\-integer}!mod@{mod}}
\index{mod@{mod}!gtx_integer@{gtx\_\-integer}}
\subsubsection[mod]{\setlength{\rightskip}{0pt plus 5cm}GLM\_\-FUNC\_\-QUALIFIER int glm::mod (int {\em x}, \/  int {\em y})}}
\label{group__gtx__integer_gb9d22df91aac4d9eb925a4910f556f1b}


Modulus. Returns x - y $\ast$ floor(x / y) for each component in x using the floating point value y. From GLM\_\-GTX\_\-integer extension. 

Definition at line 86 of file integer.inl.

\begin{Code}\begin{verbatim}87         {
88                 return x - y * (x / y);
89         }
\end{verbatim}
\end{Code}


\hypertarget{group__gtx__integer_gcbe62fd2384464c16ea30ecc4defc11c}{
\index{gtx\_\-integer@{gtx\_\-integer}!nlz@{nlz}}
\index{nlz@{nlz}!gtx_integer@{gtx\_\-integer}}
\subsubsection[nlz]{\setlength{\rightskip}{0pt plus 5cm}GLM\_\-FUNC\_\-QUALIFIER unsigned int glm::nlz (uint {\em x})}}
\label{group__gtx__integer_gcbe62fd2384464c16ea30ecc4defc11c}


Returns the number of leading zeros. From GLM\_\-GTX\_\-integer extension. 

Definition at line 171 of file integer.inl.

\begin{Code}\begin{verbatim}172         {
173                 int y, m, n;
174 
175                 y = -int(x >> 16);      // If left half of x is 0,
176                 m = (y >> 16) & 16;  // set n = 16.  If left half
177                 n = 16 - m;          // is nonzero, set n = 0 and
178                 x = x >> m;          // shift x right 16.
179                                                         // Now x is of the form 0000xxxx.
180                 y = x - 0x100;       // If positions 8-15 are 0,
181                 m = (y >> 16) & 8;   // add 8 to n and shift x left 8.
182                 n = n + m;
183                 x = x << m;
184 
185                 y = x - 0x1000;      // If positions 12-15 are 0,
186                 m = (y >> 16) & 4;   // add 4 to n and shift x left 4.
187                 n = n + m;
188                 x = x << m;
189 
190                 y = x - 0x4000;      // If positions 14-15 are 0,
191                 m = (y >> 16) & 2;   // add 2 to n and shift x left 2.
192                 n = n + m;
193                 x = x << m;
194 
195                 y = x >> 14;         // Set y = 0, 1, 2, or 3.
196                 m = y & ~(y >> 1);   // Set m = 0, 1, 2, or 2 resp.
197                 return unsigned(n + 2 - m);
198         }
\end{verbatim}
\end{Code}


\hypertarget{group__gtx__integer_ga8229e850c3cc4ad83492fe390ada044}{
\index{gtx\_\-integer@{gtx\_\-integer}!pow@{pow}}
\index{pow@{pow}!gtx_integer@{gtx\_\-integer}}
\subsubsection[pow]{\setlength{\rightskip}{0pt plus 5cm}GLM\_\-FUNC\_\-QUALIFIER uint glm::pow (uint {\em x}, \/  uint {\em y})}}
\label{group__gtx__integer_ga8229e850c3cc4ad83492fe390ada044}


Returns x raised to the y power. From GLM\_\-GTX\_\-integer extension. 

Definition at line 132 of file integer.inl.

\begin{Code}\begin{verbatim}133         {
134                 uint result = x;
135                 for(uint i = 1; i < y; ++i)
136                         result *= x;
137                 return result;
138         }
\end{verbatim}
\end{Code}


\hypertarget{group__gtx__integer_g9642514a44a67afa70966d756f040ca9}{
\index{gtx\_\-integer@{gtx\_\-integer}!pow@{pow}}
\index{pow@{pow}!gtx_integer@{gtx\_\-integer}}
\subsubsection[pow]{\setlength{\rightskip}{0pt plus 5cm}GLM\_\-FUNC\_\-QUALIFIER int glm::pow (int {\em x}, \/  int {\em y})}}
\label{group__gtx__integer_g9642514a44a67afa70966d756f040ca9}


Returns x raised to the y power. From GLM\_\-GTX\_\-integer extension. 

Definition at line 13 of file integer.inl.

\begin{Code}\begin{verbatim}14         {
15                 if(y == 0)
16                         return 1;
17                 int result = x;
18                 for(int i = 1; i < y; ++i)
19                         result *= x;
20                 return result;
21         }
\end{verbatim}
\end{Code}


\hypertarget{group__gtx__integer_g457e9efca8339bf918d319e9c55f7c8f}{
\index{gtx\_\-integer@{gtx\_\-integer}!sqrt@{sqrt}}
\index{sqrt@{sqrt}!gtx_integer@{gtx\_\-integer}}
\subsubsection[sqrt]{\setlength{\rightskip}{0pt plus 5cm}GLM\_\-FUNC\_\-QUALIFIER uint glm::sqrt (uint {\em x})}}
\label{group__gtx__integer_g457e9efca8339bf918d319e9c55f7c8f}


Returns the positive square root of x. From GLM\_\-GTX\_\-integer extension. 

Definition at line 140 of file integer.inl.

Referenced by glm::axis(), Vector2::Distance(), glm::quat\_\-cast(), and glm::refract().

\begin{Code}\begin{verbatim}141         {
142                 if(x <= 1) return x;
143 
144                 uint NextTrial = x >> 1;
145                 uint CurrentAnswer;
146 
147                 do
148                 {
149                         CurrentAnswer = NextTrial;
150                         NextTrial = (NextTrial + x / NextTrial) >> 1;
151                 } while(NextTrial < CurrentAnswer);
152 
153                 return CurrentAnswer;
154         }
\end{verbatim}
\end{Code}




Here is the caller graph for this function:\hypertarget{group__gtx__integer_g78e2e68330e91d350fcfc2f4831cad12}{
\index{gtx\_\-integer@{gtx\_\-integer}!sqrt@{sqrt}}
\index{sqrt@{sqrt}!gtx_integer@{gtx\_\-integer}}
\subsubsection[sqrt]{\setlength{\rightskip}{0pt plus 5cm}GLM\_\-FUNC\_\-QUALIFIER int glm::sqrt (int {\em x})}}
\label{group__gtx__integer_g78e2e68330e91d350fcfc2f4831cad12}


Returns the positive square root of x. From GLM\_\-GTX\_\-integer extension. 

Definition at line 24 of file integer.inl.

\begin{Code}\begin{verbatim}25         {
26                 if(x <= 1) return x;
27 
28                 int NextTrial = x >> 1;
29                 int CurrentAnswer;
30 
31                 do
32                 {
33                         CurrentAnswer = NextTrial;
34                         NextTrial = (NextTrial + x / NextTrial) >> 1;
35                 } while(NextTrial < CurrentAnswer);
36 
37                 return CurrentAnswer;
38         }
\end{verbatim}
\end{Code}


