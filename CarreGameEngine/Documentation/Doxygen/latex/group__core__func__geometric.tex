\hypertarget{group__core__func__geometric}{
\section{Geometric functions}
\label{group__core__func__geometric}\index{Geometric functions@{Geometric functions}}
}


Collaboration diagram for Geometric functions:\subsection*{Functions}
\begin{CompactItemize}
\item 
{\footnotesize template$<$typename genType$>$ }\\GLM\_\-FUNC\_\-DECL genType::value\_\-type \hyperlink{group__core__func__geometric_gd73a94d9c967e619e670156356e93b7e}{glm::length} (genType const \&x)
\item 
{\footnotesize template$<$typename genType$>$ }\\GLM\_\-FUNC\_\-DECL genType::value\_\-type \hyperlink{group__core__func__geometric_gd21e00cab9f8b4eb6d1214a16dee06c7}{glm::distance} (genType const \&p0, genType const \&p1)
\item 
{\footnotesize template$<$typename T, precision P, template$<$ typename, precision $>$ class vecType$>$ }\\GLM\_\-FUNC\_\-DECL T \hyperlink{group__core__func__geometric_gc64a3b29d01336161a668d328cac97eb}{glm::dot} (vecType$<$ T, P $>$ const \&x, vecType$<$ T, P $>$ const \&y)
\item 
{\footnotesize template$<$typename genType$>$ }\\GLM\_\-FUNC\_\-DECL genType \hyperlink{group__core__func__geometric_g93acafc6005a3433ccf1dc3fa4230d51}{glm::dot} (genType const \&x, genType const \&y)
\item 
{\footnotesize template$<$typename T, precision P$>$ }\\GLM\_\-FUNC\_\-DECL detail::tvec3$<$ T, P $>$ \hyperlink{group__core__func__geometric_g9a325364ff3650c4a85c33704c646e76}{glm::cross} (detail::tvec3$<$ T, P $>$ const \&x, detail::tvec3$<$ T, P $>$ const \&y)
\item 
{\footnotesize template$<$typename genType$>$ }\\GLM\_\-FUNC\_\-DECL genType \hyperlink{group__core__func__geometric_g0feb2bb89ee2743677ad2cb84544bd83}{glm::normalize} (genType const \&x)
\item 
{\footnotesize template$<$typename genType$>$ }\\GLM\_\-FUNC\_\-DECL genType \hyperlink{group__core__func__geometric_ga4cdf87f6fd660e7086776d9abb6cbae}{glm::faceforward} (genType const \&N, genType const \&I, genType const \&Nref)
\item 
{\footnotesize template$<$typename genType$>$ }\\GLM\_\-FUNC\_\-DECL genType \hyperlink{group__core__func__geometric_gc973ce2bc49f749a469d3ed2e2ac5a54}{glm::reflect} (genType const \&I, genType const \&N)
\item 
{\footnotesize template$<$typename T, precision P, template$<$ typename, precision $>$ class vecType$>$ }\\GLM\_\-FUNC\_\-DECL vecType$<$ T, P $>$ \hyperlink{group__core__func__geometric_g2020e01c134ebe03c1690306ff93af53}{glm::refract} (vecType$<$ T, P $>$ const \&I, vecType$<$ T, P $>$ const \&N, T const \&eta)
\end{CompactItemize}


\subsection{Detailed Description}
These operate on vectors as vectors, not component-wise. 

\subsection{Function Documentation}
\hypertarget{group__core__func__geometric_g9a325364ff3650c4a85c33704c646e76}{
\index{core\_\-func\_\-geometric@{core\_\-func\_\-geometric}!cross@{cross}}
\index{cross@{cross}!core_func_geometric@{core\_\-func\_\-geometric}}
\subsubsection[cross]{\setlength{\rightskip}{0pt plus 5cm}template$<$typename T, precision P$>$ GLM\_\-FUNC\_\-QUALIFIER detail::tvec3$<$ T, P $>$ glm::cross (detail::tvec3$<$ T, P $>$ const \& {\em x}, \/  detail::tvec3$<$ T, P $>$ const \& {\em y})\hspace{0.3cm}{\tt  \mbox{[}inline\mbox{]}}}}
\label{group__core__func__geometric_g9a325364ff3650c4a85c33704c646e76}


Returns the cross product of x and y.

\begin{Desc}
\item[Template Parameters:]
\begin{description}
\item[{\em valType}]Floating-point scalar types.\end{description}
\end{Desc}
\begin{Desc}
\item[See also:]\href{http://www.opengl.org/sdk/docs/manglsl/xhtml/cross.xml}{\tt GLSL cross man page} 

\href{http://www.opengl.org/registry/doc/GLSLangSpec.4.20.8.pdf}{\tt GLSL 4.20.8 specification, section 8.5 Geometric Functions} \end{Desc}


Definition at line 217 of file func\_\-geometric.inl.

Referenced by glm::intersectLineTriangle(), glm::intersectRayTriangle(), glm::leftHanded(), glm::lookAt(), glm::mixedProduct(), glm::orientation(), glm::orientedAngle(), glm::rightHanded(), glm::rotation(), and glm::triangleNormal().

\begin{Code}\begin{verbatim}221         {
222                 GLM_STATIC_ASSERT(std::numeric_limits<T>::is_iec559, "'cross' only accept floating-point inputs");
223 
224                 return detail::tvec3<T, P>(
225                         x.y * y.z - y.y * x.z,
226                         x.z * y.x - y.z * x.x,
227                         x.x * y.y - y.x * x.y);
228         }
\end{verbatim}
\end{Code}




Here is the caller graph for this function:\hypertarget{group__core__func__geometric_gd21e00cab9f8b4eb6d1214a16dee06c7}{
\index{core\_\-func\_\-geometric@{core\_\-func\_\-geometric}!distance@{distance}}
\index{distance@{distance}!core_func_geometric@{core\_\-func\_\-geometric}}
\subsubsection[distance]{\setlength{\rightskip}{0pt plus 5cm}template$<$typename genType$>$ GLM\_\-FUNC\_\-QUALIFIER genType glm::distance (genType const \& {\em p0}, \/  genType const \& {\em p1})\hspace{0.3cm}{\tt  \mbox{[}inline\mbox{]}}}}
\label{group__core__func__geometric_gd21e00cab9f8b4eb6d1214a16dee06c7}


Returns the distance betwwen p0 and p1, i.e., length(p0 - p1).

\begin{Desc}
\item[Template Parameters:]
\begin{description}
\item[{\em genType}]Floating-point vector types.\end{description}
\end{Desc}
\begin{Desc}
\item[See also:]\href{http://www.opengl.org/sdk/docs/manglsl/xhtml/distance.xml}{\tt GLSL distance man page} 

\href{http://www.opengl.org/registry/doc/GLSLangSpec.4.20.8.pdf}{\tt GLSL 4.20.8 specification, section 8.5 Geometric Functions} \end{Desc}


Definition at line 128 of file func\_\-geometric.inl.

References glm::length().

Referenced by glm::closestPointOnLine(), btConvexPlaneCollisionAlgorithm::collideSingleContact(), btGjkPairDetector::getClosestPointsNonVirtual(), glm::intersectRaySphere(), and btConvexPlaneCollisionAlgorithm::processCollision().

\begin{Code}\begin{verbatim}132         {
133                 GLM_STATIC_ASSERT(std::numeric_limits<genType>::is_iec559, "'distance' only accept floating-point inputs");
134 
135                 return length(p1 - p0);
136         }
\end{verbatim}
\end{Code}




Here is the call graph for this function:

Here is the caller graph for this function:\hypertarget{group__core__func__geometric_g93acafc6005a3433ccf1dc3fa4230d51}{
\index{core\_\-func\_\-geometric@{core\_\-func\_\-geometric}!dot@{dot}}
\index{dot@{dot}!core_func_geometric@{core\_\-func\_\-geometric}}
\subsubsection[dot]{\setlength{\rightskip}{0pt plus 5cm}template$<$typename genType$>$ GLM\_\-FUNC\_\-DECL genType glm::dot (genType const \& {\em x}, \/  genType const \& {\em y})\hspace{0.3cm}{\tt  \mbox{[}inline\mbox{]}}}}
\label{group__core__func__geometric_g93acafc6005a3433ccf1dc3fa4230d51}


Returns the dot product of x and y, i.e., result = x $\ast$ y.

\begin{Desc}
\item[Template Parameters:]
\begin{description}
\item[{\em genType}]Floating-point vector types.\end{description}
\end{Desc}
\begin{Desc}
\item[See also:]\href{http://www.opengl.org/sdk/docs/manglsl/xhtml/dot.xml}{\tt GLSL dot man page} 

\href{http://www.opengl.org/registry/doc/GLSLangSpec.4.20.8.pdf}{\tt GLSL 4.20.8 specification, section 8.5 Geometric Functions} \end{Desc}
\hypertarget{group__core__func__geometric_gc64a3b29d01336161a668d328cac97eb}{
\index{core\_\-func\_\-geometric@{core\_\-func\_\-geometric}!dot@{dot}}
\index{dot@{dot}!core_func_geometric@{core\_\-func\_\-geometric}}
\subsubsection[dot]{\setlength{\rightskip}{0pt plus 5cm}template$<$typename T, precision P, template$<$ typename, precision $>$ class vecType$>$ GLM\_\-FUNC\_\-QUALIFIER T glm::dot (vecType$<$ T, P $>$ const \& {\em x}, \/  vecType$<$ T, P $>$ const \& {\em y})\hspace{0.3cm}{\tt  \mbox{[}inline\mbox{]}}}}
\label{group__core__func__geometric_gc64a3b29d01336161a668d328cac97eb}


Returns the dot product of x and y, i.e., result = x $\ast$ y.

\begin{Desc}
\item[Template Parameters:]
\begin{description}
\item[{\em genType}]Floating-point vector types.\end{description}
\end{Desc}
\begin{Desc}
\item[See also:]\href{http://www.opengl.org/sdk/docs/manglsl/xhtml/dot.xml}{\tt GLSL dot man page} 

\href{http://www.opengl.org/registry/doc/GLSLangSpec.4.20.8.pdf}{\tt GLSL 4.20.8 specification, section 8.5 Geometric Functions} \end{Desc}


Definition at line 188 of file func\_\-geometric.inl.

Referenced by glm::areOrthogonal(), glm::areOrthonormal(), glm::closestPointOnLine(), glm::faceforward(), glm::fastNormalizeDot(), glm::intersectLineSphere(), glm::intersectLineTriangle(), glm::intersectRayPlane(), glm::intersectRaySphere(), glm::intersectRayTriangle(), glm::inverse(), glm::leftHanded(), glm::lerp(), glm::linearGradient(), glm::lookAt(), glm::luminosity(), glm::mixedProduct(), glm::normalizeDot(), glm::orientation(), glm::orientedAngle(), glm::orthonormalize(), glm::proj(), glm::reflect(), glm::refract(), glm::rightHanded(), glm::rotation(), and glm::shortMix().

\begin{Code}\begin{verbatim}192         {
193                 GLM_STATIC_ASSERT(std::numeric_limits<T>::is_iec559, "'dot' only accept floating-point inputs");
194                 return detail::compute_dot<vecType, T, P>::call(x, y);
195         }
\end{verbatim}
\end{Code}




Here is the caller graph for this function:\hypertarget{group__core__func__geometric_ga4cdf87f6fd660e7086776d9abb6cbae}{
\index{core\_\-func\_\-geometric@{core\_\-func\_\-geometric}!faceforward@{faceforward}}
\index{faceforward@{faceforward}!core_func_geometric@{core\_\-func\_\-geometric}}
\subsubsection[faceforward]{\setlength{\rightskip}{0pt plus 5cm}template$<$typename genType$>$ GLM\_\-FUNC\_\-QUALIFIER genType glm::faceforward (genType const \& {\em N}, \/  genType const \& {\em I}, \/  genType const \& {\em Nref})\hspace{0.3cm}{\tt  \mbox{[}inline\mbox{]}}}}
\label{group__core__func__geometric_ga4cdf87f6fd660e7086776d9abb6cbae}


If dot(Nref, I) $<$ 0.0, return N, otherwise, return -N.

\begin{Desc}
\item[Template Parameters:]
\begin{description}
\item[{\em genType}]Floating-point vector types.\end{description}
\end{Desc}
\begin{Desc}
\item[See also:]\href{http://www.opengl.org/sdk/docs/manglsl/xhtml/faceforward.xml}{\tt GLSL faceforward man page} 

\href{http://www.opengl.org/registry/doc/GLSLangSpec.4.20.8.pdf}{\tt GLSL 4.20.8 specification, section 8.5 Geometric Functions} \end{Desc}


Definition at line 282 of file func\_\-geometric.inl.

References glm::dot().

\begin{Code}\begin{verbatim}287         {
288                 return dot(Nref, I) < 0 ? N : -N;
289         }
\end{verbatim}
\end{Code}




Here is the call graph for this function:\hypertarget{group__core__func__geometric_gd73a94d9c967e619e670156356e93b7e}{
\index{core\_\-func\_\-geometric@{core\_\-func\_\-geometric}!length@{length}}
\index{length@{length}!core_func_geometric@{core\_\-func\_\-geometric}}
\subsubsection[length]{\setlength{\rightskip}{0pt plus 5cm}template$<$typename genType$>$ GLM\_\-FUNC\_\-QUALIFIER genType glm::length (genType const \& {\em x})\hspace{0.3cm}{\tt  \mbox{[}inline\mbox{]}}}}
\label{group__core__func__geometric_gd73a94d9c967e619e670156356e93b7e}


Returns the length of x, i.e., sqrt(x $\ast$ x).

\begin{Desc}
\item[Template Parameters:]
\begin{description}
\item[{\em genType}]Floating-point vector types.\end{description}
\end{Desc}
\begin{Desc}
\item[See also:]\href{http://www.opengl.org/sdk/docs/manglsl/xhtml/length.xml}{\tt GLSL length man page} 

\href{http://www.opengl.org/registry/doc/GLSLangSpec.4.20.8.pdf}{\tt GLSL 4.20.8 specification, section 8.5 Geometric Functions} \end{Desc}


Definition at line 89 of file func\_\-geometric.inl.

References glm::abs().

Referenced by glm::areOrthogonal(), glm::ballRand(), glm::diskRand(), glm::distance(), glm::exp(), glm::isNormalized(), glm::isNull(), glm::l2Norm(), glm::log(), glm::normalize(), glm::polar(), glm::rotate(), and glm::row().

\begin{Code}\begin{verbatim}92         {
93                 GLM_STATIC_ASSERT(std::numeric_limits<genType>::is_iec559, "'length' only accept floating-point inputs");
94 
95                 return abs(x);
96         }
\end{verbatim}
\end{Code}




Here is the call graph for this function:

Here is the caller graph for this function:\hypertarget{group__core__func__geometric_g0feb2bb89ee2743677ad2cb84544bd83}{
\index{core\_\-func\_\-geometric@{core\_\-func\_\-geometric}!normalize@{normalize}}
\index{normalize@{normalize}!core_func_geometric@{core\_\-func\_\-geometric}}
\subsubsection[normalize]{\setlength{\rightskip}{0pt plus 5cm}template$<$typename genType$>$ GLM\_\-FUNC\_\-QUALIFIER genType glm::normalize (genType const \& {\em x})\hspace{0.3cm}{\tt  \mbox{[}inline\mbox{]}}}}
\label{group__core__func__geometric_g0feb2bb89ee2743677ad2cb84544bd83}


Returns a vector in the same direction as x but with length of 1.

\begin{Desc}
\item[See also:]\href{http://www.opengl.org/sdk/docs/manglsl/xhtml/normalize.xml}{\tt GLSL normalize man page} 

\href{http://www.opengl.org/registry/doc/GLSLangSpec.4.20.8.pdf}{\tt GLSL 4.20.8 specification, section 8.5 Geometric Functions} \end{Desc}


Definition at line 233 of file func\_\-geometric.inl.

Referenced by glm::axisAngleMatrix(), glm::fastMix(), glm::intersectLineSphere(), glm::lookAt(), glm::orthonormalize(), glm::rotate(), glm::rotation(), and glm::triangleNormal().

\begin{Code}\begin{verbatim}236         {
237                 GLM_STATIC_ASSERT(std::numeric_limits<genType>::is_iec559, "'normalize' only accept floating-point inputs");
238 
239                 return x < genType(0) ? genType(-1) : genType(1);
240         }
\end{verbatim}
\end{Code}




Here is the caller graph for this function:\hypertarget{group__core__func__geometric_gc973ce2bc49f749a469d3ed2e2ac5a54}{
\index{core\_\-func\_\-geometric@{core\_\-func\_\-geometric}!reflect@{reflect}}
\index{reflect@{reflect}!core_func_geometric@{core\_\-func\_\-geometric}}
\subsubsection[reflect]{\setlength{\rightskip}{0pt plus 5cm}template$<$typename genType$>$ GLM\_\-FUNC\_\-QUALIFIER genType glm::reflect (genType const \& {\em I}, \/  genType const \& {\em N})\hspace{0.3cm}{\tt  \mbox{[}inline\mbox{]}}}}
\label{group__core__func__geometric_gc973ce2bc49f749a469d3ed2e2ac5a54}


For the incident vector I and surface orientation N, returns the reflection direction : result = I - 2.0 $\ast$ dot(N, I) $\ast$ N.

\begin{Desc}
\item[Template Parameters:]
\begin{description}
\item[{\em genType}]Floating-point vector types.\end{description}
\end{Desc}
\begin{Desc}
\item[See also:]\href{http://www.opengl.org/sdk/docs/manglsl/xhtml/reflect.xml}{\tt GLSL reflect man page} 

\href{http://www.opengl.org/registry/doc/GLSLangSpec.4.20.8.pdf}{\tt GLSL 4.20.8 specification, section 8.5 Geometric Functions} \end{Desc}


Definition at line 294 of file func\_\-geometric.inl.

References glm::dot().

\begin{Code}\begin{verbatim}298         {
299                 return I - N * dot(N, I) * genType(2);
300         }
\end{verbatim}
\end{Code}




Here is the call graph for this function:\hypertarget{group__core__func__geometric_g2020e01c134ebe03c1690306ff93af53}{
\index{core\_\-func\_\-geometric@{core\_\-func\_\-geometric}!refract@{refract}}
\index{refract@{refract}!core_func_geometric@{core\_\-func\_\-geometric}}
\subsubsection[refract]{\setlength{\rightskip}{0pt plus 5cm}template$<$typename T, precision P, template$<$ typename, precision $>$ class vecType$>$ GLM\_\-FUNC\_\-QUALIFIER vecType$<$ T, P $>$ glm::refract (vecType$<$ T, P $>$ const \& {\em I}, \/  vecType$<$ T, P $>$ const \& {\em N}, \/  T const \& {\em eta})\hspace{0.3cm}{\tt  \mbox{[}inline\mbox{]}}}}
\label{group__core__func__geometric_g2020e01c134ebe03c1690306ff93af53}


For the incident vector I and surface normal N, and the ratio of indices of refraction eta, return the refraction vector.

\begin{Desc}
\item[Template Parameters:]
\begin{description}
\item[{\em genType}]Floating-point vector types.\end{description}
\end{Desc}
\begin{Desc}
\item[See also:]\href{http://www.opengl.org/sdk/docs/manglsl/xhtml/refract.xml}{\tt GLSL refract man page} 

\href{http://www.opengl.org/registry/doc/GLSLangSpec.4.20.8.pdf}{\tt GLSL 4.20.8 specification, section 8.5 Geometric Functions} \end{Desc}


Definition at line 323 of file func\_\-geometric.inl.

References glm::dot(), and glm::sqrt().

\begin{Code}\begin{verbatim}328         {
329                 GLM_STATIC_ASSERT(std::numeric_limits<T>::is_iec559, "'refract' only accept floating-point inputs");
330 
331                 T dotValue = dot(N, I);
332                 T k = T(1) - eta * eta * (T(1) - dotValue * dotValue);
333                 if(k < T(0))
334                         return vecType<T, P>(0);
335                 else
336                         return eta * I - (eta * dotValue + std::sqrt(k)) * N;
337         }
\end{verbatim}
\end{Code}




Here is the call graph for this function: