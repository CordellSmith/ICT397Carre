\hypertarget{index_intro_sec}{}\section{Introduction}\label{index_intro_sec}
Bullet is a Collision Detection and Rigid Body Dynamics Library. The Library is Open Source and free for commercial use, under the ZLib license ( \href{http://opensource.org/licenses/zlib-license.php}{\tt http://opensource.org/licenses/zlib-license.php} ).

The main documentation is Bullet\_\-User\_\-Manual.pdf, included in the source code distribution. There is the Physics Forum for feedback and general Collision Detection and Physics discussions. Please visit \href{http://www.bulletphysics.org}{\tt http://www.bulletphysics.org}\hypertarget{index_install_sec}{}\section{Installation}\label{index_install_sec}
\hypertarget{index_step1}{}\subsection{Step 1: Download}\label{index_step1}
You can download the Bullet Physics Library from the github repository: \href{https://github.com/bulletphysics/bullet3/releases}{\tt https://github.com/bulletphysics/bullet3/releases}\hypertarget{index_step2}{}\subsection{Step 2: Building}\label{index_step2}
Bullet has multiple build systems, including premake, cmake and autotools. Premake and cmake support all platforms. Premake is included in the Bullet/build folder for Windows, Mac OSX and Linux. Under Windows you can click on Bullet/build/vs2010.bat to create Microsoft Visual Studio projects. On Mac OSX and Linux you can open a terminal and generate Makefile, codeblocks or Xcode4 projects: cd Bullet/build ./premake4\_\-osx gmake or ./premake4\_\-linux gmake or ./premake4\_\-linux64 gmake or (for Mac) ./premake4\_\-osx xcode4 cd Bullet/build/gmake make

An alternative to premake is cmake. You can download cmake from \href{http://www.cmake.org}{\tt http://www.cmake.org} cmake can autogenerate projectfiles for Microsoft Visual Studio, Apple Xcode, KDevelop and Unix Makefiles. The easiest is to run the CMake cmake-gui graphical user interface and choose the options and generate projectfiles. You can also use cmake in the command-line. Here are some examples for various platforms: cmake . -G \char`\"{}Visual Studio 9 2008\char`\"{} cmake . -G Xcode cmake . -G \char`\"{}Unix Makefiles\char`\"{} Although cmake is recommended, you can also use autotools for UNIX: ./autogen.sh ./configure to create a Makefile and then run make.\hypertarget{index_step3}{}\subsection{Step 3: Testing demos}\label{index_step3}
Try to run and experiment with BasicDemo executable as a starting point. Bullet can be used in several ways, as Full Rigid Body simulation, as Collision Detector Library or Low Level / Snippets like the GJK Closest Point calculation. The Dependencies can be seen in this documentation under Directories\hypertarget{index_step4}{}\subsection{Step 4: Integrating in your application, full Rigid Body and Soft Body simulation}\label{index_step4}
Check out BasicDemo how to create a \hyperlink{classbt_dynamics_world}{btDynamicsWorld}, \hyperlink{classbt_rigid_body}{btRigidBody} and btCollisionShape, Stepping the simulation and synchronizing your graphics object transform. Check out SoftDemo how to use soft body dynamics, using btSoftRigidDynamicsWorld. \hypertarget{index_step5}{}\subsection{Step 5 : Integrate the Collision Detection Library (without Dynamics and other Extras)}\label{index_step5}
Bullet Collision Detection can also be used without the Dynamics/Extras. Check out \hyperlink{classbt_collision_world}{btCollisionWorld} and btCollisionObject, and the CollisionInterfaceDemo. \hypertarget{index_step6}{}\subsection{Step 6 : Use Snippets like the GJK Closest Point calculation.}\label{index_step6}
Bullet has been designed in a modular way keeping dependencies to a minimum. The ConvexHullDistance demo demonstrates direct use of \hyperlink{classbt_gjk_pair_detector}{btGjkPairDetector}.\hypertarget{index_copyright}{}\section{Copyright}\label{index_copyright}
For up-to-data information and copyright and contributors list check out the Bullet\_\-User\_\-Manual.pdf 