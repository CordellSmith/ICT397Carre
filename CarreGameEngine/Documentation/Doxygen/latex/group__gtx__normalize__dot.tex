\hypertarget{group__gtx__normalize__dot}{
\section{GLM\_\-GTX\_\-normalize\_\-dot}
\label{group__gtx__normalize__dot}\index{GLM\_\-GTX\_\-normalize\_\-dot@{GLM\_\-GTX\_\-normalize\_\-dot}}
}


Collaboration diagram for GLM\_\-GTX\_\-normalize\_\-dot:Dot product of vectors that need to be normalize with a single square root.  
\subsection*{Functions}
\begin{CompactItemize}
\item 
{\footnotesize template$<$typename genType$>$ }\\GLM\_\-FUNC\_\-DECL genType::value\_\-type \hyperlink{group__gtx__normalize__dot_g068b0c92713a438533628dd5d0b581d4}{glm::normalizeDot} (genType const \&x, genType const \&y)
\item 
{\footnotesize template$<$typename genType$>$ }\\GLM\_\-FUNC\_\-DECL genType::value\_\-type \hyperlink{group__gtx__normalize__dot_gb3967681366216d96699232dd5e86d31}{glm::fastNormalizeDot} (genType const \&x, genType const \&y)
\end{CompactItemize}


\subsection{Detailed Description}
Dot product of vectors that need to be normalize with a single square root. 

$<$glm/gtx/normalized\_\-dot.hpp$>$ need to be included to use these functionalities. 

\subsection{Function Documentation}
\hypertarget{group__gtx__normalize__dot_gb3967681366216d96699232dd5e86d31}{
\index{gtx\_\-normalize\_\-dot@{gtx\_\-normalize\_\-dot}!fastNormalizeDot@{fastNormalizeDot}}
\index{fastNormalizeDot@{fastNormalizeDot}!gtx_normalize_dot@{gtx\_\-normalize\_\-dot}}
\subsubsection[fastNormalizeDot]{\setlength{\rightskip}{0pt plus 5cm}template$<$typename genType$>$ GLM\_\-FUNC\_\-QUALIFIER genType glm::fastNormalizeDot (genType const \& {\em x}, \/  genType const \& {\em y})\hspace{0.3cm}{\tt  \mbox{[}inline\mbox{]}}}}
\label{group__gtx__normalize__dot_gb3967681366216d96699232dd5e86d31}


Normalize parameters and returns the dot product of x and y. Faster that dot(fastNormalize(x), fastNormalize(y)). From GLM\_\-GTX\_\-normalize\_\-dot extension. 

Definition at line 66 of file normalize\_\-dot.inl.

References glm::dot(), and glm::fastInverseSqrt().

\begin{Code}\begin{verbatim}70         {
71                 return 
72                         glm::dot(x, y) * 
73                         fastInverseSqrt(glm::dot(x, x) * 
74                         glm::dot(y, y));
75         }
\end{verbatim}
\end{Code}




Here is the call graph for this function:\hypertarget{group__gtx__normalize__dot_g068b0c92713a438533628dd5d0b581d4}{
\index{gtx\_\-normalize\_\-dot@{gtx\_\-normalize\_\-dot}!normalizeDot@{normalizeDot}}
\index{normalizeDot@{normalizeDot}!gtx_normalize_dot@{gtx\_\-normalize\_\-dot}}
\subsubsection[normalizeDot]{\setlength{\rightskip}{0pt plus 5cm}template$<$typename genType$>$ GLM\_\-FUNC\_\-QUALIFIER genType glm::normalizeDot (genType const \& {\em x}, \/  genType const \& {\em y})\hspace{0.3cm}{\tt  \mbox{[}inline\mbox{]}}}}
\label{group__gtx__normalize__dot_g068b0c92713a438533628dd5d0b581d4}


Normalize parameters and returns the dot product of x and y. It's faster that dot(normalize(x), normalize(y)). From GLM\_\-GTX\_\-normalize\_\-dot extension. 

Definition at line 14 of file normalize\_\-dot.inl.

References glm::dot(), and glm::inversesqrt().

\begin{Code}\begin{verbatim}18         {
19                 return 
20                         glm::dot(x, y) * 
21                         glm::inversesqrt(glm::dot(x, x) * 
22                         glm::dot(y, y));
23         }
\end{verbatim}
\end{Code}




Here is the call graph for this function: