\hypertarget{group__gtc__reciprocal}{
\section{GLM\_\-GTC\_\-reciprocal}
\label{group__gtc__reciprocal}\index{GLM\_\-GTC\_\-reciprocal@{GLM\_\-GTC\_\-reciprocal}}
}


Collaboration diagram for GLM\_\-GTC\_\-reciprocal:Define secant, cosecant and cotangent functions.  
\subsection*{Functions}
\begin{CompactItemize}
\item 
{\footnotesize template$<$typename genType$>$ }\\GLM\_\-FUNC\_\-DECL genType \hyperlink{group__gtc__reciprocal_gef67dab7093a4d0ccc9c06ca05ddafd4}{glm::sec} (genType const \&angle)
\item 
{\footnotesize template$<$typename genType$>$ }\\GLM\_\-FUNC\_\-DECL genType \hyperlink{group__gtc__reciprocal_gdadd7293102fe18951a4acb4df1455a8}{glm::csc} (genType const \&angle)
\item 
{\footnotesize template$<$typename genType$>$ }\\GLM\_\-FUNC\_\-DECL genType \hyperlink{group__gtc__reciprocal_g8d3b73a481ad1115ca93da1410868e10}{glm::cot} (genType const \&angle)
\item 
{\footnotesize template$<$typename genType$>$ }\\GLM\_\-FUNC\_\-DECL genType \hyperlink{group__gtc__reciprocal_g79e2b1e9949a6c514eef05d462ab01a0}{glm::asec} (genType const \&x)
\item 
{\footnotesize template$<$typename genType$>$ }\\GLM\_\-FUNC\_\-DECL genType \hyperlink{group__gtc__reciprocal_g921be8522fbf0a5cfc877e78fb9abed2}{glm::acsc} (genType const \&x)
\item 
{\footnotesize template$<$typename genType$>$ }\\GLM\_\-FUNC\_\-DECL genType \hyperlink{group__gtc__reciprocal_ge06055493cfcf5e3732cf330d81fd186}{glm::acot} (genType const \&x)
\item 
{\footnotesize template$<$typename genType$>$ }\\GLM\_\-FUNC\_\-DECL genType \hyperlink{group__gtc__reciprocal_g6193b8e823dea616d6badca8076da066}{glm::sech} (genType const \&angle)
\item 
{\footnotesize template$<$typename genType$>$ }\\GLM\_\-FUNC\_\-DECL genType \hyperlink{group__gtc__reciprocal_g094e1e421d5d9320d4364deb82adb428}{glm::csch} (genType const \&angle)
\item 
{\footnotesize template$<$typename genType$>$ }\\GLM\_\-FUNC\_\-DECL genType \hyperlink{group__gtc__reciprocal_g8bc5d51e10b478b061a071eb91258d35}{glm::coth} (genType const \&angle)
\item 
{\footnotesize template$<$typename genType$>$ }\\GLM\_\-FUNC\_\-DECL genType \hyperlink{group__gtc__reciprocal_g05d1bc30693d02a9a08c9044d75c5333}{glm::asech} (genType const \&x)
\item 
{\footnotesize template$<$typename genType$>$ }\\GLM\_\-FUNC\_\-DECL genType \hyperlink{group__gtc__reciprocal_gb24c5c23d9d3d10517ad80f5af515b0e}{glm::acsch} (genType const \&x)
\item 
{\footnotesize template$<$typename genType$>$ }\\GLM\_\-FUNC\_\-DECL genType \hyperlink{group__gtc__reciprocal_g651e435e3e8f63b1ea1da0e5e7581864}{glm::acoth} (genType const \&x)
\end{CompactItemize}


\subsection{Detailed Description}
Define secant, cosecant and cotangent functions. 

$<$glm/gtc/reciprocal.hpp$>$ need to be included to use these features. 

\subsection{Function Documentation}
\hypertarget{group__gtc__reciprocal_ge06055493cfcf5e3732cf330d81fd186}{
\index{gtc\_\-reciprocal@{gtc\_\-reciprocal}!acot@{acot}}
\index{acot@{acot}!gtc_reciprocal@{gtc\_\-reciprocal}}
\subsubsection[acot]{\setlength{\rightskip}{0pt plus 5cm}template$<$typename genType$>$ GLM\_\-FUNC\_\-QUALIFIER genType glm::acot (genType const \& {\em x})\hspace{0.3cm}{\tt  \mbox{[}inline\mbox{]}}}}
\label{group__gtc__reciprocal_ge06055493cfcf5e3732cf330d81fd186}


Inverse cotangent function.

\begin{Desc}
\item[See also:]\hyperlink{group__gtc__reciprocal}{GLM\_\-GTC\_\-reciprocal} \end{Desc}


Definition at line 107 of file reciprocal.inl.

References glm::atan().

\begin{Code}\begin{verbatim}110         {
111                 GLM_STATIC_ASSERT(std::numeric_limits<genType>::is_iec559, "'acot' only accept floating-point values");
112 
113                 genType const pi_over_2 = genType(3.1415926535897932384626433832795 / 2.0);
114                 return pi_over_2 - atan(x);
115         }
\end{verbatim}
\end{Code}




Here is the call graph for this function:\hypertarget{group__gtc__reciprocal_g651e435e3e8f63b1ea1da0e5e7581864}{
\index{gtc\_\-reciprocal@{gtc\_\-reciprocal}!acoth@{acoth}}
\index{acoth@{acoth}!gtc_reciprocal@{gtc\_\-reciprocal}}
\subsubsection[acoth]{\setlength{\rightskip}{0pt plus 5cm}template$<$typename genType$>$ GLM\_\-FUNC\_\-QUALIFIER genType glm::acoth (genType const \& {\em x})\hspace{0.3cm}{\tt  \mbox{[}inline\mbox{]}}}}
\label{group__gtc__reciprocal_g651e435e3e8f63b1ea1da0e5e7581864}


Inverse cotangent hyperbolic function.

\begin{Desc}
\item[See also:]\hyperlink{group__gtc__reciprocal}{GLM\_\-GTC\_\-reciprocal} \end{Desc}


Definition at line 192 of file reciprocal.inl.

References glm::atanh().

\begin{Code}\begin{verbatim}195         {
196                 GLM_STATIC_ASSERT(std::numeric_limits<genType>::is_iec559, "'acoth' only accept floating-point values");
197 
198                 return atanh(genType(1) / x);
199         }
\end{verbatim}
\end{Code}




Here is the call graph for this function:\hypertarget{group__gtc__reciprocal_g921be8522fbf0a5cfc877e78fb9abed2}{
\index{gtc\_\-reciprocal@{gtc\_\-reciprocal}!acsc@{acsc}}
\index{acsc@{acsc}!gtc_reciprocal@{gtc\_\-reciprocal}}
\subsubsection[acsc]{\setlength{\rightskip}{0pt plus 5cm}template$<$typename genType$>$ GLM\_\-FUNC\_\-QUALIFIER genType glm::acsc (genType const \& {\em x})\hspace{0.3cm}{\tt  \mbox{[}inline\mbox{]}}}}
\label{group__gtc__reciprocal_g921be8522fbf0a5cfc877e78fb9abed2}


Inverse cosecant function.

\begin{Desc}
\item[See also:]\hyperlink{group__gtc__reciprocal}{GLM\_\-GTC\_\-reciprocal} \end{Desc}


Definition at line 93 of file reciprocal.inl.

References glm::asin().

\begin{Code}\begin{verbatim}96         {
97                 GLM_STATIC_ASSERT(std::numeric_limits<genType>::is_iec559, "'acsc' only accept floating-point values");
98 
99                 return asin(genType(1) / x);
100         }
\end{verbatim}
\end{Code}




Here is the call graph for this function:\hypertarget{group__gtc__reciprocal_gb24c5c23d9d3d10517ad80f5af515b0e}{
\index{gtc\_\-reciprocal@{gtc\_\-reciprocal}!acsch@{acsch}}
\index{acsch@{acsch}!gtc_reciprocal@{gtc\_\-reciprocal}}
\subsubsection[acsch]{\setlength{\rightskip}{0pt plus 5cm}template$<$typename genType$>$ GLM\_\-FUNC\_\-QUALIFIER genType glm::acsch (genType const \& {\em x})\hspace{0.3cm}{\tt  \mbox{[}inline\mbox{]}}}}
\label{group__gtc__reciprocal_gb24c5c23d9d3d10517ad80f5af515b0e}


Inverse cosecant hyperbolic function.

\begin{Desc}
\item[See also:]\hyperlink{group__gtc__reciprocal}{GLM\_\-GTC\_\-reciprocal} \end{Desc}


Definition at line 178 of file reciprocal.inl.

References glm::asinh().

\begin{Code}\begin{verbatim}181         {
182                 GLM_STATIC_ASSERT(std::numeric_limits<genType>::is_iec559, "'acsch' only accept floating-point values");
183 
184                 return asinh(genType(1) / x);
185         }
\end{verbatim}
\end{Code}




Here is the call graph for this function:\hypertarget{group__gtc__reciprocal_g79e2b1e9949a6c514eef05d462ab01a0}{
\index{gtc\_\-reciprocal@{gtc\_\-reciprocal}!asec@{asec}}
\index{asec@{asec}!gtc_reciprocal@{gtc\_\-reciprocal}}
\subsubsection[asec]{\setlength{\rightskip}{0pt plus 5cm}template$<$typename genType$>$ GLM\_\-FUNC\_\-QUALIFIER genType glm::asec (genType const \& {\em x})\hspace{0.3cm}{\tt  \mbox{[}inline\mbox{]}}}}
\label{group__gtc__reciprocal_g79e2b1e9949a6c514eef05d462ab01a0}


Inverse secant function.

\begin{Desc}
\item[See also:]\hyperlink{group__gtc__reciprocal}{GLM\_\-GTC\_\-reciprocal} \end{Desc}


Definition at line 79 of file reciprocal.inl.

References glm::acos().

\begin{Code}\begin{verbatim}82         {
83                 GLM_STATIC_ASSERT(std::numeric_limits<genType>::is_iec559, "'asec' only accept floating-point values");
84         
85                 return acos(genType(1) / x);
86         }
\end{verbatim}
\end{Code}




Here is the call graph for this function:\hypertarget{group__gtc__reciprocal_g05d1bc30693d02a9a08c9044d75c5333}{
\index{gtc\_\-reciprocal@{gtc\_\-reciprocal}!asech@{asech}}
\index{asech@{asech}!gtc_reciprocal@{gtc\_\-reciprocal}}
\subsubsection[asech]{\setlength{\rightskip}{0pt plus 5cm}template$<$typename genType$>$ GLM\_\-FUNC\_\-QUALIFIER genType glm::asech (genType const \& {\em x})\hspace{0.3cm}{\tt  \mbox{[}inline\mbox{]}}}}
\label{group__gtc__reciprocal_g05d1bc30693d02a9a08c9044d75c5333}


Inverse secant hyperbolic function.

\begin{Desc}
\item[See also:]\hyperlink{group__gtc__reciprocal}{GLM\_\-GTC\_\-reciprocal} \end{Desc}


Definition at line 164 of file reciprocal.inl.

References glm::acosh().

\begin{Code}\begin{verbatim}167         {
168                 GLM_STATIC_ASSERT(std::numeric_limits<genType>::is_iec559, "'asech' only accept floating-point values");
169 
170                 return acosh(genType(1) / x);
171         }
\end{verbatim}
\end{Code}




Here is the call graph for this function:\hypertarget{group__gtc__reciprocal_g8d3b73a481ad1115ca93da1410868e10}{
\index{gtc\_\-reciprocal@{gtc\_\-reciprocal}!cot@{cot}}
\index{cot@{cot}!gtc_reciprocal@{gtc\_\-reciprocal}}
\subsubsection[cot]{\setlength{\rightskip}{0pt plus 5cm}template$<$typename genType$>$ GLM\_\-FUNC\_\-QUALIFIER genType glm::cot (genType const \& {\em angle})\hspace{0.3cm}{\tt  \mbox{[}inline\mbox{]}}}}
\label{group__gtc__reciprocal_g8d3b73a481ad1115ca93da1410868e10}


Cotangent function. adjacent / opposite or 1 / tan(x)

\begin{Desc}
\item[See also:]\hyperlink{group__gtc__reciprocal}{GLM\_\-GTC\_\-reciprocal} \end{Desc}


Definition at line 65 of file reciprocal.inl.

References glm::tan().

\begin{Code}\begin{verbatim}68         {
69                 GLM_STATIC_ASSERT(std::numeric_limits<genType>::is_iec559, "'cot' only accept floating-point values");
70 
71                 return genType(1) / glm::tan(angle);
72         }
\end{verbatim}
\end{Code}




Here is the call graph for this function:\hypertarget{group__gtc__reciprocal_g8bc5d51e10b478b061a071eb91258d35}{
\index{gtc\_\-reciprocal@{gtc\_\-reciprocal}!coth@{coth}}
\index{coth@{coth}!gtc_reciprocal@{gtc\_\-reciprocal}}
\subsubsection[coth]{\setlength{\rightskip}{0pt plus 5cm}template$<$typename genType$>$ GLM\_\-FUNC\_\-QUALIFIER genType glm::coth (genType const \& {\em angle})\hspace{0.3cm}{\tt  \mbox{[}inline\mbox{]}}}}
\label{group__gtc__reciprocal_g8bc5d51e10b478b061a071eb91258d35}


Cotangent hyperbolic function.

\begin{Desc}
\item[See also:]\hyperlink{group__gtc__reciprocal}{GLM\_\-GTC\_\-reciprocal} \end{Desc}


Definition at line 150 of file reciprocal.inl.

References glm::cosh(), and glm::sinh().

\begin{Code}\begin{verbatim}153         {
154                 GLM_STATIC_ASSERT(std::numeric_limits<genType>::is_iec559, "'coth' only accept floating-point values");
155 
156                 return glm::cosh(angle) / glm::sinh(angle);
157         }
\end{verbatim}
\end{Code}




Here is the call graph for this function:\hypertarget{group__gtc__reciprocal_gdadd7293102fe18951a4acb4df1455a8}{
\index{gtc\_\-reciprocal@{gtc\_\-reciprocal}!csc@{csc}}
\index{csc@{csc}!gtc_reciprocal@{gtc\_\-reciprocal}}
\subsubsection[csc]{\setlength{\rightskip}{0pt plus 5cm}template$<$typename genType$>$ GLM\_\-FUNC\_\-QUALIFIER genType glm::csc (genType const \& {\em angle})\hspace{0.3cm}{\tt  \mbox{[}inline\mbox{]}}}}
\label{group__gtc__reciprocal_gdadd7293102fe18951a4acb4df1455a8}


Cosecant function. hypotenuse / opposite or 1 / sin(x)

\begin{Desc}
\item[See also:]\hyperlink{group__gtc__reciprocal}{GLM\_\-GTC\_\-reciprocal} \end{Desc}


Definition at line 51 of file reciprocal.inl.

References glm::sin().

\begin{Code}\begin{verbatim}54         {
55                 GLM_STATIC_ASSERT(std::numeric_limits<genType>::is_iec559, "'csc' only accept floating-point values");
56 
57                 return genType(1) / glm::sin(angle);
58         }
\end{verbatim}
\end{Code}




Here is the call graph for this function:\hypertarget{group__gtc__reciprocal_g094e1e421d5d9320d4364deb82adb428}{
\index{gtc\_\-reciprocal@{gtc\_\-reciprocal}!csch@{csch}}
\index{csch@{csch}!gtc_reciprocal@{gtc\_\-reciprocal}}
\subsubsection[csch]{\setlength{\rightskip}{0pt plus 5cm}template$<$typename genType$>$ GLM\_\-FUNC\_\-QUALIFIER genType glm::csch (genType const \& {\em angle})\hspace{0.3cm}{\tt  \mbox{[}inline\mbox{]}}}}
\label{group__gtc__reciprocal_g094e1e421d5d9320d4364deb82adb428}


Cosecant hyperbolic function.

\begin{Desc}
\item[See also:]\hyperlink{group__gtc__reciprocal}{GLM\_\-GTC\_\-reciprocal} \end{Desc}


Definition at line 136 of file reciprocal.inl.

References glm::sinh().

\begin{Code}\begin{verbatim}139         {
140                 GLM_STATIC_ASSERT(std::numeric_limits<genType>::is_iec559, "'csch' only accept floating-point values");
141 
142                 return genType(1) / glm::sinh(angle);
143         }
\end{verbatim}
\end{Code}




Here is the call graph for this function:\hypertarget{group__gtc__reciprocal_gef67dab7093a4d0ccc9c06ca05ddafd4}{
\index{gtc\_\-reciprocal@{gtc\_\-reciprocal}!sec@{sec}}
\index{sec@{sec}!gtc_reciprocal@{gtc\_\-reciprocal}}
\subsubsection[sec]{\setlength{\rightskip}{0pt plus 5cm}template$<$typename genType$>$ GLM\_\-FUNC\_\-QUALIFIER genType glm::sec (genType const \& {\em angle})\hspace{0.3cm}{\tt  \mbox{[}inline\mbox{]}}}}
\label{group__gtc__reciprocal_gef67dab7093a4d0ccc9c06ca05ddafd4}


Secant function. hypotenuse / adjacent or 1 / cos(x)

\begin{Desc}
\item[See also:]\hyperlink{group__gtc__reciprocal}{GLM\_\-GTC\_\-reciprocal} \end{Desc}


Definition at line 37 of file reciprocal.inl.

References glm::cos().

\begin{Code}\begin{verbatim}40         {
41                 GLM_STATIC_ASSERT(std::numeric_limits<genType>::is_iec559, "'sec' only accept floating-point values");
42 
43                 return genType(1) / glm::cos(angle);
44         }
\end{verbatim}
\end{Code}




Here is the call graph for this function:\hypertarget{group__gtc__reciprocal_g6193b8e823dea616d6badca8076da066}{
\index{gtc\_\-reciprocal@{gtc\_\-reciprocal}!sech@{sech}}
\index{sech@{sech}!gtc_reciprocal@{gtc\_\-reciprocal}}
\subsubsection[sech]{\setlength{\rightskip}{0pt plus 5cm}template$<$typename genType$>$ GLM\_\-FUNC\_\-QUALIFIER genType glm::sech (genType const \& {\em angle})\hspace{0.3cm}{\tt  \mbox{[}inline\mbox{]}}}}
\label{group__gtc__reciprocal_g6193b8e823dea616d6badca8076da066}


Secant hyperbolic function.

\begin{Desc}
\item[See also:]\hyperlink{group__gtc__reciprocal}{GLM\_\-GTC\_\-reciprocal} \end{Desc}


Definition at line 122 of file reciprocal.inl.

References glm::cosh().

\begin{Code}\begin{verbatim}125         {
126                 GLM_STATIC_ASSERT(std::numeric_limits<genType>::is_iec559, "'sech' only accept floating-point values");
127 
128                 return genType(1) / glm::cosh(angle);
129         }
\end{verbatim}
\end{Code}




Here is the call graph for this function: