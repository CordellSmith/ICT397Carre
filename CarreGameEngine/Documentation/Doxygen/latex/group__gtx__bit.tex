\hypertarget{group__gtx__bit}{
\section{GLM\_\-GTX\_\-bit}
\label{group__gtx__bit}\index{GLM\_\-GTX\_\-bit@{GLM\_\-GTX\_\-bit}}
}


Collaboration diagram for GLM\_\-GTX\_\-bit:Allow to perform bit operations on integer values.  
\subsection*{Functions}
\begin{CompactItemize}
\item 
{\footnotesize template$<$typename genIType$>$ }\\GLM\_\-FUNC\_\-DECL genIType \hyperlink{group__gtx__bit_g7a909d8e8b8a5f73ed643bdca7602017}{glm::mask} (genIType const \&count)
\item 
{\footnotesize template$<$typename genType$>$ }\\GLM\_\-FUNC\_\-DECL genType \hyperlink{group__gtx__bit_gda4310fc2dd8db30392da133067ed13e}{glm::highestBitValue} (genType const \&value)
\item 
{\footnotesize template$<$typename genType$>$ }\\GLM\_\-FUNC\_\-DECL bool \hyperlink{group__gtx__bit_g2b12722968dabd423334391d1fd42acd}{glm::isPowerOfTwo} (genType const \&value)
\item 
{\footnotesize template$<$typename genType$>$ }\\GLM\_\-FUNC\_\-DECL genType \hyperlink{group__gtx__bit_gf27d271ec57b96b6acae9395b9c4a365}{glm::powerOfTwoAbove} (genType const \&value)
\item 
{\footnotesize template$<$typename genType$>$ }\\GLM\_\-FUNC\_\-DECL genType \hyperlink{group__gtx__bit_ga0bb1687b43f594a471c5506cc505dce}{glm::powerOfTwoBelow} (genType const \&value)
\item 
{\footnotesize template$<$typename genType$>$ }\\GLM\_\-FUNC\_\-DECL genType \hyperlink{group__gtx__bit_g0e3c8f921e59dc07ad9c70bb1376799c}{glm::powerOfTwoNearest} (genType const \&value)
\item 
{\footnotesize template$<$typename genType$>$ }\\GLM\_\-DEPRECATED GLM\_\-FUNC\_\-DECL genType \hyperlink{group__gtx__bit_g878bcf889f80259fcf86d0e25db92af4}{glm::bitRevert} (genType const \&value)
\item 
{\footnotesize template$<$typename genType$>$ }\\GLM\_\-FUNC\_\-DECL genType \hyperlink{group__gtx__bit_g01a6893d9fa57a4df1f6f7b0d399268d}{glm::bitRotateRight} (genType const \&In, std::size\_\-t Shift)
\item 
{\footnotesize template$<$typename genType$>$ }\\GLM\_\-FUNC\_\-DECL genType \hyperlink{group__gtx__bit_g4cd980832ab9c35c73f8c76b8f309c92}{glm::bitRotateLeft} (genType const \&In, std::size\_\-t Shift)
\item 
{\footnotesize template$<$typename genIUType$>$ }\\GLM\_\-FUNC\_\-DECL genIUType \hyperlink{group__gtx__bit_g98140d04738cfe05d70f090a7f1151f9}{glm::fillBitfieldWithOne} (genIUType const \&Value, int const \&FromBit, int const \&ToBit)
\item 
{\footnotesize template$<$typename genIUType$>$ }\\GLM\_\-FUNC\_\-DECL genIUType \hyperlink{group__gtx__bit_gd7bf903dae07525ab89be5a3cad616a7}{glm::fillBitfieldWithZero} (genIUType const \&Value, int const \&FromBit, int const \&ToBit)
\item 
GLM\_\-FUNC\_\-DECL int16 \hyperlink{group__gtx__bit_g479134317bc95d99f2b2e235d3db287b}{glm::bitfieldInterleave} (int8 x, int8 y)
\item 
GLM\_\-FUNC\_\-DECL uint16 \hyperlink{group__gtx__bit_g0700a3ceb088a0ecc23d76c154096061}{glm::bitfieldInterleave} (uint8 x, uint8 y)
\item 
GLM\_\-FUNC\_\-DECL int32 \hyperlink{group__gtx__bit_g1a0264598647ae00a596865af4e1e878}{glm::bitfieldInterleave} (int16 x, int16 y)
\item 
GLM\_\-FUNC\_\-DECL uint32 \hyperlink{group__gtx__bit_g19ef8360379483e3ee245e89cb62ff93}{glm::bitfieldInterleave} (uint16 x, uint16 y)
\item 
GLM\_\-FUNC\_\-DECL int64 \hyperlink{group__gtx__bit_g0de51d5985e6a703f305a5a61479babd}{glm::bitfieldInterleave} (int32 x, int32 y)
\item 
GLM\_\-FUNC\_\-DECL uint64 \hyperlink{group__gtx__bit_g2bc87fd66f6f8471c1a46888360cef12}{glm::bitfieldInterleave} (uint32 x, uint32 y)
\item 
GLM\_\-FUNC\_\-DECL int32 \hyperlink{group__gtx__bit_g6dee2ce1c45805063bb7fc5f6fd8f5ca}{glm::bitfieldInterleave} (int8 x, int8 y, int8 z)
\item 
GLM\_\-FUNC\_\-DECL uint32 \hyperlink{group__gtx__bit_gb9d593a2e916beb8f8137a0dbeae3afe}{glm::bitfieldInterleave} (uint8 x, uint8 y, uint8 z)
\item 
GLM\_\-FUNC\_\-DECL int64 \hyperlink{group__gtx__bit_gf898f842ac089fcc8d6201c32702584a}{glm::bitfieldInterleave} (int16 x, int16 y, int16 z)
\item 
GLM\_\-FUNC\_\-DECL uint64 \hyperlink{group__gtx__bit_g3c170e2ec54f2faab5e1c5bb693d718d}{glm::bitfieldInterleave} (uint16 x, uint16 y, uint16 z)
\item 
GLM\_\-FUNC\_\-DECL int64 \hyperlink{group__gtx__bit_g64e2d84f6560af3cc639644b1e628c42}{glm::bitfieldInterleave} (int32 x, int32 y, int32 z)
\item 
GLM\_\-FUNC\_\-DECL uint64 \hyperlink{group__gtx__bit_g7c10eb37f608365cfaef5ca2c476e1ce}{glm::bitfieldInterleave} (uint32 x, uint32 y, uint32 z)
\item 
GLM\_\-FUNC\_\-DECL int32 \hyperlink{group__gtx__bit_g7da84ecc2b3a46c9c08a9f40012359cf}{glm::bitfieldInterleave} (int8 x, int8 y, int8 z, int8 w)
\item 
GLM\_\-FUNC\_\-DECL uint32 \hyperlink{group__gtx__bit_g447c0bbed9d60c14578626d8f03f3755}{glm::bitfieldInterleave} (uint8 x, uint8 y, uint8 z, uint8 w)
\item 
GLM\_\-FUNC\_\-DECL int64 \hyperlink{group__gtx__bit_g09ee0be0fac790a1607a711e597dd9bf}{glm::bitfieldInterleave} (int16 x, int16 y, int16 z, int16 w)
\item 
GLM\_\-FUNC\_\-DECL uint64 \hyperlink{group__gtx__bit_gc8a926a7bfd9b23c22a4f685193fbfe1}{glm::bitfieldInterleave} (uint16 x, uint16 y, uint16 z, uint16 w)
\end{CompactItemize}


\subsection{Detailed Description}
Allow to perform bit operations on integer values. 

$<$glm/gtx/bit.hpp$>$ need to be included to use these functionalities. 

\subsection{Function Documentation}
\hypertarget{group__gtx__bit_gc8a926a7bfd9b23c22a4f685193fbfe1}{
\index{gtx\_\-bit@{gtx\_\-bit}!bitfieldInterleave@{bitfieldInterleave}}
\index{bitfieldInterleave@{bitfieldInterleave}!gtx_bit@{gtx\_\-bit}}
\subsubsection[bitfieldInterleave]{\setlength{\rightskip}{0pt plus 5cm}GLM\_\-FUNC\_\-QUALIFIER uint64 glm::bitfieldInterleave (uint16 {\em x}, \/  uint16 {\em y}, \/  uint16 {\em z}, \/  uint16 {\em w})\hspace{0.3cm}{\tt  \mbox{[}inline\mbox{]}}}}
\label{group__gtx__bit_gc8a926a7bfd9b23c22a4f685193fbfe1}


Interleaves the bits of x, y, z and w. The first bit is the first bit of x followed by the first bit of y, the first bit of z and finally the first bit of w. The other bits are interleaved following the previous sequence.

\begin{Desc}
\item[See also:]\hyperlink{group__gtx__bit}{GLM\_\-GTX\_\-bit} \end{Desc}


Definition at line 777 of file bit.inl.

\begin{Code}\begin{verbatim}778         {
779                 return detail::bitfieldInterleave<uint16, uint64>(x, y, z, w);
780         }
\end{verbatim}
\end{Code}


\hypertarget{group__gtx__bit_g09ee0be0fac790a1607a711e597dd9bf}{
\index{gtx\_\-bit@{gtx\_\-bit}!bitfieldInterleave@{bitfieldInterleave}}
\index{bitfieldInterleave@{bitfieldInterleave}!gtx_bit@{gtx\_\-bit}}
\subsubsection[bitfieldInterleave]{\setlength{\rightskip}{0pt plus 5cm}GLM\_\-FUNC\_\-QUALIFIER int64 glm::bitfieldInterleave (int16 {\em x}, \/  int16 {\em y}, \/  int16 {\em z}, \/  int16 {\em w})}}
\label{group__gtx__bit_g09ee0be0fac790a1607a711e597dd9bf}


Interleaves the bits of x, y, z and w. The first bit is the first bit of x followed by the first bit of y, the first bit of z and finally the first bit of w. The other bits are interleaved following the previous sequence.

\begin{Desc}
\item[See also:]\hyperlink{group__gtx__bit}{GLM\_\-GTX\_\-bit} \end{Desc}


Definition at line 754 of file bit.inl.

References glm::bitfieldInterleave().

\begin{Code}\begin{verbatim}755         {
756                 union sign16
757                 {
758                         int16 i;
759                         uint16 u;
760                 } sign_x, sign_y, sign_z, sign_w;
761 
762                 union sign64
763                 {
764                         int64 i;
765                         uint64 u;
766                 } result;
767 
768                 sign_x.i = x;
769                 sign_y.i = y;
770                 sign_z.i = z;
771                 sign_w.i = w;
772                 result.u = bitfieldInterleave(sign_x.u, sign_y.u, sign_z.u, sign_w.u);
773 
774                 return result.i;
775         }
\end{verbatim}
\end{Code}




Here is the call graph for this function:\hypertarget{group__gtx__bit_g447c0bbed9d60c14578626d8f03f3755}{
\index{gtx\_\-bit@{gtx\_\-bit}!bitfieldInterleave@{bitfieldInterleave}}
\index{bitfieldInterleave@{bitfieldInterleave}!gtx_bit@{gtx\_\-bit}}
\subsubsection[bitfieldInterleave]{\setlength{\rightskip}{0pt plus 5cm}GLM\_\-FUNC\_\-QUALIFIER uint32 glm::bitfieldInterleave (uint8 {\em x}, \/  uint8 {\em y}, \/  uint8 {\em z}, \/  uint8 {\em w})\hspace{0.3cm}{\tt  \mbox{[}inline\mbox{]}}}}
\label{group__gtx__bit_g447c0bbed9d60c14578626d8f03f3755}


Interleaves the bits of x, y, z and w. The first bit is the first bit of x followed by the first bit of y, the first bit of z and finally the first bit of w. The other bits are interleaved following the previous sequence.

\begin{Desc}
\item[See also:]\hyperlink{group__gtx__bit}{GLM\_\-GTX\_\-bit} \end{Desc}


Definition at line 749 of file bit.inl.

\begin{Code}\begin{verbatim}750         {
751                 return detail::bitfieldInterleave<uint8, uint32>(x, y, z, w);
752         }
\end{verbatim}
\end{Code}


\hypertarget{group__gtx__bit_g7da84ecc2b3a46c9c08a9f40012359cf}{
\index{gtx\_\-bit@{gtx\_\-bit}!bitfieldInterleave@{bitfieldInterleave}}
\index{bitfieldInterleave@{bitfieldInterleave}!gtx_bit@{gtx\_\-bit}}
\subsubsection[bitfieldInterleave]{\setlength{\rightskip}{0pt plus 5cm}GLM\_\-FUNC\_\-QUALIFIER int32 glm::bitfieldInterleave (int8 {\em x}, \/  int8 {\em y}, \/  int8 {\em z}, \/  int8 {\em w})}}
\label{group__gtx__bit_g7da84ecc2b3a46c9c08a9f40012359cf}


Interleaves the bits of x, y, z and w. The first bit is the first bit of x followed by the first bit of y, the first bit of z and finally the first bit of w. The other bits are interleaved following the previous sequence.

\begin{Desc}
\item[See also:]\hyperlink{group__gtx__bit}{GLM\_\-GTX\_\-bit} \end{Desc}


Definition at line 726 of file bit.inl.

References glm::bitfieldInterleave().

\begin{Code}\begin{verbatim}727         {
728                 union sign8
729                 {
730                         int8 i;
731                         uint8 u;
732                 } sign_x, sign_y, sign_z, sign_w;
733 
734                 union sign32
735                 {
736                         int32 i;
737                         uint32 u;
738                 } result;
739 
740                 sign_x.i = x;
741                 sign_y.i = y;
742                 sign_z.i = z;
743                 sign_w.i = w;
744                 result.u = bitfieldInterleave(sign_x.u, sign_y.u, sign_z.u, sign_w.u);
745 
746                 return result.i;
747         }
\end{verbatim}
\end{Code}




Here is the call graph for this function:\hypertarget{group__gtx__bit_g7c10eb37f608365cfaef5ca2c476e1ce}{
\index{gtx\_\-bit@{gtx\_\-bit}!bitfieldInterleave@{bitfieldInterleave}}
\index{bitfieldInterleave@{bitfieldInterleave}!gtx_bit@{gtx\_\-bit}}
\subsubsection[bitfieldInterleave]{\setlength{\rightskip}{0pt plus 5cm}GLM\_\-FUNC\_\-QUALIFIER uint64 glm::bitfieldInterleave (uint32 {\em x}, \/  uint32 {\em y}, \/  uint32 {\em z})\hspace{0.3cm}{\tt  \mbox{[}inline\mbox{]}}}}
\label{group__gtx__bit_g7c10eb37f608365cfaef5ca2c476e1ce}


Interleaves the bits of x, y and z. The first bit is the first bit of x followed by the first bit of y and the first bit of z. The other bits are interleaved following the previous sequence.

\begin{Desc}
\item[See also:]\hyperlink{group__gtx__bit}{GLM\_\-GTX\_\-bit} \end{Desc}


Definition at line 721 of file bit.inl.

\begin{Code}\begin{verbatim}722         {
723                 return detail::bitfieldInterleave<uint32, uint64>(x, y, z);
724         }
\end{verbatim}
\end{Code}


\hypertarget{group__gtx__bit_g64e2d84f6560af3cc639644b1e628c42}{
\index{gtx\_\-bit@{gtx\_\-bit}!bitfieldInterleave@{bitfieldInterleave}}
\index{bitfieldInterleave@{bitfieldInterleave}!gtx_bit@{gtx\_\-bit}}
\subsubsection[bitfieldInterleave]{\setlength{\rightskip}{0pt plus 5cm}GLM\_\-FUNC\_\-QUALIFIER int64 glm::bitfieldInterleave (int32 {\em x}, \/  int32 {\em y}, \/  int32 {\em z})}}
\label{group__gtx__bit_g64e2d84f6560af3cc639644b1e628c42}


Interleaves the bits of x, y and z. The first bit is the first bit of x followed by the first bit of y and the first bit of z. The other bits are interleaved following the previous sequence.

\begin{Desc}
\item[See also:]\hyperlink{group__gtx__bit}{GLM\_\-GTX\_\-bit} \end{Desc}


Definition at line 699 of file bit.inl.

References glm::bitfieldInterleave().

\begin{Code}\begin{verbatim}700         {
701                 union sign16
702                 {
703                         int32 i;
704                         uint32 u;
705                 } sign_x, sign_y, sign_z;
706 
707                 union sign64
708                 {
709                         int64 i;
710                         uint64 u;
711                 } result;
712 
713                 sign_x.i = x;
714                 sign_y.i = y;
715                 sign_z.i = z;
716                 result.u = bitfieldInterleave(sign_x.u, sign_y.u, sign_z.u);
717 
718                 return result.i;
719         }
\end{verbatim}
\end{Code}




Here is the call graph for this function:\hypertarget{group__gtx__bit_g3c170e2ec54f2faab5e1c5bb693d718d}{
\index{gtx\_\-bit@{gtx\_\-bit}!bitfieldInterleave@{bitfieldInterleave}}
\index{bitfieldInterleave@{bitfieldInterleave}!gtx_bit@{gtx\_\-bit}}
\subsubsection[bitfieldInterleave]{\setlength{\rightskip}{0pt plus 5cm}GLM\_\-FUNC\_\-QUALIFIER uint64 glm::bitfieldInterleave (uint16 {\em x}, \/  uint16 {\em y}, \/  uint16 {\em z})\hspace{0.3cm}{\tt  \mbox{[}inline\mbox{]}}}}
\label{group__gtx__bit_g3c170e2ec54f2faab5e1c5bb693d718d}


Interleaves the bits of x, y and z. The first bit is the first bit of x followed by the first bit of y and the first bit of z. The other bits are interleaved following the previous sequence.

\begin{Desc}
\item[See also:]\hyperlink{group__gtx__bit}{GLM\_\-GTX\_\-bit} \end{Desc}


Definition at line 694 of file bit.inl.

\begin{Code}\begin{verbatim}695         {
696                 return detail::bitfieldInterleave<uint32, uint64>(x, y, z);
697         }
\end{verbatim}
\end{Code}


\hypertarget{group__gtx__bit_gf898f842ac089fcc8d6201c32702584a}{
\index{gtx\_\-bit@{gtx\_\-bit}!bitfieldInterleave@{bitfieldInterleave}}
\index{bitfieldInterleave@{bitfieldInterleave}!gtx_bit@{gtx\_\-bit}}
\subsubsection[bitfieldInterleave]{\setlength{\rightskip}{0pt plus 5cm}GLM\_\-FUNC\_\-QUALIFIER int64 glm::bitfieldInterleave (int16 {\em x}, \/  int16 {\em y}, \/  int16 {\em z})}}
\label{group__gtx__bit_gf898f842ac089fcc8d6201c32702584a}


Interleaves the bits of x, y and z. The first bit is the first bit of x followed by the first bit of y and the first bit of z. The other bits are interleaved following the previous sequence.

\begin{Desc}
\item[See also:]\hyperlink{group__gtx__bit}{GLM\_\-GTX\_\-bit} \end{Desc}


Definition at line 672 of file bit.inl.

References glm::bitfieldInterleave().

\begin{Code}\begin{verbatim}673         {
674                 union sign16
675                 {
676                         int16 i;
677                         uint16 u;
678                 } sign_x, sign_y, sign_z;
679 
680                 union sign64
681                 {
682                         int64 i;
683                         uint64 u;
684                 } result;
685 
686                 sign_x.i = x;
687                 sign_y.i = y;
688                 sign_z.i = z;
689                 result.u = bitfieldInterleave(sign_x.u, sign_y.u, sign_z.u);
690 
691                 return result.i;
692         }
\end{verbatim}
\end{Code}




Here is the call graph for this function:\hypertarget{group__gtx__bit_gb9d593a2e916beb8f8137a0dbeae3afe}{
\index{gtx\_\-bit@{gtx\_\-bit}!bitfieldInterleave@{bitfieldInterleave}}
\index{bitfieldInterleave@{bitfieldInterleave}!gtx_bit@{gtx\_\-bit}}
\subsubsection[bitfieldInterleave]{\setlength{\rightskip}{0pt plus 5cm}GLM\_\-FUNC\_\-QUALIFIER uint32 glm::bitfieldInterleave (uint8 {\em x}, \/  uint8 {\em y}, \/  uint8 {\em z})\hspace{0.3cm}{\tt  \mbox{[}inline\mbox{]}}}}
\label{group__gtx__bit_gb9d593a2e916beb8f8137a0dbeae3afe}


Interleaves the bits of x, y and z. The first bit is the first bit of x followed by the first bit of y and the first bit of z. The other bits are interleaved following the previous sequence.

\begin{Desc}
\item[See also:]\hyperlink{group__gtx__bit}{GLM\_\-GTX\_\-bit} \end{Desc}


Definition at line 667 of file bit.inl.

\begin{Code}\begin{verbatim}668         {
669                 return detail::bitfieldInterleave<uint8, uint32>(x, y, z);
670         }
\end{verbatim}
\end{Code}


\hypertarget{group__gtx__bit_g6dee2ce1c45805063bb7fc5f6fd8f5ca}{
\index{gtx\_\-bit@{gtx\_\-bit}!bitfieldInterleave@{bitfieldInterleave}}
\index{bitfieldInterleave@{bitfieldInterleave}!gtx_bit@{gtx\_\-bit}}
\subsubsection[bitfieldInterleave]{\setlength{\rightskip}{0pt plus 5cm}GLM\_\-FUNC\_\-QUALIFIER int32 glm::bitfieldInterleave (int8 {\em x}, \/  int8 {\em y}, \/  int8 {\em z})}}
\label{group__gtx__bit_g6dee2ce1c45805063bb7fc5f6fd8f5ca}


Interleaves the bits of x, y and z. The first bit is the first bit of x followed by the first bit of y and the first bit of z. The other bits are interleaved following the previous sequence.

\begin{Desc}
\item[See also:]\hyperlink{group__gtx__bit}{GLM\_\-GTX\_\-bit} \end{Desc}


Definition at line 645 of file bit.inl.

References glm::bitfieldInterleave().

\begin{Code}\begin{verbatim}646         {
647                 union sign8
648                 {
649                         int8 i;
650                         uint8 u;
651                 } sign_x, sign_y, sign_z;
652 
653                 union sign32
654                 {
655                         int32 i;
656                         uint32 u;
657                 } result;
658 
659                 sign_x.i = x;
660                 sign_y.i = y;
661                 sign_z.i = z;
662                 result.u = bitfieldInterleave(sign_x.u, sign_y.u, sign_z.u);
663 
664                 return result.i;
665         }
\end{verbatim}
\end{Code}




Here is the call graph for this function:\hypertarget{group__gtx__bit_g2bc87fd66f6f8471c1a46888360cef12}{
\index{gtx\_\-bit@{gtx\_\-bit}!bitfieldInterleave@{bitfieldInterleave}}
\index{bitfieldInterleave@{bitfieldInterleave}!gtx_bit@{gtx\_\-bit}}
\subsubsection[bitfieldInterleave]{\setlength{\rightskip}{0pt plus 5cm}GLM\_\-FUNC\_\-QUALIFIER uint64 glm::bitfieldInterleave (uint32 {\em x}, \/  uint32 {\em y})\hspace{0.3cm}{\tt  \mbox{[}inline\mbox{]}}}}
\label{group__gtx__bit_g2bc87fd66f6f8471c1a46888360cef12}


Interleaves the bits of x and y. The first bit is the first bit of x followed by the first bit of y. The other bits are interleaved following the previous sequence.

\begin{Desc}
\item[See also:]\hyperlink{group__gtx__bit}{GLM\_\-GTX\_\-bit} \end{Desc}


Definition at line 640 of file bit.inl.

\begin{Code}\begin{verbatim}641         {
642                 return detail::bitfieldInterleave<uint32, uint64>(x, y);
643         }
\end{verbatim}
\end{Code}


\hypertarget{group__gtx__bit_g0de51d5985e6a703f305a5a61479babd}{
\index{gtx\_\-bit@{gtx\_\-bit}!bitfieldInterleave@{bitfieldInterleave}}
\index{bitfieldInterleave@{bitfieldInterleave}!gtx_bit@{gtx\_\-bit}}
\subsubsection[bitfieldInterleave]{\setlength{\rightskip}{0pt plus 5cm}GLM\_\-FUNC\_\-QUALIFIER int64 glm::bitfieldInterleave (int32 {\em x}, \/  int32 {\em y})}}
\label{group__gtx__bit_g0de51d5985e6a703f305a5a61479babd}


Interleaves the bits of x and y. The first bit is the first bit of x followed by the first bit of y. The other bits are interleaved following the previous sequence.

\begin{Desc}
\item[See also:]\hyperlink{group__gtx__bit}{GLM\_\-GTX\_\-bit} \end{Desc}


Definition at line 619 of file bit.inl.

References glm::bitfieldInterleave().

\begin{Code}\begin{verbatim}620         {
621                 union sign32
622                 {
623                         int32 i;
624                         uint32 u;
625                 } sign_x, sign_y;
626 
627                 union sign64
628                 {
629                         int64 i;
630                         uint64 u;
631                 } result;
632 
633                 sign_x.i = x;
634                 sign_y.i = y;
635                 result.u = bitfieldInterleave(sign_x.u, sign_y.u);
636 
637                 return result.i;
638         }
\end{verbatim}
\end{Code}




Here is the call graph for this function:\hypertarget{group__gtx__bit_g19ef8360379483e3ee245e89cb62ff93}{
\index{gtx\_\-bit@{gtx\_\-bit}!bitfieldInterleave@{bitfieldInterleave}}
\index{bitfieldInterleave@{bitfieldInterleave}!gtx_bit@{gtx\_\-bit}}
\subsubsection[bitfieldInterleave]{\setlength{\rightskip}{0pt plus 5cm}GLM\_\-FUNC\_\-QUALIFIER uint32 glm::bitfieldInterleave (uint16 {\em x}, \/  uint16 {\em y})\hspace{0.3cm}{\tt  \mbox{[}inline\mbox{]}}}}
\label{group__gtx__bit_g19ef8360379483e3ee245e89cb62ff93}


Interleaves the bits of x and y. The first bit is the first bit of x followed by the first bit of y. The other bits are interleaved following the previous sequence.

\begin{Desc}
\item[See also:]\hyperlink{group__gtx__bit}{GLM\_\-GTX\_\-bit} \end{Desc}


Definition at line 614 of file bit.inl.

\begin{Code}\begin{verbatim}615         {
616                 return detail::bitfieldInterleave<uint16, uint32>(x, y);
617         }
\end{verbatim}
\end{Code}


\hypertarget{group__gtx__bit_g1a0264598647ae00a596865af4e1e878}{
\index{gtx\_\-bit@{gtx\_\-bit}!bitfieldInterleave@{bitfieldInterleave}}
\index{bitfieldInterleave@{bitfieldInterleave}!gtx_bit@{gtx\_\-bit}}
\subsubsection[bitfieldInterleave]{\setlength{\rightskip}{0pt plus 5cm}GLM\_\-FUNC\_\-QUALIFIER int32 glm::bitfieldInterleave (int16 {\em x}, \/  int16 {\em y})}}
\label{group__gtx__bit_g1a0264598647ae00a596865af4e1e878}


Interleaves the bits of x and y. The first bit is the first bit of x followed by the first bit of y. The other bits are interleaved following the previous sequence.

\begin{Desc}
\item[See also:]\hyperlink{group__gtx__bit}{GLM\_\-GTX\_\-bit} \end{Desc}


Definition at line 593 of file bit.inl.

References glm::bitfieldInterleave().

\begin{Code}\begin{verbatim}594         {
595                 union sign16
596                 {
597                         int16 i;
598                         uint16 u;
599                 } sign_x, sign_y;
600 
601                 union sign32
602                 {
603                         int32 i;
604                         uint32 u;
605                 } result;
606 
607                 sign_x.i = x;
608                 sign_y.i = y;
609                 result.u = bitfieldInterleave(sign_x.u, sign_y.u);
610 
611                 return result.i;
612         }
\end{verbatim}
\end{Code}




Here is the call graph for this function:\hypertarget{group__gtx__bit_g0700a3ceb088a0ecc23d76c154096061}{
\index{gtx\_\-bit@{gtx\_\-bit}!bitfieldInterleave@{bitfieldInterleave}}
\index{bitfieldInterleave@{bitfieldInterleave}!gtx_bit@{gtx\_\-bit}}
\subsubsection[bitfieldInterleave]{\setlength{\rightskip}{0pt plus 5cm}GLM\_\-FUNC\_\-QUALIFIER uint16 glm::bitfieldInterleave (uint8 {\em x}, \/  uint8 {\em y})\hspace{0.3cm}{\tt  \mbox{[}inline\mbox{]}}}}
\label{group__gtx__bit_g0700a3ceb088a0ecc23d76c154096061}


Interleaves the bits of x and y. The first bit is the first bit of x followed by the first bit of y. The other bits are interleaved following the previous sequence.

\begin{Desc}
\item[See also:]\hyperlink{group__gtx__bit}{GLM\_\-GTX\_\-bit} \end{Desc}


Definition at line 588 of file bit.inl.

\begin{Code}\begin{verbatim}589         {
590                 return detail::bitfieldInterleave<uint8, uint16>(x, y);
591         }
\end{verbatim}
\end{Code}


\hypertarget{group__gtx__bit_g479134317bc95d99f2b2e235d3db287b}{
\index{gtx\_\-bit@{gtx\_\-bit}!bitfieldInterleave@{bitfieldInterleave}}
\index{bitfieldInterleave@{bitfieldInterleave}!gtx_bit@{gtx\_\-bit}}
\subsubsection[bitfieldInterleave]{\setlength{\rightskip}{0pt plus 5cm}GLM\_\-FUNC\_\-QUALIFIER int16 glm::bitfieldInterleave (int8 {\em x}, \/  int8 {\em y})}}
\label{group__gtx__bit_g479134317bc95d99f2b2e235d3db287b}


Interleaves the bits of x and y. The first bit is the first bit of x followed by the first bit of y. The other bits are interleaved following the previous sequence.

\begin{Desc}
\item[See also:]\hyperlink{group__gtx__bit}{GLM\_\-GTX\_\-bit} \end{Desc}


Definition at line 567 of file bit.inl.

Referenced by glm::bitfieldInterleave().

\begin{Code}\begin{verbatim}568         {
569                 union sign8
570                 {
571                         int8 i;
572                         uint8 u;
573                 } sign_x, sign_y;
574 
575                 union sign16
576                 {
577                         int16 i;
578                         uint16 u;
579                 } result;
580 
581                 sign_x.i = x;
582                 sign_y.i = y;
583                 result.u = bitfieldInterleave(sign_x.u, sign_y.u);
584 
585                 return result.i;
586         }
\end{verbatim}
\end{Code}




Here is the caller graph for this function:\hypertarget{group__gtx__bit_g878bcf889f80259fcf86d0e25db92af4}{
\index{gtx\_\-bit@{gtx\_\-bit}!bitRevert@{bitRevert}}
\index{bitRevert@{bitRevert}!gtx_bit@{gtx\_\-bit}}
\subsubsection[bitRevert]{\setlength{\rightskip}{0pt plus 5cm}template$<$typename genType$>$ GLM\_\-FUNC\_\-QUALIFIER genType glm::bitRevert (genType const \& {\em value})\hspace{0.3cm}{\tt  \mbox{[}inline\mbox{]}}}}
\label{group__gtx__bit_g878bcf889f80259fcf86d0e25db92af4}


Revert all bits of any integer based type. \begin{Desc}
\item[See also:]\hyperlink{group__gtx__bit}{GLM\_\-GTX\_\-bit} \end{Desc}


Definition at line 168 of file bit.inl.

\begin{Code}\begin{verbatim}169         {
170                 GLM_STATIC_ASSERT(std::numeric_limits<genType>::is_integer, "'bitRevert' only accept integer values");
171 
172                 genType Out = 0;
173                 std::size_t BitSize = sizeof(genType) * 8;
174                 for(std::size_t i = 0; i < BitSize; ++i)
175                         if(In & (genType(1) << i))
176                                 Out |= genType(1) << (BitSize - 1 - i);
177                 return Out;
178         }
\end{verbatim}
\end{Code}


\hypertarget{group__gtx__bit_g4cd980832ab9c35c73f8c76b8f309c92}{
\index{gtx\_\-bit@{gtx\_\-bit}!bitRotateLeft@{bitRotateLeft}}
\index{bitRotateLeft@{bitRotateLeft}!gtx_bit@{gtx\_\-bit}}
\subsubsection[bitRotateLeft]{\setlength{\rightskip}{0pt plus 5cm}template$<$typename genType$>$ GLM\_\-FUNC\_\-QUALIFIER genType glm::bitRotateLeft (genType const \& {\em In}, \/  std::size\_\-t {\em Shift})\hspace{0.3cm}{\tt  \mbox{[}inline\mbox{]}}}}
\label{group__gtx__bit_g4cd980832ab9c35c73f8c76b8f309c92}


Rotate all bits to the left. \begin{Desc}
\item[See also:]\hyperlink{group__gtx__bit}{GLM\_\-GTX\_\-bit} \end{Desc}


Definition at line 231 of file bit.inl.

\begin{Code}\begin{verbatim}232         {
233                 GLM_STATIC_ASSERT(std::numeric_limits<genType>::is_integer, "'bitRotateLeft' only accept integer values");
234 
235                 std::size_t BitSize = sizeof(genType) * 8;
236                 return (In >> Shift) | (In << (BitSize - Shift));
237         }
\end{verbatim}
\end{Code}


\hypertarget{group__gtx__bit_g01a6893d9fa57a4df1f6f7b0d399268d}{
\index{gtx\_\-bit@{gtx\_\-bit}!bitRotateRight@{bitRotateRight}}
\index{bitRotateRight@{bitRotateRight}!gtx_bit@{gtx\_\-bit}}
\subsubsection[bitRotateRight]{\setlength{\rightskip}{0pt plus 5cm}template$<$typename genType$>$ GLM\_\-FUNC\_\-QUALIFIER genType glm::bitRotateRight (genType const \& {\em In}, \/  std::size\_\-t {\em Shift})\hspace{0.3cm}{\tt  \mbox{[}inline\mbox{]}}}}
\label{group__gtx__bit_g01a6893d9fa57a4df1f6f7b0d399268d}


Rotate all bits to the right. \begin{Desc}
\item[See also:]\hyperlink{group__gtx__bit}{GLM\_\-GTX\_\-bit} \end{Desc}


Definition at line 183 of file bit.inl.

\begin{Code}\begin{verbatim}184         {
185                 GLM_STATIC_ASSERT(std::numeric_limits<genType>::is_integer, "'bitRotateRight' only accept integer values");
186 
187                 std::size_t BitSize = sizeof(genType) * 8;
188                 return (In << Shift) | (In >> (BitSize - Shift));
189         }
\end{verbatim}
\end{Code}


\hypertarget{group__gtx__bit_g98140d04738cfe05d70f090a7f1151f9}{
\index{gtx\_\-bit@{gtx\_\-bit}!fillBitfieldWithOne@{fillBitfieldWithOne}}
\index{fillBitfieldWithOne@{fillBitfieldWithOne}!gtx_bit@{gtx\_\-bit}}
\subsubsection[fillBitfieldWithOne]{\setlength{\rightskip}{0pt plus 5cm}template$<$typename genIUType$>$ GLM\_\-FUNC\_\-QUALIFIER genIUType glm::fillBitfieldWithOne (genIUType const \& {\em Value}, \/  int const \& {\em FromBit}, \/  int const \& {\em ToBit})\hspace{0.3cm}{\tt  \mbox{[}inline\mbox{]}}}}
\label{group__gtx__bit_g98140d04738cfe05d70f090a7f1151f9}


Set to 1 a range of bits. \begin{Desc}
\item[See also:]\hyperlink{group__gtx__bit}{GLM\_\-GTX\_\-bit} \end{Desc}


Definition at line 280 of file bit.inl.

\begin{Code}\begin{verbatim}285         {
286                 assert(FromBit <= ToBit);
287                 assert(ToBit <= sizeof(genIUType) * std::size_t(8));
288 
289                 genIUType Result = Value;
290                 for(signed i = 0; i <= ToBit; ++i)
291                         Result |= (1 << i);
292                 return Result;
293         }
\end{verbatim}
\end{Code}


\hypertarget{group__gtx__bit_gd7bf903dae07525ab89be5a3cad616a7}{
\index{gtx\_\-bit@{gtx\_\-bit}!fillBitfieldWithZero@{fillBitfieldWithZero}}
\index{fillBitfieldWithZero@{fillBitfieldWithZero}!gtx_bit@{gtx\_\-bit}}
\subsubsection[fillBitfieldWithZero]{\setlength{\rightskip}{0pt plus 5cm}template$<$typename genIUType$>$ GLM\_\-FUNC\_\-QUALIFIER genIUType glm::fillBitfieldWithZero (genIUType const \& {\em Value}, \/  int const \& {\em FromBit}, \/  int const \& {\em ToBit})\hspace{0.3cm}{\tt  \mbox{[}inline\mbox{]}}}}
\label{group__gtx__bit_gd7bf903dae07525ab89be5a3cad616a7}


Set to 0 a range of bits. \begin{Desc}
\item[See also:]\hyperlink{group__gtx__bit}{GLM\_\-GTX\_\-bit} \end{Desc}


Definition at line 297 of file bit.inl.

\begin{Code}\begin{verbatim}302         {
303                 assert(FromBit <= ToBit);
304                 assert(ToBit <= sizeof(genIUType) * std::size_t(8));
305 
306                 genIUType Result = Value;
307                 for(signed i = 0; i <= ToBit; ++i)
308                         Result &= ~(1 << i);
309                 return Result;
310         }
\end{verbatim}
\end{Code}


\hypertarget{group__gtx__bit_gda4310fc2dd8db30392da133067ed13e}{
\index{gtx\_\-bit@{gtx\_\-bit}!highestBitValue@{highestBitValue}}
\index{highestBitValue@{highestBitValue}!gtx_bit@{gtx\_\-bit}}
\subsubsection[highestBitValue]{\setlength{\rightskip}{0pt plus 5cm}template$<$typename genType$>$ GLM\_\-FUNC\_\-QUALIFIER genType glm::highestBitValue (genType const \& {\em value})\hspace{0.3cm}{\tt  \mbox{[}inline\mbox{]}}}}
\label{group__gtx__bit_gda4310fc2dd8db30392da133067ed13e}


Find the highest bit set to 1 in a integer variable and return its value. \begin{Desc}
\item[See also:]\hyperlink{group__gtx__bit}{GLM\_\-GTX\_\-bit} \end{Desc}


Definition at line 29 of file bit.inl.

Referenced by glm::powerOfTwoAbove(), glm::powerOfTwoBelow(), and glm::powerOfTwoNearest().

\begin{Code}\begin{verbatim}32         {
33                 genType tmp = value;
34                 genType result = genType(0);
35                 while(tmp)
36                 {
37                         result = (tmp & (~tmp + 1)); // grab lowest bit
38                         tmp &= ~result; // clear lowest bit
39                 }
40                 return result;
41         }
\end{verbatim}
\end{Code}




Here is the caller graph for this function:\hypertarget{group__gtx__bit_g2b12722968dabd423334391d1fd42acd}{
\index{gtx\_\-bit@{gtx\_\-bit}!isPowerOfTwo@{isPowerOfTwo}}
\index{isPowerOfTwo@{isPowerOfTwo}!gtx_bit@{gtx\_\-bit}}
\subsubsection[isPowerOfTwo]{\setlength{\rightskip}{0pt plus 5cm}template$<$typename genType$>$ GLM\_\-FUNC\_\-QUALIFIER bool glm::isPowerOfTwo (genType const \& {\em value})\hspace{0.3cm}{\tt  \mbox{[}inline\mbox{]}}}}
\label{group__gtx__bit_g2b12722968dabd423334391d1fd42acd}


Return true if the value is a power of two number. \begin{Desc}
\item[See also:]\hyperlink{group__gtx__bit}{GLM\_\-GTX\_\-bit} \end{Desc}


Definition at line 81 of file bit.inl.

References glm::abs().

Referenced by glm::powerOfTwoAbove(), glm::powerOfTwoBelow(), and glm::powerOfTwoNearest().

\begin{Code}\begin{verbatim}82         {
83                 //detail::If<std::numeric_limits<genType>::is_signed>::apply(abs, Value);
84                 //return !(Value & (Value - 1));
85 
86                 // For old complier?
87                 genType Result = Value;
88                 if(std::numeric_limits<genType>::is_signed)
89                         Result = abs(Result);
90                 return !(Result & (Result - 1));
91         }
\end{verbatim}
\end{Code}




Here is the call graph for this function:

Here is the caller graph for this function:\hypertarget{group__gtx__bit_g7a909d8e8b8a5f73ed643bdca7602017}{
\index{gtx\_\-bit@{gtx\_\-bit}!mask@{mask}}
\index{mask@{mask}!gtx_bit@{gtx\_\-bit}}
\subsubsection[mask]{\setlength{\rightskip}{0pt plus 5cm}template$<$typename genIType$>$ GLM\_\-FUNC\_\-QUALIFIER genIType glm::mask (genIType const \& {\em count})\hspace{0.3cm}{\tt  \mbox{[}inline\mbox{]}}}}
\label{group__gtx__bit_g7a909d8e8b8a5f73ed643bdca7602017}


Build a mask of 'count' bits \begin{Desc}
\item[See also:]\hyperlink{group__gtx__bit}{GLM\_\-GTX\_\-bit} \end{Desc}


Definition at line 17 of file bit.inl.

\begin{Code}\begin{verbatim}20         {
21                 return ((genIType(1) << (count)) - genIType(1));
22         }
\end{verbatim}
\end{Code}


\hypertarget{group__gtx__bit_gf27d271ec57b96b6acae9395b9c4a365}{
\index{gtx\_\-bit@{gtx\_\-bit}!powerOfTwoAbove@{powerOfTwoAbove}}
\index{powerOfTwoAbove@{powerOfTwoAbove}!gtx_bit@{gtx\_\-bit}}
\subsubsection[powerOfTwoAbove]{\setlength{\rightskip}{0pt plus 5cm}template$<$typename genType$>$ GLM\_\-FUNC\_\-QUALIFIER genType glm::powerOfTwoAbove (genType const \& {\em value})\hspace{0.3cm}{\tt  \mbox{[}inline\mbox{]}}}}
\label{group__gtx__bit_gf27d271ec57b96b6acae9395b9c4a365}


Return the power of two number which value is just higher the input value. \begin{Desc}
\item[See also:]\hyperlink{group__gtx__bit}{GLM\_\-GTX\_\-bit} \end{Desc}


Definition at line 131 of file bit.inl.

References glm::highestBitValue(), and glm::isPowerOfTwo().

\begin{Code}\begin{verbatim}132         {
133                 return isPowerOfTwo(value) ? value : highestBitValue(value) << 1;
134         }
\end{verbatim}
\end{Code}




Here is the call graph for this function:\hypertarget{group__gtx__bit_ga0bb1687b43f594a471c5506cc505dce}{
\index{gtx\_\-bit@{gtx\_\-bit}!powerOfTwoBelow@{powerOfTwoBelow}}
\index{powerOfTwoBelow@{powerOfTwoBelow}!gtx_bit@{gtx\_\-bit}}
\subsubsection[powerOfTwoBelow]{\setlength{\rightskip}{0pt plus 5cm}template$<$typename genType$>$ GLM\_\-FUNC\_\-QUALIFIER genType glm::powerOfTwoBelow (genType const \& {\em value})\hspace{0.3cm}{\tt  \mbox{[}inline\mbox{]}}}}
\label{group__gtx__bit_ga0bb1687b43f594a471c5506cc505dce}


Return the power of two number which value is just lower the input value. \begin{Desc}
\item[See also:]\hyperlink{group__gtx__bit}{GLM\_\-GTX\_\-bit} \end{Desc}


Definition at line 141 of file bit.inl.

References glm::highestBitValue(), and glm::isPowerOfTwo().

\begin{Code}\begin{verbatim}144         {
145                 return isPowerOfTwo(value) ? value : highestBitValue(value);
146         }
\end{verbatim}
\end{Code}




Here is the call graph for this function:\hypertarget{group__gtx__bit_g0e3c8f921e59dc07ad9c70bb1376799c}{
\index{gtx\_\-bit@{gtx\_\-bit}!powerOfTwoNearest@{powerOfTwoNearest}}
\index{powerOfTwoNearest@{powerOfTwoNearest}!gtx_bit@{gtx\_\-bit}}
\subsubsection[powerOfTwoNearest]{\setlength{\rightskip}{0pt plus 5cm}template$<$typename genType$>$ GLM\_\-FUNC\_\-QUALIFIER genType glm::powerOfTwoNearest (genType const \& {\em value})\hspace{0.3cm}{\tt  \mbox{[}inline\mbox{]}}}}
\label{group__gtx__bit_g0e3c8f921e59dc07ad9c70bb1376799c}


Return the power of two number which value is the closet to the input value. \begin{Desc}
\item[See also:]\hyperlink{group__gtx__bit}{GLM\_\-GTX\_\-bit} \end{Desc}


Definition at line 153 of file bit.inl.

References glm::highestBitValue(), and glm::isPowerOfTwo().

\begin{Code}\begin{verbatim}156         {
157                 if(isPowerOfTwo(value))
158                         return value;
159 
160                 genType prev = highestBitValue(value);
161                 genType next = prev << 1;
162                 return (next - value) < (value - prev) ? next : prev;
163         }
\end{verbatim}
\end{Code}




Here is the call graph for this function: