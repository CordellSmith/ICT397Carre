\hypertarget{group__gtx__color__space}{
\section{GLM\_\-GTX\_\-color\_\-space}
\label{group__gtx__color__space}\index{GLM\_\-GTX\_\-color\_\-space@{GLM\_\-GTX\_\-color\_\-space}}
}


Collaboration diagram for GLM\_\-GTX\_\-color\_\-space:Related to RGB to HSV conversions and operations.  
\subsection*{Functions}
\begin{CompactItemize}
\item 
{\footnotesize template$<$typename T, precision P$>$ }\\GLM\_\-FUNC\_\-DECL detail::tvec3$<$ T, P $>$ \hyperlink{group__gtx__color__space_g35a9210371395c95b185f5aac6c0020e}{glm::rgbColor} (detail::tvec3$<$ T, P $>$ const \&hsvValue)
\item 
{\footnotesize template$<$typename T, precision P$>$ }\\GLM\_\-FUNC\_\-DECL detail::tvec3$<$ T, P $>$ \hyperlink{group__gtx__color__space_g2532e85174ba333c4f60127d03d71655}{glm::hsvColor} (detail::tvec3$<$ T, P $>$ const \&rgbValue)
\item 
{\footnotesize template$<$typename T$>$ }\\GLM\_\-FUNC\_\-DECL detail::tmat4x4$<$ T, defaultp $>$ \hyperlink{group__gtx__color__space_g53a08c053e194bad0bfc172ef950bfe7}{glm::saturation} (T const s)
\item 
{\footnotesize template$<$typename T, precision P$>$ }\\GLM\_\-FUNC\_\-DECL detail::tvec3$<$ T, P $>$ \hyperlink{group__gtx__color__space_gc45433ff3d2f2f3657edfcf9ee24800d}{glm::saturation} (T const s, detail::tvec3$<$ T, P $>$ const \&color)
\item 
{\footnotesize template$<$typename T, precision P$>$ }\\GLM\_\-FUNC\_\-DECL detail::tvec4$<$ T, P $>$ \hyperlink{group__gtx__color__space_g1ab05270ac2afa8e67b0d268e5c92573}{glm::saturation} (T const s, detail::tvec4$<$ T, P $>$ const \&color)
\item 
{\footnotesize template$<$typename T, precision P$>$ }\\GLM\_\-FUNC\_\-DECL T \hyperlink{group__gtx__color__space_gdd1c8feae48a4fcf9e575648b25b914f}{glm::luminosity} (detail::tvec3$<$ T, P $>$ const \&color)
\end{CompactItemize}


\subsection{Detailed Description}
Related to RGB to HSV conversions and operations. 

$<$glm/gtx/color\_\-space.hpp$>$ need to be included to use these functionalities. 

\subsection{Function Documentation}
\hypertarget{group__gtx__color__space_g2532e85174ba333c4f60127d03d71655}{
\index{gtx\_\-color\_\-space@{gtx\_\-color\_\-space}!hsvColor@{hsvColor}}
\index{hsvColor@{hsvColor}!gtx_color_space@{gtx\_\-color\_\-space}}
\subsubsection[hsvColor]{\setlength{\rightskip}{0pt plus 5cm}template$<$typename T, precision P$>$ GLM\_\-FUNC\_\-QUALIFIER detail::tvec3$<$ T, P $>$ glm::hsvColor (detail::tvec3$<$ T, P $>$ const \& {\em rgbValue})\hspace{0.3cm}{\tt  \mbox{[}inline\mbox{]}}}}
\label{group__gtx__color__space_g2532e85174ba333c4f60127d03d71655}


Converts a color from RGB color space to its color in HSV color space. \begin{Desc}
\item[See also:]\hyperlink{group__gtx__color__space}{GLM\_\-GTX\_\-color\_\-space} \end{Desc}


Definition at line 70 of file color\_\-space.inl.

References glm::max(), and glm::min().

\begin{Code}\begin{verbatim}71         {
72                 detail::tvec3<T, P> hsv = rgbColor;
73                 float Min   = min(min(rgbColor.r, rgbColor.g), rgbColor.b);
74                 float Max   = max(max(rgbColor.r, rgbColor.g), rgbColor.b);
75                 float Delta = Max - Min;
76 
77                 hsv.z = Max;                               
78 
79                 if(Max != static_cast<T>(0))
80                 {
81                         hsv.y = Delta / hsv.z;    
82                         T h = static_cast<T>(0);
83 
84                         if(rgbColor.r == Max)
85                                 // between yellow & magenta
86                                 h = static_cast<T>(0) + T(60) * (rgbColor.g - rgbColor.b) / Delta;
87                         else if(rgbColor.g == Max)
88                                 // between cyan & yellow
89                                 h = static_cast<T>(120) + T(60) * (rgbColor.b - rgbColor.r) / Delta;
90                         else
91                                 // between magenta & cyan
92                                 h = static_cast<T>(240) + T(60) * (rgbColor.r - rgbColor.g) / Delta;
93 
94                         if(h < T(0)) 
95                                 hsv.x = h + T(360);
96                         else
97                                 hsv.x = h;
98                 }
99                 else
100                 {
101                         // If r = g = b = 0 then s = 0, h is undefined
102                         hsv.y = static_cast<T>(0);
103                         hsv.x = static_cast<T>(0);
104                 }
105 
106                 return hsv;
107         }
\end{verbatim}
\end{Code}




Here is the call graph for this function:\hypertarget{group__gtx__color__space_gdd1c8feae48a4fcf9e575648b25b914f}{
\index{gtx\_\-color\_\-space@{gtx\_\-color\_\-space}!luminosity@{luminosity}}
\index{luminosity@{luminosity}!gtx_color_space@{gtx\_\-color\_\-space}}
\subsubsection[luminosity]{\setlength{\rightskip}{0pt plus 5cm}template$<$typename T, precision P$>$ GLM\_\-FUNC\_\-QUALIFIER T glm::luminosity (detail::tvec3$<$ T, P $>$ const \& {\em color})\hspace{0.3cm}{\tt  \mbox{[}inline\mbox{]}}}}
\label{group__gtx__color__space_gdd1c8feae48a4fcf9e575648b25b914f}


Compute color luminosity associating ratios (0.33, 0.59, 0.11) to RGB canals. \begin{Desc}
\item[See also:]\hyperlink{group__gtx__color__space}{GLM\_\-GTX\_\-color\_\-space} \end{Desc}


Definition at line 144 of file color\_\-space.inl.

References glm::dot().

\begin{Code}\begin{verbatim}145         {
146                 const detail::tvec3<T, P> tmp = detail::tvec3<T, P>(0.33, 0.59, 0.11);
147                 return dot(color, tmp);
148         }
\end{verbatim}
\end{Code}




Here is the call graph for this function:\hypertarget{group__gtx__color__space_g35a9210371395c95b185f5aac6c0020e}{
\index{gtx\_\-color\_\-space@{gtx\_\-color\_\-space}!rgbColor@{rgbColor}}
\index{rgbColor@{rgbColor}!gtx_color_space@{gtx\_\-color\_\-space}}
\subsubsection[rgbColor]{\setlength{\rightskip}{0pt plus 5cm}template$<$typename T, precision P$>$ GLM\_\-FUNC\_\-QUALIFIER detail::tvec3$<$ T, P $>$ glm::rgbColor (detail::tvec3$<$ T, P $>$ const \& {\em hsvValue})\hspace{0.3cm}{\tt  \mbox{[}inline\mbox{]}}}}
\label{group__gtx__color__space_g35a9210371395c95b185f5aac6c0020e}


Converts a color from HSV color space to its color in RGB color space. \begin{Desc}
\item[See also:]\hyperlink{group__gtx__color__space}{GLM\_\-GTX\_\-color\_\-space} \end{Desc}


Definition at line 13 of file color\_\-space.inl.

References glm::floor().

\begin{Code}\begin{verbatim}14         {
15                 detail::tvec3<T, P> hsv = hsvColor;
16                 detail::tvec3<T, P> rgbColor;
17 
18                 if(hsv.y == static_cast<T>(0))
19                         // achromatic (grey)
20                         rgbColor = detail::tvec3<T, P>(hsv.z);
21                 else
22                 {
23                         T sector = floor(hsv.x / T(60));
24                         T frac = (hsv.x / T(60)) - sector;
25                         // factorial part of h
26                         T o = hsv.z * (T(1) - hsv.y);
27                         T p = hsv.z * (T(1) - hsv.y * frac);
28                         T q = hsv.z * (T(1) - hsv.y * (T(1) - frac));
29 
30                         switch(int(sector))
31                         {
32                         default:
33                         case 0:
34                                 rgbColor.r = hsv.z;
35                                 rgbColor.g = q;
36                                 rgbColor.b = o;
37                                 break;
38                         case 1:
39                                 rgbColor.r = p;
40                                 rgbColor.g = hsv.z;
41                                 rgbColor.b = o;
42                                 break;
43                         case 2:
44                                 rgbColor.r = o;
45                                 rgbColor.g = hsv.z;
46                                 rgbColor.b = q;
47                                 break;
48                         case 3:
49                                 rgbColor.r = o;
50                                 rgbColor.g = p;
51                                 rgbColor.b = hsv.z;
52                                 break;
53                         case 4:
54                                 rgbColor.r = q; 
55                                 rgbColor.g = o; 
56                                 rgbColor.b = hsv.z;
57                                 break;
58                         case 5:
59                                 rgbColor.r = hsv.z; 
60                                 rgbColor.g = o; 
61                                 rgbColor.b = p;
62                                 break;
63                         }
64                 }
65 
66                 return rgbColor;
67         }
\end{verbatim}
\end{Code}




Here is the call graph for this function:\hypertarget{group__gtx__color__space_g1ab05270ac2afa8e67b0d268e5c92573}{
\index{gtx\_\-color\_\-space@{gtx\_\-color\_\-space}!saturation@{saturation}}
\index{saturation@{saturation}!gtx_color_space@{gtx\_\-color\_\-space}}
\subsubsection[saturation]{\setlength{\rightskip}{0pt plus 5cm}template$<$typename T, precision P$>$ GLM\_\-FUNC\_\-QUALIFIER detail::tvec4$<$ T, P $>$ glm::saturation (T const {\em s}, \/  detail::tvec4$<$ T, P $>$ const \& {\em color})\hspace{0.3cm}{\tt  \mbox{[}inline\mbox{]}}}}
\label{group__gtx__color__space_g1ab05270ac2afa8e67b0d268e5c92573}


Modify the saturation of a color. \begin{Desc}
\item[See also:]\hyperlink{group__gtx__color__space}{GLM\_\-GTX\_\-color\_\-space} \end{Desc}


Definition at line 138 of file color\_\-space.inl.

References glm::saturation().

\begin{Code}\begin{verbatim}139         {
140                 return saturation(s) * color;
141         }
\end{verbatim}
\end{Code}




Here is the call graph for this function:\hypertarget{group__gtx__color__space_gc45433ff3d2f2f3657edfcf9ee24800d}{
\index{gtx\_\-color\_\-space@{gtx\_\-color\_\-space}!saturation@{saturation}}
\index{saturation@{saturation}!gtx_color_space@{gtx\_\-color\_\-space}}
\subsubsection[saturation]{\setlength{\rightskip}{0pt plus 5cm}template$<$typename T, precision P$>$ GLM\_\-FUNC\_\-QUALIFIER detail::tvec3$<$ T, P $>$ glm::saturation (T const {\em s}, \/  detail::tvec3$<$ T, P $>$ const \& {\em color})\hspace{0.3cm}{\tt  \mbox{[}inline\mbox{]}}}}
\label{group__gtx__color__space_gc45433ff3d2f2f3657edfcf9ee24800d}


Modify the saturation of a color. \begin{Desc}
\item[See also:]\hyperlink{group__gtx__color__space}{GLM\_\-GTX\_\-color\_\-space} \end{Desc}


Definition at line 132 of file color\_\-space.inl.

References glm::saturation().

\begin{Code}\begin{verbatim}133         {
134                 return detail::tvec3<T, P>(saturation(s) * detail::tvec4<T, P>(color, T(0)));
135         }
\end{verbatim}
\end{Code}




Here is the call graph for this function:\hypertarget{group__gtx__color__space_g53a08c053e194bad0bfc172ef950bfe7}{
\index{gtx\_\-color\_\-space@{gtx\_\-color\_\-space}!saturation@{saturation}}
\index{saturation@{saturation}!gtx_color_space@{gtx\_\-color\_\-space}}
\subsubsection[saturation]{\setlength{\rightskip}{0pt plus 5cm}template$<$typename T$>$ GLM\_\-FUNC\_\-QUALIFIER detail::tmat4x4$<$ T, defaultp $>$ glm::saturation (T const  {\em s})\hspace{0.3cm}{\tt  \mbox{[}inline\mbox{]}}}}
\label{group__gtx__color__space_g53a08c053e194bad0bfc172ef950bfe7}


Build a saturation matrix. \begin{Desc}
\item[See also:]\hyperlink{group__gtx__color__space}{GLM\_\-GTX\_\-color\_\-space} \end{Desc}


Definition at line 110 of file color\_\-space.inl.

Referenced by glm::saturation().

\begin{Code}\begin{verbatim}111         {
112                 detail::tvec3<T, defaultp> rgbw = detail::tvec3<T, defaultp>(T(0.2126), T(0.7152), T(0.0722));
113 
114                 T col0 = (T(1) - s) * rgbw.r;
115                 T col1 = (T(1) - s) * rgbw.g;
116                 T col2 = (T(1) - s) * rgbw.b;
117 
118                 detail::tmat4x4<T, defaultp> result(T(1));
119                 result[0][0] = col0 + s;
120                 result[0][1] = col0;
121                 result[0][2] = col0;
122                 result[1][0] = col1;
123                 result[1][1] = col1 + s;
124                 result[1][2] = col1;
125                 result[2][0] = col2;
126                 result[2][1] = col2;
127                 result[2][2] = col2 + s;
128                 return result;
129         }
\end{verbatim}
\end{Code}




Here is the caller graph for this function: