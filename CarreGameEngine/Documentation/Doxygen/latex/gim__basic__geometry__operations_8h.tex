\hypertarget{gim__basic__geometry__operations_8h}{
\section{C:/Users/New/Documents/Games\_\-Technology/Year4\_\-Semester1/ICT397/$\sim$My Work/Assignment2/ICT397Carre/CarreGameEngine/Dependencies/BulletPhysicsEngine/include/BulletCollision/Gimpact/gim\_\-basic\_\-geometry\_\-operations.h File Reference}
\label{gim__basic__geometry__operations_8h}\index{C:/Users/New/Documents/Games\_\-Technology/Year4\_\-Semester1/ICT397/$\sim$My Work/Assignment2/ICT397Carre/CarreGameEngine/Dependencies/BulletPhysicsEngine/include/BulletCollision/Gimpact/gim\_\-basic\_\-geometry\_\-operations.h@{C:/Users/New/Documents/Games\_\-Technology/Year4\_\-Semester1/ICT397/$\sim$My Work/Assignment2/ICT397Carre/CarreGameEngine/Dependencies/BulletPhysicsEngine/include/BulletCollision/Gimpact/gim\_\-basic\_\-geometry\_\-operations.h}}
}
{\tt \#include \char`\"{}gim\_\-linear\_\-math.h\char`\"{}}\par


Include dependency graph for gim\_\-basic\_\-geometry\_\-operations.h:

This graph shows which files directly or indirectly include this file:\subsection*{Defines}
\begin{CompactItemize}
\item 
\hypertarget{gim__basic__geometry__operations_8h_3a555c6ce6f2cba54d48cf5397616975}{
\#define \hyperlink{gim__basic__geometry__operations_8h_3a555c6ce6f2cba54d48cf5397616975}{TRIANGLE\_\-PLANE}(v1, v2, v3, plane)}
\label{gim__basic__geometry__operations_8h_3a555c6ce6f2cba54d48cf5397616975}

\begin{CompactList}\small\item\em plane is a vec4f \item\end{CompactList}\item 
\hypertarget{gim__basic__geometry__operations_8h_488bb318b2847ded4f0d7d5ca9f6f62e}{
\#define \hyperlink{gim__basic__geometry__operations_8h_488bb318b2847ded4f0d7d5ca9f6f62e}{TRIANGLE\_\-PLANE\_\-FAST}(v1, v2, v3, plane)}
\label{gim__basic__geometry__operations_8h_488bb318b2847ded4f0d7d5ca9f6f62e}

\begin{CompactList}\small\item\em plane is a vec4f \item\end{CompactList}\item 
\hypertarget{gim__basic__geometry__operations_8h_4f359987a785fce8ab0f3be33335f9d9}{
\#define \hyperlink{gim__basic__geometry__operations_8h_4f359987a785fce8ab0f3be33335f9d9}{EDGE\_\-PLANE}(e1, e2, n, plane)}
\label{gim__basic__geometry__operations_8h_4f359987a785fce8ab0f3be33335f9d9}

\begin{CompactList}\small\item\em Calc a plane from an edge an a normal. plane is a vec4f. \item\end{CompactList}\item 
\hypertarget{gim__basic__geometry__operations_8h_c5e074cfe96be68f9c9f17174243ff20}{
\#define \hyperlink{gim__basic__geometry__operations_8h_c5e074cfe96be68f9c9f17174243ff20}{PLANE\_\-MINOR\_\-AXES}(plane, i0, i1)~VEC\_\-MINOR\_\-AXES(plane, i0, i1)}
\label{gim__basic__geometry__operations_8h_c5e074cfe96be68f9c9f17174243ff20}

\begin{CompactList}\small\item\em Finds the 2 smallest cartesian coordinates of a plane normal. \item\end{CompactList}\end{CompactItemize}
\subsection*{Functions}
\begin{CompactItemize}
\item 
\hypertarget{gim__basic__geometry__operations_8h_bbba9933965e587594244b81ceb32c04}{
{\footnotesize template$<$typename CLASS\_\-POINT, typename CLASS\_\-PLANE$>$ }\\SIMD\_\-FORCE\_\-INLINE bool \hyperlink{gim__basic__geometry__operations_8h_bbba9933965e587594244b81ceb32c04}{POINT\_\-IN\_\-HULL} (const CLASS\_\-POINT \&point, const CLASS\_\-PLANE $\ast$planes, GUINT plane\_\-count)}
\label{gim__basic__geometry__operations_8h_bbba9933965e587594244b81ceb32c04}

\begin{CompactList}\small\item\em Verifies if a point is in the plane hull. \item\end{CompactList}\item 
{\footnotesize template$<$typename CLASS\_\-POINT, typename CLASS\_\-PLANE$>$ }\\SIMD\_\-FORCE\_\-INLINE eLINE\_\-PLANE\_\-INTERSECTION\_\-TYPE \hyperlink{gim__basic__geometry__operations_8h_c1357f9a99f5709463c3c11874c993c4}{PLANE\_\-CLIP\_\-SEGMENT2} (const CLASS\_\-POINT \&s1, const CLASS\_\-POINT \&s2, const CLASS\_\-PLANE \&plane, CLASS\_\-POINT \&clipped)
\begin{CompactList}\small\item\em Confirms if the plane intersect the edge or nor. \item\end{CompactList}\item 
{\footnotesize template$<$typename CLASS\_\-POINT, typename CLASS\_\-PLANE$>$ }\\SIMD\_\-FORCE\_\-INLINE eLINE\_\-PLANE\_\-INTERSECTION\_\-TYPE \hyperlink{gim__basic__geometry__operations_8h_a931e17a661df001eaa717335ae537ec}{PLANE\_\-CLIP\_\-SEGMENT\_\-CLOSEST} (const CLASS\_\-POINT \&s1, const CLASS\_\-POINT \&s2, const CLASS\_\-PLANE \&plane, CLASS\_\-POINT \&clipped1, CLASS\_\-POINT \&clipped2)
\begin{CompactList}\small\item\em Confirms if the plane intersect the edge or not. \item\end{CompactList}\item 
{\footnotesize template$<$typename T, typename CLASS\_\-POINT, typename CLASS\_\-PLANE$>$ }\\SIMD\_\-FORCE\_\-INLINE bool \hyperlink{gim__basic__geometry__operations_8h_76424e5fa2f0f7697652e0088e296397}{RAY\_\-PLANE\_\-COLLISION} (const CLASS\_\-PLANE \&plane, const CLASS\_\-POINT \&vDir, const CLASS\_\-POINT \&vPoint, CLASS\_\-POINT \&pout, T \&tparam)
\begin{CompactList}\small\item\em Ray plane collision in one way. \item\end{CompactList}\item 
{\footnotesize template$<$typename T, typename CLASS\_\-POINT, typename CLASS\_\-PLANE$>$ }\\SIMD\_\-FORCE\_\-INLINE GUINT \hyperlink{gim__basic__geometry__operations_8h_e039e4c642dd58adc8e99d774db9342d}{LINE\_\-PLANE\_\-COLLISION} (const CLASS\_\-PLANE \&plane, const CLASS\_\-POINT \&vDir, const CLASS\_\-POINT \&vPoint, CLASS\_\-POINT \&pout, T \&tparam, T tmin, T tmax)
\begin{CompactList}\small\item\em line collision \item\end{CompactList}\item 
{\footnotesize template$<$typename CLASS\_\-POINT, typename CLASS\_\-PLANE$>$ }\\SIMD\_\-FORCE\_\-INLINE bool \hyperlink{gim__basic__geometry__operations_8h_27491c62dc80b8efd8be348f6b41f5cb}{INTERSECT\_\-PLANES} (const CLASS\_\-PLANE \&p1, const CLASS\_\-PLANE \&p2, CLASS\_\-POINT \&p, CLASS\_\-POINT \&d)
\begin{CompactList}\small\item\em Returns the Ray on which 2 planes intersect if they do. Written by Rodrigo Hernandez on ODE convex collision. \item\end{CompactList}\item 
{\footnotesize template$<$typename CLASS\_\-POINT$>$ }\\SIMD\_\-FORCE\_\-INLINE void \hyperlink{gim__basic__geometry__operations_8h_b66e93d424a6170788e539099622709d}{CLOSEST\_\-POINT\_\-ON\_\-SEGMENT} (CLASS\_\-POINT \&cp, const CLASS\_\-POINT \&v, const CLASS\_\-POINT \&e1, const CLASS\_\-POINT \&e2)
\item 
{\footnotesize template$<$typename T, typename CLASS\_\-POINT$>$ }\\SIMD\_\-FORCE\_\-INLINE bool \hyperlink{gim__basic__geometry__operations_8h_1e7d4f2ec752876cb69dbfc273aa808b}{LINE\_\-INTERSECTION\_\-PARAMS} (const CLASS\_\-POINT \&dir1, CLASS\_\-POINT \&point1, const CLASS\_\-POINT \&dir2, CLASS\_\-POINT \&point2, T \&t1, T \&t2)
\begin{CompactList}\small\item\em Finds the line params where these lines intersect. \item\end{CompactList}\item 
\hypertarget{gim__basic__geometry__operations_8h_8a9e46c990cbb6a4aa3e85e4c75aad85}{
{\footnotesize template$<$typename CLASS\_\-POINT$>$ }\\SIMD\_\-FORCE\_\-INLINE void \hyperlink{gim__basic__geometry__operations_8h_8a9e46c990cbb6a4aa3e85e4c75aad85}{SEGMENT\_\-COLLISION} (const CLASS\_\-POINT \&vA1, const CLASS\_\-POINT \&vA2, const CLASS\_\-POINT \&vB1, const CLASS\_\-POINT \&vB2, CLASS\_\-POINT \&vPointA, CLASS\_\-POINT \&vPointB)}
\label{gim__basic__geometry__operations_8h_8a9e46c990cbb6a4aa3e85e4c75aad85}

\begin{CompactList}\small\item\em Find closest points on segments. \item\end{CompactList}\item 
{\footnotesize template$<$typename T$>$ }\\SIMD\_\-FORCE\_\-INLINE bool \hyperlink{gim__basic__geometry__operations_8h_263c8b6948964b535bc3ea3e095a7bdf}{BOX\_\-AXIS\_\-INTERSECT} (T pos, T dir, T bmin, T bmax, T \&tfirst, T \&tlast)
\begin{CompactList}\small\item\em Line box intersection in one dimension. \item\end{CompactList}\item 
\hypertarget{gim__basic__geometry__operations_8h_fe4f0bb3d7ff5247cdd216dd723da675}{
{\footnotesize template$<$typename T$>$ }\\SIMD\_\-FORCE\_\-INLINE void \hyperlink{gim__basic__geometry__operations_8h_fe4f0bb3d7ff5247cdd216dd723da675}{SORT\_\-3\_\-INDICES} (const T $\ast$values, GUINT $\ast$order\_\-indices)}
\label{gim__basic__geometry__operations_8h_fe4f0bb3d7ff5247cdd216dd723da675}

\begin{CompactList}\small\item\em Sorts 3 componets. \item\end{CompactList}\end{CompactItemize}


\subsection{Detailed Description}
\begin{Desc}
\item[Author:]Francisco Leon Najera type independant geometry routines \end{Desc}


Definition in file \hyperlink{gim__basic__geometry__operations_8h-source}{gim\_\-basic\_\-geometry\_\-operations.h}.

\subsection{Function Documentation}
\hypertarget{gim__basic__geometry__operations_8h_263c8b6948964b535bc3ea3e095a7bdf}{
\index{gim\_\-basic\_\-geometry\_\-operations.h@{gim\_\-basic\_\-geometry\_\-operations.h}!BOX\_\-AXIS\_\-INTERSECT@{BOX\_\-AXIS\_\-INTERSECT}}
\index{BOX\_\-AXIS\_\-INTERSECT@{BOX\_\-AXIS\_\-INTERSECT}!gim_basic_geometry_operations.h@{gim\_\-basic\_\-geometry\_\-operations.h}}
\subsubsection[BOX\_\-AXIS\_\-INTERSECT]{\setlength{\rightskip}{0pt plus 5cm}template$<$typename T$>$ SIMD\_\-FORCE\_\-INLINE bool BOX\_\-AXIS\_\-INTERSECT (T {\em pos}, \/  T {\em dir}, \/  T {\em bmin}, \/  T {\em bmax}, \/  T \& {\em tfirst}, \/  T \& {\em tlast})\hspace{0.3cm}{\tt  \mbox{[}inline\mbox{]}}}}
\label{gim__basic__geometry__operations_8h_263c8b6948964b535bc3ea3e095a7bdf}


Line box intersection in one dimension. 

\begin{Desc}
\item[Parameters:]
\begin{description}
\item[{\em pos}]Position of the ray \item[{\em dir}]Projection of the Direction of the ray \item[{\em bmin}]Minimum bound of the box \item[{\em bmax}]Maximum bound of the box \item[{\em tfirst}]the minimum projection. Assign to 0 at first. \item[{\em tlast}]the maximum projection. Assign to INFINITY at first. \end{description}
\end{Desc}
\begin{Desc}
\item[Returns:]true if there is an intersection. \end{Desc}


Definition at line 501 of file gim\_\-basic\_\-geometry\_\-operations.h.

References GIM\_\-SWAP\_\-NUMBERS.

\begin{Code}\begin{verbatim}502 {
503         if(GIM_IS_ZERO(dir))
504         {
505         return !(pos < bmin || pos > bmax);
506         }
507         GREAL a0 = (bmin - pos) / dir;
508         GREAL a1 = (bmax - pos) / dir;
509         if(a0 > a1)   GIM_SWAP_NUMBERS(a0, a1);
510         tfirst = GIM_MAX(a0, tfirst);
511         tlast = GIM_MIN(a1, tlast);
512         if (tlast < tfirst) return false;
513         return true;
514 }
\end{verbatim}
\end{Code}


\hypertarget{gim__basic__geometry__operations_8h_b66e93d424a6170788e539099622709d}{
\index{gim\_\-basic\_\-geometry\_\-operations.h@{gim\_\-basic\_\-geometry\_\-operations.h}!CLOSEST\_\-POINT\_\-ON\_\-SEGMENT@{CLOSEST\_\-POINT\_\-ON\_\-SEGMENT}}
\index{CLOSEST\_\-POINT\_\-ON\_\-SEGMENT@{CLOSEST\_\-POINT\_\-ON\_\-SEGMENT}!gim_basic_geometry_operations.h@{gim\_\-basic\_\-geometry\_\-operations.h}}
\subsubsection[CLOSEST\_\-POINT\_\-ON\_\-SEGMENT]{\setlength{\rightskip}{0pt plus 5cm}template$<$typename CLASS\_\-POINT$>$ SIMD\_\-FORCE\_\-INLINE void CLOSEST\_\-POINT\_\-ON\_\-SEGMENT (CLASS\_\-POINT \& {\em cp}, \/  const CLASS\_\-POINT \& {\em v}, \/  const CLASS\_\-POINT \& {\em e1}, \/  const CLASS\_\-POINT \& {\em e2})\hspace{0.3cm}{\tt  \mbox{[}inline\mbox{]}}}}
\label{gim__basic__geometry__operations_8h_b66e93d424a6170788e539099622709d}


Finds the closest point(cp) to (v) on a segment (e1,e2) 

Definition at line 341 of file gim\_\-basic\_\-geometry\_\-operations.h.

References VEC\_\-COPY, VEC\_\-DIFF, VEC\_\-DOT, VEC\_\-SCALE, and VEC\_\-SUM.

Referenced by SEGMENT\_\-COLLISION().

\begin{Code}\begin{verbatim}344 {
345     vec3f _n;
346     VEC_DIFF(_n,e2,e1);
347     VEC_DIFF(cp,v,e1);
348         GREAL _scalar = VEC_DOT(cp, _n);
349         _scalar/= VEC_DOT(_n, _n);
350         if(_scalar <0.0f)
351         {
352             VEC_COPY(cp,e1);
353         }
354         else if(_scalar >1.0f)
355         {
356             VEC_COPY(cp,e2);
357         }
358         else
359         {
360         VEC_SCALE(cp,_scalar,_n);
361         VEC_SUM(cp,cp,e1);
362         }
363 }
\end{verbatim}
\end{Code}




Here is the caller graph for this function:\hypertarget{gim__basic__geometry__operations_8h_27491c62dc80b8efd8be348f6b41f5cb}{
\index{gim\_\-basic\_\-geometry\_\-operations.h@{gim\_\-basic\_\-geometry\_\-operations.h}!INTERSECT\_\-PLANES@{INTERSECT\_\-PLANES}}
\index{INTERSECT\_\-PLANES@{INTERSECT\_\-PLANES}!gim_basic_geometry_operations.h@{gim\_\-basic\_\-geometry\_\-operations.h}}
\subsubsection[INTERSECT\_\-PLANES]{\setlength{\rightskip}{0pt plus 5cm}template$<$typename CLASS\_\-POINT, typename CLASS\_\-PLANE$>$ SIMD\_\-FORCE\_\-INLINE bool INTERSECT\_\-PLANES (const CLASS\_\-PLANE \& {\em p1}, \/  const CLASS\_\-PLANE \& {\em p2}, \/  CLASS\_\-POINT \& {\em p}, \/  CLASS\_\-POINT \& {\em d})\hspace{0.3cm}{\tt  \mbox{[}inline\mbox{]}}}}
\label{gim__basic__geometry__operations_8h_27491c62dc80b8efd8be348f6b41f5cb}


Returns the Ray on which 2 planes intersect if they do. Written by Rodrigo Hernandez on ODE convex collision. 

\begin{Desc}
\item[Parameters:]
\begin{description}
\item[{\em p1}]Plane 1 \item[{\em p2}]Plane 2 \item[{\em p}]Contains the origin of the ray upon returning if planes intersect \item[{\em d}]Contains the direction of the ray upon returning if planes intersect \end{description}
\end{Desc}
\begin{Desc}
\item[Returns:]true if the planes intersect, 0 if paralell. \end{Desc}


Definition at line 316 of file gim\_\-basic\_\-geometry\_\-operations.h.

References VEC\_\-CROSS, and VEC\_\-DOT.

\begin{Code}\begin{verbatim}321 {
322         VEC_CROSS(d,p1,p2);
323         GREAL denom = VEC_DOT(d, d);
324         if(GIM_IS_ZERO(denom)) return false;
325         vec3f _n;
326         _n[0]=p1[3]*p2[0] - p2[3]*p1[0];
327         _n[1]=p1[3]*p2[1] - p2[3]*p1[1];
328         _n[2]=p1[3]*p2[2] - p2[3]*p1[2];
329         VEC_CROSS(p,_n,d);
330         p[0]/=denom;
331         p[1]/=denom;
332         p[2]/=denom;
333         return true;
334 }
\end{verbatim}
\end{Code}


\hypertarget{gim__basic__geometry__operations_8h_1e7d4f2ec752876cb69dbfc273aa808b}{
\index{gim\_\-basic\_\-geometry\_\-operations.h@{gim\_\-basic\_\-geometry\_\-operations.h}!LINE\_\-INTERSECTION\_\-PARAMS@{LINE\_\-INTERSECTION\_\-PARAMS}}
\index{LINE\_\-INTERSECTION\_\-PARAMS@{LINE\_\-INTERSECTION\_\-PARAMS}!gim_basic_geometry_operations.h@{gim\_\-basic\_\-geometry\_\-operations.h}}
\subsubsection[LINE\_\-INTERSECTION\_\-PARAMS]{\setlength{\rightskip}{0pt plus 5cm}template$<$typename T, typename CLASS\_\-POINT$>$ SIMD\_\-FORCE\_\-INLINE bool LINE\_\-INTERSECTION\_\-PARAMS (const CLASS\_\-POINT \& {\em dir1}, \/  CLASS\_\-POINT \& {\em point1}, \/  const CLASS\_\-POINT \& {\em dir2}, \/  CLASS\_\-POINT \& {\em point2}, \/  T \& {\em t1}, \/  T \& {\em t2})\hspace{0.3cm}{\tt  \mbox{[}inline\mbox{]}}}}
\label{gim__basic__geometry__operations_8h_1e7d4f2ec752876cb69dbfc273aa808b}


Finds the line params where these lines intersect. 

\begin{Desc}
\item[Parameters:]
\begin{description}
\item[{\em dir1}]Direction of line 1 \item[{\em point1}]Point of line 1 \item[{\em dir2}]Direction of line 2 \item[{\em point2}]Point of line 2 \item[{\em t1}]Result Parameter for line 1 \item[{\em t2}]Result Parameter for line 2 \item[{\em dointersect}]0 if the lines won't intersect, else 1 \end{description}
\end{Desc}


Definition at line 378 of file gim\_\-basic\_\-geometry\_\-operations.h.

References VEC\_\-DIFF, and VEC\_\-DOT.

\begin{Code}\begin{verbatim}384 {
385     GREAL det;
386         GREAL e1e1 = VEC_DOT(dir1,dir1);
387         GREAL e1e2 = VEC_DOT(dir1,dir2);
388         GREAL e2e2 = VEC_DOT(dir2,dir2);
389         vec3f p1p2;
390     VEC_DIFF(p1p2,point1,point2);
391     GREAL p1p2e1 = VEC_DOT(p1p2,dir1);
392         GREAL p1p2e2 = VEC_DOT(p1p2,dir2);
393         det = e1e2*e1e2 - e1e1*e2e2;
394         if(GIM_IS_ZERO(det)) return false;
395         t1 = (e1e2*p1p2e2 - e2e2*p1p2e1)/det;
396         t2 = (e1e1*p1p2e2 - e1e2*p1p2e1)/det;
397         return true;
398 }
\end{verbatim}
\end{Code}


\hypertarget{gim__basic__geometry__operations_8h_e039e4c642dd58adc8e99d774db9342d}{
\index{gim\_\-basic\_\-geometry\_\-operations.h@{gim\_\-basic\_\-geometry\_\-operations.h}!LINE\_\-PLANE\_\-COLLISION@{LINE\_\-PLANE\_\-COLLISION}}
\index{LINE\_\-PLANE\_\-COLLISION@{LINE\_\-PLANE\_\-COLLISION}!gim_basic_geometry_operations.h@{gim\_\-basic\_\-geometry\_\-operations.h}}
\subsubsection[LINE\_\-PLANE\_\-COLLISION]{\setlength{\rightskip}{0pt plus 5cm}template$<$typename T, typename CLASS\_\-POINT, typename CLASS\_\-PLANE$>$ SIMD\_\-FORCE\_\-INLINE GUINT LINE\_\-PLANE\_\-COLLISION (const CLASS\_\-PLANE \& {\em plane}, \/  const CLASS\_\-POINT \& {\em vDir}, \/  const CLASS\_\-POINT \& {\em vPoint}, \/  CLASS\_\-POINT \& {\em pout}, \/  T \& {\em tparam}, \/  T {\em tmin}, \/  T {\em tmax})\hspace{0.3cm}{\tt  \mbox{[}inline\mbox{]}}}}
\label{gim__basic__geometry__operations_8h_e039e4c642dd58adc8e99d774db9342d}


line collision 

\begin{Desc}
\item[Returns:]-0 if the ray never intersects -1 if the ray collides in front -2 if the ray collides in back \end{Desc}


Definition at line 270 of file gim\_\-basic\_\-geometry\_\-operations.h.

References VEC\_\-DOT, VEC\_\-SCALE, and VEC\_\-SUM.

Referenced by GIM\_\-TRIANGLE::ray\_\-collision(), GIM\_\-TRIANGLE::ray\_\-collision\_\-front\_\-side(), and SEGMENT\_\-COLLISION().

\begin{Code}\begin{verbatim}277 {
278         GREAL _dis,_dotdir;
279         _dotdir = VEC_DOT(plane,vDir);
280         if(btFabs(_dotdir)<PLANEDIREPSILON)
281         {
282                 tparam = tmax;
283             return 0;
284         }
285         _dis = DISTANCE_PLANE_POINT(plane,vPoint);
286         char returnvalue = _dis<0.0f?2:1;
287         tparam = -_dis/_dotdir;
288 
289         if(tparam<tmin)
290         {
291                 returnvalue = 0;
292                 tparam = tmin;
293         }
294         else if(tparam>tmax)
295         {
296                 returnvalue = 0;
297                 tparam = tmax;
298         }
299 
300         VEC_SCALE(pout,tparam,vDir);
301         VEC_SUM(pout,vPoint,pout);
302         return returnvalue;
303 }
\end{verbatim}
\end{Code}




Here is the caller graph for this function:\hypertarget{gim__basic__geometry__operations_8h_c1357f9a99f5709463c3c11874c993c4}{
\index{gim\_\-basic\_\-geometry\_\-operations.h@{gim\_\-basic\_\-geometry\_\-operations.h}!PLANE\_\-CLIP\_\-SEGMENT2@{PLANE\_\-CLIP\_\-SEGMENT2}}
\index{PLANE\_\-CLIP\_\-SEGMENT2@{PLANE\_\-CLIP\_\-SEGMENT2}!gim_basic_geometry_operations.h@{gim\_\-basic\_\-geometry\_\-operations.h}}
\subsubsection[PLANE\_\-CLIP\_\-SEGMENT2]{\setlength{\rightskip}{0pt plus 5cm}template$<$typename CLASS\_\-POINT, typename CLASS\_\-PLANE$>$ SIMD\_\-FORCE\_\-INLINE eLINE\_\-PLANE\_\-INTERSECTION\_\-TYPE PLANE\_\-CLIP\_\-SEGMENT2 (const CLASS\_\-POINT \& {\em s1}, \/  const CLASS\_\-POINT \& {\em s2}, \/  const CLASS\_\-PLANE \& {\em plane}, \/  CLASS\_\-POINT \& {\em clipped})\hspace{0.3cm}{\tt  \mbox{[}inline\mbox{]}}}}
\label{gim__basic__geometry__operations_8h_c1357f9a99f5709463c3c11874c993c4}


Confirms if the plane intersect the edge or nor. 

intersection type must have the following values \begin{itemize}
\item 0 : Segment in front of plane, s1 closest \item 1 : Segment in front of plane, s2 closest \item 2 : Segment in back of plane, s1 closest \item 3 : Segment in back of plane, s2 closest \item 4 : Segment collides plane, s1 in back \item 5 : Segment collides plane, s2 in back \end{itemize}


Definition at line 156 of file gim\_\-basic\_\-geometry\_\-operations.h.

References VEC\_\-DIFF, VEC\_\-DOT, VEC\_\-SCALE, and VEC\_\-SUM.

Referenced by PLANE\_\-CLIP\_\-SEGMENT\_\-CLOSEST().

\begin{Code}\begin{verbatim}160 {
161         GREAL _dis1 = DISTANCE_PLANE_POINT(plane,s1);
162         GREAL _dis2 = DISTANCE_PLANE_POINT(plane,s2);
163         if(_dis1 >-G_EPSILON && _dis2 >-G_EPSILON)
164         {
165             if(_dis1<_dis2) return G_FRONT_PLANE_S1;
166             return G_FRONT_PLANE_S2;
167         }
168         else if(_dis1 <G_EPSILON && _dis2 <G_EPSILON)
169         {
170             if(_dis1>_dis2) return G_BACK_PLANE_S1;
171             return G_BACK_PLANE_S2;
172         }
173 
174         VEC_DIFF(clipped,s2,s1);
175         _dis2 = VEC_DOT(clipped,plane);
176         VEC_SCALE(clipped,-_dis1/_dis2,clipped);
177         VEC_SUM(clipped,clipped,s1);
178         if(_dis1<_dis2) return G_COLLIDE_PLANE_S1;
179         return G_COLLIDE_PLANE_S2;
180 }
\end{verbatim}
\end{Code}




Here is the caller graph for this function:\hypertarget{gim__basic__geometry__operations_8h_a931e17a661df001eaa717335ae537ec}{
\index{gim\_\-basic\_\-geometry\_\-operations.h@{gim\_\-basic\_\-geometry\_\-operations.h}!PLANE\_\-CLIP\_\-SEGMENT\_\-CLOSEST@{PLANE\_\-CLIP\_\-SEGMENT\_\-CLOSEST}}
\index{PLANE\_\-CLIP\_\-SEGMENT\_\-CLOSEST@{PLANE\_\-CLIP\_\-SEGMENT\_\-CLOSEST}!gim_basic_geometry_operations.h@{gim\_\-basic\_\-geometry\_\-operations.h}}
\subsubsection[PLANE\_\-CLIP\_\-SEGMENT\_\-CLOSEST]{\setlength{\rightskip}{0pt plus 5cm}template$<$typename CLASS\_\-POINT, typename CLASS\_\-PLANE$>$ SIMD\_\-FORCE\_\-INLINE eLINE\_\-PLANE\_\-INTERSECTION\_\-TYPE PLANE\_\-CLIP\_\-SEGMENT\_\-CLOSEST (const CLASS\_\-POINT \& {\em s1}, \/  const CLASS\_\-POINT \& {\em s2}, \/  const CLASS\_\-PLANE \& {\em plane}, \/  CLASS\_\-POINT \& {\em clipped1}, \/  CLASS\_\-POINT \& {\em clipped2})\hspace{0.3cm}{\tt  \mbox{[}inline\mbox{]}}}}
\label{gim__basic__geometry__operations_8h_a931e17a661df001eaa717335ae537ec}


Confirms if the plane intersect the edge or not. 

clipped1 and clipped2 are the vertices behind the plane. clipped1 is the closest

intersection\_\-type must have the following values \begin{itemize}
\item 0 : Segment in front of plane, s1 closest \item 1 : Segment in front of plane, s2 closest \item 2 : Segment in back of plane, s1 closest \item 3 : Segment in back of plane, s2 closest \item 4 : Segment collides plane, s1 in back \item 5 : Segment collides plane, s2 in back \end{itemize}


Definition at line 198 of file gim\_\-basic\_\-geometry\_\-operations.h.

References PLANE\_\-CLIP\_\-SEGMENT2(), and VEC\_\-COPY.

\begin{Code}\begin{verbatim}203 {
204         eLINE_PLANE_INTERSECTION_TYPE intersection_type = PLANE_CLIP_SEGMENT2(s1,s2,plane,clipped1);
205         switch(intersection_type)
206         {
207         case G_FRONT_PLANE_S1:
208                 VEC_COPY(clipped1,s1);
209             VEC_COPY(clipped2,s2);
210                 break;
211         case G_FRONT_PLANE_S2:
212                 VEC_COPY(clipped1,s2);
213             VEC_COPY(clipped2,s1);
214                 break;
215         case G_BACK_PLANE_S1:
216                 VEC_COPY(clipped1,s1);
217             VEC_COPY(clipped2,s2);
218                 break;
219         case G_BACK_PLANE_S2:
220                 VEC_COPY(clipped1,s2);
221             VEC_COPY(clipped2,s1);
222                 break;
223         case G_COLLIDE_PLANE_S1:
224                 VEC_COPY(clipped2,s1);
225                 break;
226         case G_COLLIDE_PLANE_S2:
227                 VEC_COPY(clipped2,s2);
228                 break;
229         }
230         return intersection_type;
231 }
\end{verbatim}
\end{Code}




Here is the call graph for this function:\hypertarget{gim__basic__geometry__operations_8h_76424e5fa2f0f7697652e0088e296397}{
\index{gim\_\-basic\_\-geometry\_\-operations.h@{gim\_\-basic\_\-geometry\_\-operations.h}!RAY\_\-PLANE\_\-COLLISION@{RAY\_\-PLANE\_\-COLLISION}}
\index{RAY\_\-PLANE\_\-COLLISION@{RAY\_\-PLANE\_\-COLLISION}!gim_basic_geometry_operations.h@{gim\_\-basic\_\-geometry\_\-operations.h}}
\subsubsection[RAY\_\-PLANE\_\-COLLISION]{\setlength{\rightskip}{0pt plus 5cm}template$<$typename T, typename CLASS\_\-POINT, typename CLASS\_\-PLANE$>$ SIMD\_\-FORCE\_\-INLINE bool RAY\_\-PLANE\_\-COLLISION (const CLASS\_\-PLANE \& {\em plane}, \/  const CLASS\_\-POINT \& {\em vDir}, \/  const CLASS\_\-POINT \& {\em vPoint}, \/  CLASS\_\-POINT \& {\em pout}, \/  T \& {\em tparam})\hspace{0.3cm}{\tt  \mbox{[}inline\mbox{]}}}}
\label{gim__basic__geometry__operations_8h_76424e5fa2f0f7697652e0088e296397}


Ray plane collision in one way. 

Intersects plane in one way only. The ray must face the plane (normals must be in opossite directions).\par
 It uses the PLANEDIREPSILON constant. 

Definition at line 243 of file gim\_\-basic\_\-geometry\_\-operations.h.

References VEC\_\-DOT, VEC\_\-SCALE, and VEC\_\-SUM.

\begin{Code}\begin{verbatim}248 {
249         GREAL _dis,_dotdir;
250         _dotdir = VEC_DOT(plane,vDir);
251         if(_dotdir<PLANEDIREPSILON)
252         {
253             return false;
254         }
255         _dis = DISTANCE_PLANE_POINT(plane,vPoint);
256         tparam = -_dis/_dotdir;
257         VEC_SCALE(pout,tparam,vDir);
258         VEC_SUM(pout,vPoint,pout);
259         return true;
260 }
\end{verbatim}
\end{Code}


